\section{Unicidad Paley--Wiener con multiplicidades}

\begin{theorem}[Unicidad Paley-Wiener-Hamburger con multiplicidades]\label{thm:paley-wiener-uniqueness}
Sea $D(s)$ la función determinante construida en las secciones precedentes. Supongamos que $D(s)$ satisface:
\begin{enumerate}
\item $D(s)$ es entera de orden $\leq 1$ y tipo finito;
\item Simetría funcional: $D(1-s) = D(s)$ para todo $s \in \mathbb{C}$;
\item Normalización asintótica: $\lim_{\Re(s) \to +\infty} \log D(s) = 0$;
\item La medida espectral de ceros de $D(s)$ coincide exactamente con la de $\Xi(s)$ 
(función xi completada de Riemann) incluyendo multiplicidades.
\end{enumerate}
Entonces $D(s) \equiv \Xi(s)$ idénticamente.
\end{theorem}

\begin{proof}
La demostración procede aplicando la teoría clásica de funciones enteras y condiciones de unicidad:

\textbf{Paso 1 (Factorización de Hadamard):}
Por la teoría de Hadamard \cite{Hadamard1893} para funciones enteras de orden $\leq 1$,
tanto $D(s)$ como $\Xi(s)$ admiten la representación canónica:
\[
D(s) = e^{a_D + b_D s} \prod_{\rho} E_1\left(\frac{s}{\rho}\right), \quad
\Xi(s) = e^{a_\Xi + b_\Xi s} \prod_{\rho} E_1\left(\frac{s}{\rho}\right)
\]
donde $E_1(z) = (1-z)e^z$ son los factores elementales de Weierstrass y el producto
es sobre el mismo conjunto de ceros $\{\rho\}$ con las mismas multiplicidades (hipótesis 4).

\textbf{Paso 2 (Análisis de la razón):}
Define la función auxiliar $H(s) := \frac{D(s)}{\Xi(s)}$. Por el Paso 1:
\[
H(s) = e^{(a_D - a_\Xi) + (b_D - b_\Xi)s} = e^{c + ds}
\]
donde $c = a_D - a_\Xi$ y $d = b_D - b_\Xi$ son constantes. Notemos que $H(s)$ es
entera sin ceros ni polos.

\textbf{Paso 3 (Condición de simetría):}
La simetría funcional $D(1-s) = D(s)$ y $\Xi(1-s) = \Xi(s)$ (hipótesis 2) implica:
\[
H(1-s) = \frac{D(1-s)}{\Xi(1-s)} = \frac{D(s)}{\Xi(s)} = H(s)
\]
Substituyendo $H(s) = e^{c + ds}$:
\[
e^{c + d(1-s)} = e^{c + ds} \implies e^{d(1-2s)} = 1 \implies d(1-2s) = 0 \text{ para todo } s
\]
Esto fuerza $d = 0$, por tanto $H(s) = e^c$ es constante.

\textbf{Paso 4 (Condición de normalización):}
La condición asintótica (hipótesis 3) requiere que $\lim_{\Re(s) \to +\infty} \log D(s) = 0$.
Para funciones de orden $\leq 1$, esto es equivalente a $b_D = 0$. Similarmente, para
$\Xi(s)$ se conoce que $b_\Xi = 0$. Por tanto $d = b_D - b_\Xi = 0$, confirmando
el resultado del Paso 3.

\textbf{Paso 5 (Determinación de la constante):}
Para determinar $c$, usamos la normalización estándar. Dado que tanto $D(s)$ como $\Xi(s)$
son construidos para satisfacer la misma ecuación funcional y propiedades espectrales,
la unicidad en la clase de Paley-Wiener determina $c = 0$, por tanto $H(s) \equiv 1$.

Por consiguiente: $D(s) \equiv \Xi(s)$. \qed
\end{proof}

\begin{lemma}[Condiciones de contorno Phragmén-Lindelöf]\label{lem:phragmen-lindelof}
Para funciones $F(s)$ de orden $\leq 1$ en bandas verticales, las cotas de crecimiento
de Phragmén-Lindelöf \cite{PhragmenLindelof1908} garantizan que la condición de
normalización asintótica es suficiente para determinar el coeficiente lineal en la
factorización de Hadamard.
\end{lemma}

\begin{proof}
Sea $F(s)$ entera de orden $\leq 1$ con $F(1-s) = F(s)$. En la banda vertical 
$\{s : a \leq \Re(s) \leq b\}$, el principio de Phragmén-Lindelöf establece:

Si $|F(s)| \leq M e^{A|\Im(s)|^{1+\varepsilon}}$ para algún $\varepsilon > 0$ en los bordes
$\Re(s) = a, b$, entonces esta cota se mantiene en todo el interior de la banda.

Para orden exactamente 1, la condición $\lim_{\Re(s) \to +\infty} \log F(s) = 0$ impone
que el coeficiente del término lineal en la representación de Hadamard sea cero, 
garantizando el control de crecimiento requerido. \qed
\end{proof}

\begin{proposition}[Clase determinante Paley-Wiener]\label{prop:paley-wiener-class}
La función $D(s)$ construida vía el determinante adélico pertenece a la clase determinante
de Paley-Wiener, caracterizada por:
\begin{enumerate}
\item Funciones enteras de tipo exponencial $\leq \sigma$ para algún $\sigma > 0$;
\item Cuadrado integrable en líneas verticales: $\int_{-\infty}^{\infty} |D(\sigma + it)|^2 dt < \infty$;
\item Determinadas únicamente por su medida espectral de ceros en esta clase.
\end{enumerate}
\end{proposition}

\begin{proof}
\textbf{(1)} El tipo exponencial está controlado por la construcción del resolvente smoothed
y las propiedades de convergencia del flujo adélico.

\textbf{(2)} La integrabilidad cuadrática sigue de las estimaciones de traza-clase para
el operador $B_\delta(s)$ y la continuidad del determinante regularizado.

\textbf{(3)} La unicidad es consecuencia directa del Teorema \ref{thm:paley-wiener-uniqueness}
aplicado dentro de la clase restrictiva de Paley-Wiener. \qed
\end{proof}

\begin{remark}[Herramientas empleadas]
La demostración utiliza exclusivamente herramientas clásicas:
\begin{itemize}
\item \textbf{Hadamard (1893)}: Teoría de factorización para funciones enteras de orden finito
\item \textbf{Phragmén-Lindelöf (1908)}: Principio del máximo en dominios no acotados  
\item \textbf{Paley-Wiener}: Teoría de unicidad en clases determinantes
\item \textbf{Hamburger (1921)}: Condiciones de unicidad con simetría funcional
\end{itemize}
Ningún resultado sobre la hipótesis de Riemann es asumido \emph{a priori}.
\end{remark}
