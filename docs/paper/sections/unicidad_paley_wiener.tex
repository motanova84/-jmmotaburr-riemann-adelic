\section{Unicidad Paley--Wiener con multiplicidades}

Establecemos que las propiedades analíticas básicas (orden, simetría, divisor
de ceros y normalización) determinan $D(s)$ de forma única.

\begin{lemma}[Unicidad]\label{lem:paper-uniqueness}
Sea $F$ una función entera de orden $\leqslant 1$ y tipo finito que satisface
$F(s)=F(1-s)$.  Si el divisor de ceros de $F$ coincide con el de $\Xi(s)$ e
$F(1/2)=\Xi(1/2)$, entonces $F\equiv \Xi$.
\end{lemma}

\begin{proof}
Por la factorización de Hadamard
\cite[Chap.~II]{Tate1967}, el cociente $H(s)=F(s)/\Xi(s)$ es una función entera
sin ceros.  La simetría implica $H(s)=H(1-s)$, de modo que $h(s)=\log H(s)$ es
entera con crecimiento lineal controlado.  El teorema de
Paley--Wiener--Hamburger
\cite[Thm.~5]{Hamburger1921}
identifica $h$ como transformada de Fourier de una medida compactamente
soportada.  La normalización $H(1/2)=1$ obliga a que la medida tenga masa total
nula; si fuese no trivial, $h$ crecería linealmente en alguna dirección
imaginaria, contradiciendo el crecimiento de orden $\leqslant1$.  Por tanto,
$h\equiv0$ y $F=\Xi$.
\end{proof}

Este lema excluye soluciones ``exóticas'': cualquier función entera con las
propiedades postuladas coincide con la función de Riemann completada.
