\section{Teorema de rigidez arquimediana}

\paragraph{Estado actual.}
El enunciado depende de completar los cálculos presentados en la
Sección~\ref{thm:paper-weil} y en la Proposición~\ref{prop:paper-stationary}; la
documentación actual recoge únicamente la estrategia.  La prueba definitiva debe
seguir de los entregables P3.1--P3.2.

\begin{theorem}
Sea $D(s)$ una función entera de orden $\leqslant 1$ con simetría funcional
$D(1-s)=D(s)$ y factores locales normalizados por el índice de Weil.
Entonces el factor local en $\mathbb{R}$ debe ser $\pi^{-s/2}\Gamma(s/2)$.
\end{theorem}

\begin{proof}
El argumento combina el cálculo explícito del Teorema~\ref{thm:paper-weil} con la
ley de producto del índice de Weil \cite[Cor.~2]{Weil1964}.  Cualquier otra
normalización en el lugar infinito violaría esa ley, puesto que los factores
finitos ya están fijados por la construcción S-finita.  La
Proposición~\ref{prop:paper-stationary} refuerza la unicidad al reproducir el
mismo factor mediante fase estacionaria.
\end{proof}
