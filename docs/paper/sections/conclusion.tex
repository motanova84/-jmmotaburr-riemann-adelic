We have presented a complete conditional resolution of the Riemann Hypothesis through the lens of S-finite adelic spectral systems. The key insight is that the canonical determinant $D(s)$, constructed purely from operator-theoretic principles, naturally embodies the essential analytical properties required for the proof.

\subsection{Summary of Results}

Our main results can be summarized as follows:

\begin{theorem}[Main Result]
Under the S-finite axioms and spectral regularity conditions established in this framework, all non-trivial zeros of the Riemann zeta function $\zeta(s)$ lie on the critical line $\Re(s) = 1/2$.
\end{theorem}

The proof strategy combines:
\begin{itemize}
\item The rigorous construction of $D(s)$ from scale-invariant flows
\item The Paley-Wiener uniqueness theorem ensuring $D(s) \equiv \Xi(s)$
\item Dual verification through both de Branges theory and Weil-Guinand positivity
\end{itemize}

\subsection{Significance and Impact}

This work demonstrates that the Riemann Hypothesis emerges naturally from spectral-theoretic considerations when properly formulated in the adelic setting. The approach avoids many of the traditional difficulties by working directly with the completed zeta function rather than attempting to analyze the classical Euler product.

\subsection{Future Directions}

Several avenues for further development emerge from this work:

\begin{enumerate}
\item \textbf{Computational Verification}: Extensive numerical validation of the spectral framework for large ranges of zeros.
\item \textbf{Generalization}: Extension to other L-functions and automorphic forms.
\item \textbf{Effective Bounds}: Derivation of explicit constants and error terms in the asymptotic estimates.
\end{enumerate}

\subsection{Final Remarks}

This conditional proof is offered to the mathematical community for rigorous scrutiny. While the framework presented here provides a novel and mathematically consistent approach to the Riemann Hypothesis, the ultimate validation rests on detailed verification of the S-finite axioms and their consequences.

The complete computational implementation, numerical data, and detailed technical appendices ensure full transparency and reproducibility of all claims made in this work.