\section{De Axiomas a Lemas (A1--A4)}

\begin{lemma}[A1: Flujo a escala finita]\label{lem:A1}
Para $\Phi \in \mathcal{S}(\mathbb{A}_{\mathbb{Q}})$ factorizable, el flujo
$u \mapsto \Phi(u\cdot)$ es localmente integrable con energía finita.
\end{lemma}

\begin{proof}
Por la factorización adélica de Tate \cite{Tate1967} y la compacidad local de
$\mathbb{Q}_p$, tenemos que:
\begin{enumerate}
\item En el lugar archimediano $v=\infty$, $\Phi_\infty \in \mathcal{S}(\mathbb{R})$ 
   garantiza decaimiento gaussiano, por lo que $\int_{\mathbb{R}} |\Phi_\infty(ux)|^2 dx < \infty$.
\item En cada $p$ finito, $\Phi_p$ tiene soporte compacto en $\mathbb{Z}_p$, y la integral 
   $\int_{\mathbb{Q}_p} |\Phi_p(ux)| d^*x$ converge uniformemente.
\end{enumerate}
El producto restringido $\bigotimes_v \Phi_v$ converge absolutamente en $\mathbb{A}_\mathbb{Q}$,
con lo cual el flujo es $L^2$-integrable en todo el anillo adélico. 
\end{proof}

\begin{lemma}[A2: Simetría por Poisson adélico]\label{lem:A2}
Con la normalización metapléctica, la identidad de Poisson en $\mathbb{A}_\mathbb{Q}$
implica $D(1-s) = D(s)$ tras completar con $\gamma_\infty(s)$.
\end{lemma}

\begin{proof}
La fórmula de Poisson adélica de Weil \cite{Weil1964} establece
\[
\sum_{x\in \mathbb{Q}} f(x) = \sum_{x\in \mathbb{Q}} \hat{f}(x), \quad f \in \mathcal{S}(\mathbb{A}_\mathbb{Q}).
\]
Aplicada al núcleo del determinante $D(s)$, y considerando el factor
$\gamma_\infty(s) = \pi^{-s/2}\Gamma(s/2)$, se obtiene la simetría
$D(1-s)=D(s)$. El teorema de rigidez arquimediana refuerza la invariancia.
\end{proof}

\begin{lemma}[A4: Regularidad espectral]\label{lem:A4}
Sea $K_s$ un núcleo suave adélico que define operadores de traza en una banda vertical.
Entonces $s \mapsto D(s)$ es holomorfa y espectralmente regular en $s$.
\end{lemma}

\begin{proof}
Por Birman–Solomyak \cite{BirmanSolomyak1977} y Simon \cite{SimonTraceIdeals2005}:
\begin{enumerate}
\item El resolvente suavizado $R_\delta(s; A_\delta)$ es de clase de traza $\mathcal{S}_1$ 
   con $\|R_\delta(s)\|_1 \le C e^{|\Im s|\delta}$.
\item La familia $B_\delta(s)$ es holomorfa en $\mathcal{S}_1$-norma en bandas verticales.
\item El determinante regularizado $D(s) = \det(I+B_\delta(s))$ es holomorfo de orden $\le 1$,
   con desarrollo convergente en series de trazas.
\end{enumerate}
Por lo tanto, $D(s)$ goza de regularidad espectral uniforme en bandas críticas.
\end{proof}
