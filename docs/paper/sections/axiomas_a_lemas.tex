\section{De Axiomas a Lemas (A1--A4)}

\subsection{Factorización Explícita de Schwartz-Bruhat}

Para toda función $\Phi \in \mathcal{S}(\mathbb{A}_\mathbb{Q})$ en el espacio de Schwartz-Bruhat sobre los adeles racionales, existe una factorización tensorial explícita:
\[
\Phi = \bigotimes_{v} \Phi_v
\]
donde $v$ recorre todos los lugares de $\mathbb{Q}$ (el lugar arquimediano $v = \infty$ y los lugares finitos $v = p$ para primos $p$).

\begin{proposition}[Factorización local con soporte finito]
Para $\Phi \in \mathcal{S}(\mathbb{A}_\mathbb{Q})$, existen:
\begin{itemize}
\item $\Phi_\infty \in \mathcal{S}(\mathbb{R})$ de decaimiento rápido
\item $\Phi_p \in \mathcal{S}(\mathbb{Q}_p)$ con soporte compacto para un número finito de primos $p$
\item Para casi todo primo $p$: $\Phi_p = \mathbf{1}_{\mathbb{Z}_p}$ (función característica)
\end{itemize}
\end{proposition}

\subsection{Límites de Convergencia con Integrales}

Los límites de convergencia se calculan mediante integrales sobre $\mathbb{R}$ y $\mathbb{Q}_p$:

\textbf{Límite arquimediano:}
\[
\lim_{T \to \infty} \int_{-T}^{T} \Phi_\infty(x) |x|^s d^\times x = \int_{\mathbb{R}^\times} \Phi_\infty(x) |x|^s d^\times x
\]

\textbf{Límites p-ádicos:}
\[
\lim_{n \to \infty} \int_{p^{-n}\mathbb{Z}_p} \Phi_p(x) |x|_p^s d^\times x = \int_{\mathbb{Q}_p^\times} \Phi_p(x) |x|_p^s d^\times x
\]

donde $|\cdot|_p$ denota la valuación p-ádica normalizada.

\begin{lemma}[A1: Flujo a Escala Finita]
Para $\Phi\in\mathcal S(\mathbb{A}_\mathbb{Q})$ factorizable, el flujo $u\mapsto \Phi(u\cdot)$
es localmente integrable con energía finita. En particular, A1 es consecuencia del
decaimiento gaussiano en $\mathbb{R}$ y la compacidad en $\mathbb{Q}_p$.

\begin{proof}
La integrabilidad local sigue de la factorización $\Phi = \bigotimes_v \Phi_v$ y las propiedades:
\begin{enumerate}
\item En $v = \infty$: $\Phi_\infty \in \mathcal{S}(\mathbb{R})$ tiene decaimiento más rápido que cualquier polinomio
\item En $v = p$: $\Phi_p$ tiene soporte compacto en $\mathbb{Q}_p$, luego es integrable
\item La energía total es el producto de las energías locales, todas finitas
\end{enumerate}
\end{proof}
\end{lemma}

\begin{lemma}[A2: Simetría por Poisson Adélico]
\label{lem:A2-symmetry}

\textbf{Identidad de Poisson en $\mathbb{A}_\mathbb{Q}$ paso a paso:}

\emph{Paso 1: Fórmula de Poisson global}
\[
\sum_{x \in \mathbb{Q}} \Phi(x) = \sum_{x \in \mathbb{Q}} \widehat{\Phi}(x)
\]

\emph{Paso 2: Descomposición local por lugares}
\[
\Phi(x) = \prod_{v} \Phi_v(x_v), \quad \widehat{\Phi}(x) = \prod_{v} \widehat{\Phi_v}(x_v)
\]

\emph{Paso 3: Ecuación funcional local para integrales de Tate}
Para cada lugar $v$:
\[
Z_v(\Phi_v, s) = \int_{\mathbb{Q}_v^\times} \Phi_v(x) |x|_v^s d^\times x = \gamma_v(s) \cdot Z_v(\widehat{\Phi_v}, 1-s)
\]

\emph{Paso 4: Operador J efectivo}
El operador $J$ actúa como transformada de Fourier adélica:
\[
J: \mathcal{S}(\mathbb{A}_\mathbb{Q}) \to \mathcal{S}(\mathbb{A}_\mathbb{Q}), \quad (J\Phi)(x) = \widehat{\Phi}(x)
\]

Con la normalización metapléctica (índice de Weil \cite{Weil1964}), se obtiene:
\[
J^2 = \text{Id} \quad \text{y} \quad J \circ T_s = T_{1-s} \circ J
\]
donde $T_s$ denota la acción de escalamiento $(T_s \Phi)(x) = |x|^s \Phi(x)$.

\emph{Paso 5: Simetría efectiva $D(1-s) = D(s)$}
La función canónica $D(s)$ se define como:
\[
D(s) = \prod_v Z_v(\Phi_v, s) = \prod_v \gamma_v(s) \cdot \prod_v Z_v(\widehat{\Phi_v}, 1-s)
\]

Por la ley de producto de Weil: $\prod_v \gamma_v(s) = 1$, obtenemos:
\[
D(s) = \prod_v Z_v(\widehat{\Phi_v}, 1-s) = D(1-s)
\]

\begin{proof}
La simetría se establece mediante el Teorema de Rigidez Arquimediana (véase \cite{Weil1964} y Sección~\ref{sec:rigidez}). El operador $J$ preserva la clase de funciones y la normalización metapléctica garantiza que la transformación $s \mapsto 1-s$ preserve el determinante canónico.
\end{proof}
\end{lemma}

\begin{lemma}[A4: Regularidad Espectral via Birman-Solomyak]
\label{lem:A4-spectral}

Utilizando los resultados de las familias de operadores de Birman-Solomyak \cite{BirmanSolomyak1967} y Simon \cite{simon2005}:

\emph{Convergencia explícita vía serie de Lidskii:}
Sea $K_s$ un núcleo suave adélico que define operadores de clase traza. El logaritmo del determinante admite la expansión:
\[
\log D(s) = \sum_{j=1}^\infty \frac{\text{tr}(K_s^j)}{j}
\]

donde la convergencia es uniforme en bandas verticales $|\text{Im}(s)| \leq T$ para cualquier $T > 0$.

\emph{Regularidad uniforme en bandas verticales:}
Por los resultados de Simon sobre ideales de traza, para operadores $K_s$ en clase de Schatten $\mathcal{S}_1$:
\[
\|K_s - K_{s'}\|_1 \leq C |s - s'| \quad \text{para } |\text{Im}(s)|, |\text{Im}(s')| \leq T
\]

Esto implica continuidad uniforme de $s \mapsto \log D(s)$ en bandas verticales.

\begin{proof}
La continuidad en traza sigue del teorema de Birman-Solomyak sobre la continuidad de funciones de operadores en ideales de von Neumann. La regularidad espectral uniforme en $s$ es consecuencia directa de la teoría de perturbaciones para operadores compactos auto-adjuntos.
\end{proof}
\end{lemma}
