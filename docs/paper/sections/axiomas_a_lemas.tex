\section{De Axiomas a Lemas (A1--A4)}

Trabajamos en el anillo de adeles $\mathbb{A}_\mathbb{Q}$ con la medida de Haar
$dx$ y multiplicativa $d^{\times}x$.  Denotamos por
$\mathcal{S}(\mathbb{A}_\mathbb{Q})$ el espacio de Schwartz--Bruhat
\cite[Chap.~I]{Tate1967}.  El objetivo es mostrar que las condiciones A1--A4
introducidas en la construcción de $D(s)$ se desprenden de resultados estándar.

\begin{lemma}[A1: flujo a escala finita]\label{lem:A1-paper}
Para $\Phi\in\mathcal{S}(\mathbb{A}_\mathbb{Q})$ factorizable como
$\Phi=\prod_v \Phi_v$, el flujo $u\mapsto\Phi(u\cdot)$ es localmente integrable con
energía finita.  En particular, A1 es consecuencia del decaimiento gaussiano en
$\mathbb{R}$ y la compacidad en $\mathbb{Q}_p$.
\end{lemma}

\begin{proof}
La descripción de $\mathcal{S}(\mathbb{A}_\mathbb{Q})$ como producto restringido
\cite[Prop.~2]{Tate1967} garantiza que $\Phi_\infty\in\mathcal{S}(\mathbb{R})$ y que
$\Phi_p$ es compactamente soportada para casi todo $p$.  Dado un compacto
$U\subset\mathbb{A}_\mathbb{Q}^{\times}$, la integral

\[
  \int_U\!\int_{\mathbb{A}_\mathbb{Q}} |\Phi(u x)|^2\,dx\,d^{\times}u
\]

se separa como un producto de integrales locales acotadas por un factor
$C_U$, gracias al decaimiento gaussiano en el lugar infinito y a la compacidad
$p$-ádica.  Por tanto, el flujo posee energía finita.
\end{proof}

\begin{lemma}[A2: simetría por Poisson adélico]\label{lem:A2-paper}
Sea $Z(\Phi,s)$ la transformada de Mellin de Tate asociada a $\Phi$.  Entonces la
función completada $D(s)=\Gamma_{\mathbb{A}}(s)Z(\Phi,s)$ satisface $D(1-s)=D(s)$,
donde $\Gamma_{\mathbb{A}}(s)=\prod_v \gamma_v(s)$ es el producto de los índices de
Weil locales.
\end{lemma}

\begin{proof}
La identidad de Poisson adélica
\cite[Thm.~2]{Tate1967}
implica que $Z(\widehat{\Phi},1-s)=Z(\Phi,s)$ siempre que la transformada local se
normalice mediante $\gamma_v$ \cite[§II.3]{Weil1964}.  La ley de producto
$\prod_v\gamma_v(s)=1$ asegura que $\Gamma_{\mathbb{A}}(1-s)=\Gamma_{\mathbb{A}}(s)$, y la
identidad $D(1-s)=D(s)$ sigue.
\end{proof}

\begin{lemma}[A4: regularidad espectral]\label{lem:A4-paper}
Sea $T_s$ el operador integral en $L^2(\mathbb{A}_\mathbb{Q})$

\[
  (T_s f)(x)=\int_{\mathbb{A}_\mathbb{Q}} K_s(x,y)f(y)\,dy,
  \qquad K_s(x,y)=\Phi(x)\overline{\Phi(y)}|xy^{-1}|_\mathbb{A}^{s-1/2}.
\]

Entonces $T_s$ es de traza, depende holomórficamente de $s$ en bandas verticales
y su espectro varía continuamente.
\end{lemma}

\begin{proof}
Para $\Re(s)=\tfrac{1}{2}$ el núcleo $K_s$ pertenece a $L^2(\mathbb{A}_\mathbb{Q}^2)$, de
modo que $T_s$ es de Hilbert--Schmidt.  Las estimaciones de crecimiento de $\Phi$
y $|xy^{-1}|^{\sigma}$ implican que $\|K_s\|_{L^2}$ depende holomórficamente de $s$
en bandas verticales acotadas.  El teorema de Birman--Solomyak sobre
familias holomorfas de operadores integrales
\cite[Thm.~1]{BirmanSolomyak1967}
proporciona la continuidad espectral requerida.
\end{proof}

De este modo, los axiomas A1--A4 quedan reincorporados al cuerpo de resultados
adélicos clásicos.
