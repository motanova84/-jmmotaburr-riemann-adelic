\section{De Axiomas a Lemas (A1--A4)}

\newtheorem{theoremE}{Theorem}[section]
\newtheorem{lemmaE}[theoremE]{Lemma}

\begin{lemmaE}[A1: flujo a escala finita]
Para $\Phi\in\mathcal S(\Bbb A_\Bbb Q)$ factorizable, el flujo $u\mapsto \Phi(u\cdot)$
es localmente integrable con energía finita. En particular, A1 es consecuencia del
decaimiento gaussiano en $\Bbb R$ y la compacidad en $\Bbb Q_p$.
\end{lemmaE}

\begin{lemmaE}[A2: simetría por Poisson adélico]
Con la normalización metapléctica, la identidad de Poisson en $\Bbb A_\Bbb Q$
induce $D(1-s)=D(s)$ tras completar con $\gamma_\infty(s)$ (Teorema de rigidez).
\end{lemmaE}

\begin{lemmaE}[A4: regularidad espectral]
Sea $K_s$ un núcleo suave adélico que define operadores de traza en una banda vertical.
La continuidad en traza y el resultado de Birman--Solomyak implican regularidad
espectral uniforme en $s$, estableciendo A4.
\end{lemmaE}
