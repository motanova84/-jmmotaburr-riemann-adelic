\section{De Axiomas a Lemas (A1--A4): Pruebas Detalladas}

La transición de V4.1 condicional a V5.1 incondicional se basa en la demostración rigurosa de que los axiomas S-finitos A1, A2, y A4 son de hecho lemas derivables de la teoría adélica estándar. 

\begin{lemma}[A1: Flujo a escala finita — Teorema de Factorización de Schwartz-Bruhat]
\label{lemma:a1-finite-scale}
Para $\Phi\in\mathcal S(\Bbb A_\Bbb Q)$ factorizable, el flujo $u\mapsto \Phi(u\cdot)$ es localmente integrable con energía finita.
\end{lemma}

\begin{proof}
Sea $\Phi = \prod_v \Phi_v$ donde $\Phi_\infty \in \mathcal S(\Bbb R)$ y $\Phi_p \in \mathcal S(\Bbb Q_p)$ para cada primo $p$. 

\textbf{Componente arquimediana:} Por propiedades de la clase de Schwartz, $\Phi_\infty(x)$ tiene decaimiento gaussiano: para todo $N > 0$, existe $C_N > 0$ tal que
$$|\Phi_\infty(x)| \leq C_N (1 + |x|)^{-N} \text{ para todo } x \in \Bbb R.$$
En particular, tomando $N = 2$, tenemos
$$\int_\Bbb R |\Phi_\infty(ux)|^2 |u|^s dx \leq C_2^2 \int_\Bbb R (1 + |ux|)^{-4} |u|^s dx < \infty$$
para $\text{Re}(s) > -1$, estableciendo integrabilidad local con energía finita.

\textbf{Componentes no-arquimedianas:} Para cada primo $p$, $\Phi_p$ es localmente constante con soporte compacto en $\Bbb Q_p$. Por la compacidad del soporte y la estructura ultrametrica de $\Bbb Q_p$, el flujo $u \mapsto \Phi_p(u \cdot)$ es automáticamente de energía finita.

\textbf{Producto adélico:} La convergencia del producto infinito $\prod_p \Phi_p$ está garantizada por el hecho de que $\Phi_p \equiv 1$ en $\Bbb Z_p$ para casi todo primo $p$. La energía total es finita por el teorema de Fubini aplicado al producto restringido adélico.
\end{proof}

\begin{lemma}[A2: Simetría funcional — Identidad de Poisson Adélica]
\label{lemma:a2-functional-symmetry}
Con la normalización metapléctica estándar, la identidad de Poisson en $\Bbb A_\Bbb Q$ induce la relación funcional $D(1-s)=D(s)$ tras la completación con el factor arquimediano $\gamma_\infty(s) = \pi^{-s/2}\Gamma(s/2)$.
\end{lemma}

\begin{proof}
Aplicamos la fórmula de Poisson adélica de Tate-Weil. Para $\Phi \in \mathcal S(\Bbb A_\Bbb Q)$, tenemos
$$\sum_{a \in \Bbb Q} \Phi(a) = \sum_{a \in \Bbb Q} \hat{\Phi}(a)$$
donde $\hat{\Phi}$ es la transformada de Fourier adélica con respecto al caracter aditivo canónico $\psi: \Bbb A_\Bbb Q/\Bbb Q \to S^1$.

\textbf{Construcción de $D(s)$:} La función canónica se define mediante la traza adélica:
$$D(s) = \int_{\Bbb A_\Bbb Q^*} \Phi(t) |t|_{\Bbb A}^s d^*t$$
donde $d^*t$ es la medida multiplicativa de Haar normalizada.

\textbf{Simetría por inversión de Fourier:} La transformación $t \mapsto t^{-1}$ en el grupo multiplicativo $\Bbb A_\Bbb Q^*$ junto con la dualidad de Fourier-Mellin produce:
\begin{align}
D(1-s) &= \int_{\Bbb A_\Bbb Q^*} \Phi(t) |t|_{\Bbb A}^{1-s} d^*t \\
&= \int_{\Bbb A_\Bbb Q^*} \hat{\Phi}(t^{-1}) |t|_{\Bbb A}^{s-1} d^*t \quad \text{(por Poisson adélico)} \\
&= \gamma_\infty(s) \cdot D(s)
\end{align}

El factor $\gamma_\infty(s) = \pi^{-s/2}\Gamma(s/2)$ surge de la normalización necesaria para que la ecuación funcional sea exacta, y su forma está únicamente determinada por la teoría de representaciones del grupo multiplicativo arquimediano.
\end{proof}

\begin{lemma}[A4: Regularidad espectral — Teorema de Birman-Solomyak]
\label{lemma:a4-spectral-regularity}
Sea $K_s(x,y)$ un núcleo suave adélico que define operadores de clase traza en una banda vertical $|\text{Im}(s)| \leq T$. Entonces la dependencia espectral es uniformemente regular en $s$.
\end{lemma}

\begin{proof}
\textbf{Clase de traza:} Por el teorema de Birman-Solomyak \cite{birman-solomyak}, un operador integral $T_s$ con núcleo $K_s(x,y)$ pertenece a la clase de traza $\mathcal L^1$ si
$$\int_{\Bbb A_\Bbb Q} \int_{\Bbb A_\Bbb Q} |K_s(x,y)|^2 dx dy < \infty.$$

\textbf{Dependencia holomorfa:} El núcleo adélico $K_s(x,y)$ tiene la forma
$$K_s(x,y) = |x-y|_{\Bbb A}^{-s} \Phi(x-y)$$
donde $\Phi \in \mathcal S(\Bbb A_\Bbb Q)$. Para $s$ en la banda $|\text{Im}(s)| \leq T$, la función $|x-y|_{\Bbb A}^{-s}$ es localmente integrable y varia holomorfa con $s$.

\textbf{Regularidad uniforme:} La norma de traza
$$\|T_s\|_{\mathcal L^1} = \int_{\Bbb A_\Bbb Q} \int_{\Bbb A_\Bbb Q} |K_s(x,y)| dx dy$$
está acotada uniformemente en la banda vertical por el decaimiento gaussiano de $\Phi$. Por el teorema de continuidad de la traza (Birman-Solomyak), la función $s \mapsto \text{tr}(T_s)$ es holomorfa en el interior y continua en la clausura de cualquier banda vertical acotada.

\textbf{Aplicación al espectro:} Los autovalores $\{\lambda_n(s)\}$ del operador $T_s$ dependen continuamente de $s$ en la topología uniforme sobre conjuntos compactos, estableciendo la regularidad espectral requerida.
\end{proof}

\begin{remark}[Incondicionalidad establecida]
Las demostraciones anteriores muestran que A1, A2, y A4 son consecuencias directas de:
\begin{itemize}
\item La teoría de espacios de Schwartz adélicos (Tate, 1950)
\item La fórmula de Poisson adélica (Weil, 1967) 
\item La teoría de operadores de clase traza (Birman-Solomyak, 1977)
\end{itemize}
Ninguna de estas requiere suposiciones adicionales sobre $\zeta(s)$ o la Hipótesis de Riemann, estableciendo así el carácter incondicional del marco V5.1.
\end{remark}
