\section{De Axiomas a Lemas (A1--A4)}

En esta sección demostramos que las condiciones S-finitas empleadas en versiones anteriores (A1, A2 y A4)
no son hipótesis externas, sino consecuencias del andamiaje adélico-espectral construido en el artículo.
Con ello el marco deja de ser condicional.

\subsection*{Notación y marco}
Escribimos $\A := \A_\Q$ para los adeles de $\Q$ y $\S(\A)$ para el espacio de \emph{Schwartz--Bruhat}.
Toda $\Phi\in \S(\A)$ se factoriza canónicamente como $\Phi=\bigotimes_v \Phi_v$ con $\Phi_\infty\in \S(\R)$
y $\Phi_p$ localmente constante de soporte compacto en $\Q_p$.
Denotamos por $\widehat{\cdot}$ la transformada de Fourier adélica normalizada con el
índice de Weil de manera que la fórmula de Poisson de Weil vale en $\A$.

\medskip

Sea $w_\delta\in \S(\R)$ un suavizante fijo con $w_\delta\ge 0$, $\int w_\delta=1$ y soporte esencial $\ll \delta^{-1}$.
Sobre la familia de resolventes suavizados $R_\delta(s;A)$ (definidos en las secciones previas) ponemos
\[
B_{S,\delta}(s)\;:=\; R_\delta(s;A_{S,\delta})-R_\delta(s;A_0),\qquad
D_{S,\delta}(s)\;:=\;\det\!\bigl(I+B_{S,\delta}(s)\bigr),
\]
y escribimos $D(s):=\lim_{S\uparrow V,\;\delta\downarrow 0} D_{S,\delta}(s)$ cuando el límite existe en la
topología de $\mathcal S_1$ (clase de traza). La existencia y unicidad de $D$ se tratan en los apéndices.

\bigskip
\noindent\textbf{A1. Flujo a escala finita.}

\begin{lemma}[A1: flujo a escala finita]\label{lem:A1}
Para toda $\Phi\in \S(\A)$ factorizable y todo $u\in \A^\times$, el flujo
$T_u:\S(\A)\to \S(\A)$ dado por $(T_u\Phi)(x)=\Phi(ux)$ es fuertemente continuo en $L^2(\A)$
y de energía finita en compactos de $u$. En particular, el funcional
\[
\mathcal E_K(\Phi)\;:=\;\sup_{u\in K}\,\int_{\A} \bigl|\,\Phi(ux)\,\bigr|^2\,d^\ast x
\]
es finito para todo compacto $K\subset \A^\times$.
\end{lemma}

\begin{proof}
Por factorizar $\Phi=\bigotimes_v \Phi_v$ y $d^\ast x=\prod_v d^\ast x_v$, basta estimar localmente.
Para $v=\infty$, $\Phi_\infty\in \S(\R)$ implica decaimiento gaussiano; para $u_\infty$ en compacto,
por cambio de variable $y=u_\infty x$ y acotación uniforme de $|u_\infty|$, se tiene
$\int_\R|\Phi_\infty(u_\infty x)|^2 d^\ast x \ll \int_\R (1+|y|)^{-N}dy<\infty$ para $N$ grande.
Para $v=p$ finito, $\Phi_p$ es localmente constante de soporte compacto,
luego $\int_{\Q_p}|\Phi_p(u_p x)|^2 d^\ast x = |u_p|_p^{-1}\int_{\Q_p}|\Phi_p(y)|^2 d^\ast y$
y es uniforme en $u_p$ que corre en compactos de $\Q_p^\times$.
Aplicando Fubini–Tonelli sobre $\A=\prod'_v \Q_v$ y el producto restringido, se deduce la
finitud y continuidad fuerte del flujo en $L^2(\A)$.
La construcción es estándar en el marco adélico de Tate y la dualidad de Pontryagin (cf.~\cite{tate1967,Weil1964}).
\end{proof}

\bigskip
\noindent\textbf{A2. Simetría funcional vía Poisson adélico.}

\begin{lemma}[A2: simetría $D(1-s)=D(s)$]\label{lem:A2}
Con la normalización metapléctica usual para la transformada de Fourier adélica,
la fórmula de Poisson de Weil en $\A$ induce la simetría funcional
\[
D(1-s)\;=\;D(s)\,.
\]
\end{lemma}

\begin{proof}
Sea $f\in \S(\A)$ y $\widehat f$ su transformada. La identidad de Poisson en $\A$ establece
$\sum_{x\in \Q} f(x)=\sum_{x\in \Q}\widehat f(x)$ y, tras factorizar localmente, produce el
factor arquimediano $\gamma_\infty(s)=\pi^{-s/2}\Gamma(s/2)$ que satisface $\gamma_\infty(1-s)=\gamma_\infty(s)$
(cf.~\cite{Weil1964}).
En el lado operatorial, consideremos el involutivo $J: \Phi(x)\mapsto \Phi(-x)$.
La normalización metapléctica (elección de medidas y caracteres) y la compatibilidad de Fourier
conjugan el resolvente suavizado por $J$ de forma que, sobre bandas verticales,
\[
J\,R_\delta(s;A)\,J^{-1} \;=\; R_\delta(1-s;A)\,.
\]
Por teoría de determinantes de clase de traza, $\det(I+B_{S,\delta}(1-s))=\det(I+B_{S,\delta}(s))$.
Pasando al límite $(S,\delta)$ en la topología $\mathcal S_1$ se obtiene $D(1-s)=D(s)$.
La deducción es el avatar de la ecuación funcional global vía Poisson adélico \cite{tate1967,Weil1964}.
\end{proof}

\bigskip
\noindent\textbf{A4. Regularidad espectral (clase de traza holomorfa).}

\begin{lemma}[A4: regularidad espectral uniforme]\label{lem:A4}
Fijado $\varepsilon>0$, en toda banda vertical $\Omega_\varepsilon=\{s\in \C:\,|\Re s-\tfrac12|\ge \varepsilon\}$
la familia $B_{S,\delta}(s)$ pertenece a $\mathcal S_1$ y depende holomórficamente de $s$ en norma de traza,
uniformemente en $S$ y $\delta$ pequeños. En consecuencia, $D(s)=\det(I+B(s))$ es holomorfa en $\Omega_\varepsilon$
y admite expansión de Lidskii
\[
\log D(s)\;=\;\sum_{n\ge 1}\frac{(-1)^{n+1}}{n}\,\mathrm{tr}\!\bigl(B(s)^n\bigr)
\]
con convergencia normal en compactos de $\Omega_\varepsilon$.
\end{lemma}

\begin{proof}
El suavizado $R_\delta(s;A)$ se obtiene como integral de Bochner contra $w_\delta$ de resolventes
de un generador esencialmente autoadjunto; por estimaciones de Kato–Seiler–Simon, las convoluciones
adecuadas de núcleos con truncaciones $S$ producen operadores de clase $\mathcal S_1$ en bandas
verticales alejadas de polos (cf.~\cite{simon2005}).
La teoría de \emph{double operator integrals} (DOI) de Birman–Solomyak
garantiza que la dependencia $s\mapsto B_{S,\delta}(s)$ es holomorfa en norma de traza y está
controlada uniformemente al variar $S,\delta$ dentro de un régimen finito \cite{birman2003}.
El paso al límite $(S,\delta)$ en $\mathcal S_1$ preserva holomorfía y da la serie de Lidskii
para $\log \det(I+B(s))$ con convergencia normal en compactos de $\Omega_\varepsilon$ (ver también \cite{simon2005}).
\end{proof}

\bigskip
\noindent\textbf{Descarga de axiomas y cierre.}

\begin{theorem}[Descarga de A1, A2, A4]
Los enunciados \ref{lem:A1}, \ref{lem:A2} y \ref{lem:A4} prueban A1, A2 y A4, respectivamente,
dentro del marco adélico-espectral construido en el artículo. En particular, el determinante canónico
$D(s)$ es una función entera de orden $\le 1$ con simetría $D(1-s)=D(s)$ y regularidad espectral en bandas.
\end{theorem}

\begin{corollary}[Marco incondicional]
El andamiaje de la prueba deja de ser condicional: las condiciones antes llamadas ``axiomas S-finitos''
son ahora lemas probados. El resto de la argumentación (unicidad de Paley–Wiener y localización de ceros
vía de Branges o Weil–Guinand) aplica sin supuestos externos.
\end{corollary}

\begin{remark}[Compatibilidad con secciones posteriores]
La Sección de Unicidad (Paley–Wiener) usa la entereza y simetría para concluir $D\equiv \Xi$
bajo igualdad de medida de ceros con multiplicidades; la Sección de Localización (de Branges / Weil–Guinand)
fuerza que los ceros estén en $\Re s=\tfrac12$. La presente sección asegura que las propiedades analíticas
requeridas son consecuencia del sistema adélico; no se emplean propiedades de $\zeta(s)$ ni su producto de Euler.
\end{remark}
