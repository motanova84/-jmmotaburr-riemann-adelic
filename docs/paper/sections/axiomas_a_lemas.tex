\section{De Axiomas a Lemas (A1--A4)}

\begin{lemma}[A1: flujo a escala finita -- Tate-Weil Theory]\label{lem:A1}
Para $\Phi\in\mathcal S(\Bbb A_\Bbb Q)$ factorizable, el flujo $u\mapsto \Phi(u\cdot)$
es localmente integrable con energía finita. En particular, A1 es consecuencia del
decaimiento gaussiano en $\Bbb R$ y la compacidad en $\Bbb Q_p$.
\end{lemma}

\begin{proof}
Siguiendo la teoría adélica de Tate \cite{Tate1967} y la dualidad de Pontryagin 
establecida por Weil \cite{Weil1964}, consideramos la factorización canónica
$\Phi = \bigotimes_v \Phi_v$ donde:

\begin{enumerate}
\item Para $v=\infty$: $\Phi_\infty \in \mathcal{S}(\mathbb{R})$ con decaimiento gaussiano
implica $\int_{\mathbb{R}} |\Phi_\infty(ux)|^2 |x|^s dx < \infty$ para $\Re(s) > -1/2$.

\item Para $v$ finito: La medida de Haar en $\mathbb{Q}_p$ es compacta en $\mathbb{Z}_p$, 
por tanto $\Phi_v$ tiene soporte compacto y la integral $\int_{\mathbb{Q}_p} |\Phi_v(ux)| d^*x$ 
converge uniformemente en $u$.

\item Por el teorema de convergencia de Tate para funciones L adélicas, el producto 
infinito $\prod_v \zeta_v(s, \Phi_v)$ converge para $\Re(s) > 1/2$, estableciendo 
la integrabilidad del flujo con energía finita.
\end{enumerate}

La factorización adélica garantiza que el operador de traslación $T_u: \Phi \mapsto \Phi(u\cdot)$ 
define un semigrupo fuertemente continuo en $L^2(\mathbb{A}_{\mathbb{Q}})$. \qed
\end{proof}

\begin{lemma}[A2: simetría por Poisson adélico -- Weil Rigidity]\label{lem:A2}
Con la normalización metapléctica, la identidad de Poisson en $\Bbb A_\Bbb Q$
induce $D(1-s)=D(s)$ tras completar con $\gamma_\infty(s)$ (Teorema de rigidez).
\end{lemma}

\begin{proof}
La ecuación funcional se deriva de la fórmula de Poisson adélica de Weil \cite{Weil1964}.
Para $f \in \mathcal{S}(\mathbb{A}_{\mathbb{Q}})$, la transformada de Fourier adélica satisface:
$$\sum_{x \in \mathbb{Q}} f(x) = \sum_{x \in \mathbb{Q}} \hat{f}(x)$$

Aplicando esto al núcleo generador de $D(s)$:
\begin{enumerate}
\item El factor arquimediano $\gamma_\infty(s) = \pi^{-s/2}\Gamma(s/2)$ satisface la ecuación 
funcional $\gamma_\infty(1-s) = \gamma_\infty(s)$ por propiedades de la función Gamma.

\item Para lugares finitos, la dualidad local $\mathbb{Q}_p \leftrightarrow \hat{\mathbb{Q}}_p$ 
bajo la medida autoreferencial preserva la simetría $s \leftrightarrow 1-s$.

\item La convolución adélica del operador $J: \varphi(x) \mapsto \varphi(-x)$ con el 
resolvente smoothed produce la identidad $JA_\delta J^{-1} = 1 - A_\delta$, forzando
$D(1-s) = D(s)$ a nivel del determinante.
\end{enumerate}

El teorema de rigidez arquimediana (cf. Sección \ref{sec:rigidez}) completa la prueba. \qed
\end{proof}

\begin{lemma}[A4: regularidad espectral -- Birman-Solomyak Theory]\label{lem:A4}
Sea $K_s$ un núcleo suave adélico que define operadores de traza en una banda vertical.
La continuidad en traza y el resultado de Birman--Solomyak implican regularidad
espectral uniforme en $s$, estableciendo A4.
\end{lemma}

\begin{proof}
Aplicamos la teoría de operadores de traza de Birman-Solomyak \cite{BirmanSolomyak1977}
y los resultados de Simon \cite{SimonTraceIdeals2005} sobre ideales de traza:

\begin{enumerate}
\item \textbf{Clase de traza}: Para $\Re(s) > 1/2$, el operador resolvente 
$R_\delta(s; A_\delta)$ pertenece a la clase de Schatten $\mathcal{S}_1$ con norma
$$\|R_\delta(s; A_\delta)\|_1 \leq C e^{|\Im(s)|\delta}$$
para alguna constante $C$ independiente de $s$.

\item \textbf{Continuidad holomorfa}: Por el teorema de continuidad de ideales de traza
(Simon \cite{SimonTraceIdeals2005}, Theorem 3.4), la familia $\{B_\delta(s)\}_{s \in \mathbb{C}}$ 
es holomorfa en $\mathcal{S}_1$-norma en bandas verticales $|\Re(s) - 1/2| \geq \varepsilon$.

\item \textbf{Uniformidad espectral}: La descomposición espectral del operador 
$A_\delta = Z + K_\delta$ preserva la regularidad por el resultado de Birman-Solomyak
sobre perturbaciones de rango finito de operadores autoadjuntos no acotados.

\item \textbf{Determinante regularizado}: El determinante $D(s) = \det(I + B_\delta(s))$
hereda la regularidad espectral vía la fórmula de Lidskii:
$$\log D(s) = \sum_{n=1}^{\infty} \frac{(-1)^{n+1}}{n} \text{tr}(B_\delta(s)^n)$$
con convergencia uniforme en bandas compactas.
\end{enumerate}

La regularidad espectral uniforme sigue del control de la norma de Schatten y la
teoría de perturbaciones espectrales para familias holomorfas. \qed
\end{proof}

\begin{remark}[No-circularidad]
Es crucial observar que ninguno de estos lemas utiliza propiedades de $\zeta(s)$ o
su producto de Euler. La construcción es puramente adélica-espectral, derivando
las propiedades aritméticas como consecuencias geométricas del flujo espectral.
\end{remark}
