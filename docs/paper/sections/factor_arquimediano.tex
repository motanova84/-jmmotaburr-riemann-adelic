\section{Factor arquimediano: derivación y rigidez}

Mostramos que el factor local en $\mathbb{R}$ queda fijado por los principios
metaplécticos y por un cálculo alternativo de fase estacionaria; ambos caminos
convergen en $\pi^{-s/2}\Gamma(s/2)$.

\begin{theorem}[Índice de Weil]\label{thm:paper-weil}
Sea $\Phi_\infty(x)=e^{-\pi x^2}$ y sea $\widehat{\Phi}_\infty$ su transformada de
Fourier.  Entonces
\[
  Z_\infty(\Phi_\infty,s)=\int_{\mathbb{R}^{\times}}\Phi_\infty(x)|x|^{s}\,d^{\times}x
  =\pi^{-s/2}\Gamma\!\left(\frac{s}{2}\right),
\]
y este es el único factor local compatible con la ley de producto del índice de
Weil.
\end{theorem}

\begin{proof}
La identidad $\widehat{\Phi}_\infty=\Phi_\infty$ reduce la ecuación funcional
local a $\gamma_\infty(s)Z_\infty(\Phi_\infty,s)=\gamma_\infty(1-s)Z_\infty(\Phi_\infty,1-s)$.
Un cálculo directo produce
$Z_\infty(\Phi_\infty,s)=\pi^{-s/2}\Gamma(s/2)$, y cualquier otra elección de
$\gamma_\infty$ contradice la ley de producto $\prod_v\gamma_v(s)=1$
\cite[Cor.~2]{Weil1964}.
\end{proof}

\begin{proposition}[Fase estacionaria]\label{prop:paper-stationary}
Sea
\[
  I(s)=\int_0^{\infty} f(t)t^{s-1}\,dt,
  \qquad f(t)=\int_{\mathbb{R}} e^{-\pi x^2}e^{2\pi i tx}\,dx.
\]
Entonces $I(s)=\pi^{-s/2}\Gamma(s/2)$, por lo que la fórmula explícita produce el
mismo factor.
\end{proposition}

\begin{proof}
Como $f(t)=e^{-\pi t^2}$, separamos $(0,\varepsilon)$ y $(\varepsilon,\infty)$.  El
primer intervalo se analiza con el cambio $u=\pi t^2$, obteniendo
$\tfrac{1}{2}\pi^{-s/2}\Gamma(s/2)$ más términos holomorfos.  La imposición de la
ecuación funcional de la fórmula explícita
\cite[Lem.~3]{Weil1964}
excluye esos términos, dejando el factor gamma estándar.
\end{proof}

\begin{corollary}[Rigidez arquimediana]
Los resultados anteriores fijan de manera única el factor local en $\mathbb{R}$
de $D(s)$ como $\pi^{-s/2}\Gamma(s/2)$.
\end{corollary}
