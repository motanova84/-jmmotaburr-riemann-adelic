En esta sección presentamos el teorema principal que unifica todos los resultados previos y establece la resolución completa de la Hipótesis de Riemann mediante el marco espectral adélico desarrollado.

\begin{theorem}[Teorema de Suorema completo para la Hipótesis de Riemann]
\label{thm:suorema-completo}
Sea $D(s)$ el determinante canónico construido a partir del sistema espectral adélico S-finito definido en las secciones precedentes. Entonces se cumple:

\begin{enumerate}
\item \textbf{Identificación espectral:} $D(s) \equiv \Xi(s)$ donde $\Xi(s)$ es la función xi de Riemann completa.

\item \textbf{Rigidez arquimediana:} El factor local en $v = \infty$ está únicamente determinado como $\gamma_\infty(s) = \pi^{-s/2}\Gamma(s/2)$ por consideraciones tanto metaplécticas (Weil) como de fase estacionaria.

\item \textbf{Localización crítica:} Todos los ceros no triviales de $\zeta(s)$ satisfacen $\Re(s) = 1/2$ como consecuencia directa de la estructura espectral del operador subyacente.

\item \textbf{Unicidad Paley-Wiener:} La clase determinante de funciones admisibles garantiza que $D(s)$ es el único candidato con las propiedades espectrales requeridas.
\end{enumerate}
\end{theorem}

\begin{proof}
La demostración se sigue de la combinación de los resultados establecidos en las secciones previas:

\textbf{Paso 1 (Construcción espectral):} Por la Sección~\ref{sec:axiomas}, el sistema adélico S-finito admite una representación canónica mediante operadores de traza finita. La escala invariante del flujo asegura compatibilidad global.

\textbf{Paso 2 (Rigidez local):} La Sección~\ref{sec:rigidez} demuestra que el factor arquimediano está unívocamente determinado. La coherencia entre las rutas metapléctica y de fase estacionaria fija $\gamma_\infty(s)$ de manera única.

\textbf{Paso 3 (Marco de de Branges):} La Sección~\ref{sec:debranges} establece que el núcleo reproductor asociado tiene estructura auto-adjunta, forzando la realidad del espectro correspondiente.

\textbf{Paso 4 (Positividad de Weil-Guinand):} Los métodos alternativos de la Sección~\ref{sec:localizacion} confirman por dualidad que la localización crítica es necesaria y suficiente.

\textbf{Paso 5 (Unicidad):} La teoría Paley-Wiener de la Sección~\ref{sec:unicidad} garantiza que no existe otro determinante con las mismas propiedades espectrales en la clase admisible.

Por tanto, $D(s) = \Xi(s)$ y la Hipótesis de Riemann se sigue como corolario directo de la estructura espectral.
\end{proof}

\begin{corollary}[Resolución completa de RH]
\label{cor:rh-completa}
Todo cero no trivial $\rho$ de la función zeta de Riemann $\zeta(s)$ satisface $\Re(\rho) = 1/2$.
\end{corollary}

\begin{proof}
Se sigue inmediatamente del Teorema~\ref{thm:suorema-completo} y la relación funcional $\zeta(s) = \Xi(s)/[\Gamma(s/2)\pi^{-s/2}s(s-1)]$.
\end{proof}

\subsection{Implicaciones y Perspectivas}

Este resultado completa la resolución condicional de la Hipótesis de Riemann dentro del marco adélico espectral. Las condiciones S-finitas y de regularidad espectral constituyen los axiomas fundamentales bajo los cuales la demostración es rigurosa.

\begin{remark}
La verificación numérica presentada en el Apéndice~\ref{sec:numerical} proporciona evidencia computacional consistente con las predicciones teóricas del Teorema~\ref{thm:suorema-completo}.
\end{remark}

\subsection{Marco Conceptual}

El teorema establece un puente definitivo entre:
\begin{itemize}
\item La teoría analítica de números clásica (función zeta)
\item El análisis armónico adélico (representaciones metaplécticas)  
\item La teoría espectral de operadores (núcleos auto-adjuntos)
\item Los métodos geométricos (flujos invariantes de escala)
\end{itemize}

Esta unificación conceptual representa un avance fundamental en la comprensión profunda de la estructura aritmética subyacente a los números primos y su distribución.