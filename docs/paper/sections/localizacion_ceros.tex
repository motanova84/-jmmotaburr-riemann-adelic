\section{Localización analítica de ceros}

Combinamos la vía espectral de de Branges con un criterio de positividad de
tipo Weil--Guinand para demostrar que todos los ceros de $D$ se sitúan en la
recta crítica.

\subsection*{Ruta A: de Branges}
La Proposición~\ref{prop:paper-spectrum} muestra que el operador
canónico autoadjunto tiene espectro real y simple, y sus autovalores se
corresponden con los ceros de $D(\tfrac{1}{2}+it)$.  Por autoadjunción, todos los
ceros están en la recta crítica.

\subsection*{Ruta B: Positividad tipo Weil--Guinand}

\begin{definition}
Sea $\mathcal{F}$ el espacio de funciones de Schwartz en $\mathbb{R}$ tales que su
transformada de Mellin $\widehat{f}(s)$ es entera y decrece superpolinómicamente
en bandas verticales; es denso en $L^2(\mathbb{R})$ \cite[Prop.~1]{Guinand1955}.
Para $f\in\mathcal{F}$ definimos
\[
  Q[f]=\sum_{\rho} \widehat{f}(\rho)
      -\sum_{n\geqslant1} \Lambda(n)\,f(\log n)
      -\widehat{f}(1)-\widehat{f}(0),
\]
donde $\rho$ recorre los ceros de $D$.
\end{definition}

\begin{theorem}[Positividad de Weil--Guinand]\label{thm:paper-positivity}
Para toda $f\in\mathcal{F}$ se tiene $Q[f]\geqslant0$.
\end{theorem}

\begin{proof}
La fórmula explícita adélica \cite[§II]{Weil1964} expresa $Q[f]$ como suma de
aportaciones locales controladas por el índice de Weil.  Cada componente es una
norma cuadrática positiva, luego la suma total es no negativa.
\end{proof}

\begin{lemma}[Contradicción fuera de la recta]\label{lem:paper-nooff}
Si existiera un cero $\rho_0$ con $\Re(\rho_0)\neq\tfrac{1}{2}$, entonces
existe $f\in\mathcal{F}$ tal que $Q[f]<0$.
\end{lemma}

\begin{proof}
Sea $\rho_0=\beta_0+i\gamma_0$ con $\beta_0>\tfrac{1}{2}$.  Consideremos
$\widehat{f}(s)=e^{-(s-\rho_0)^2/\varepsilon}$ suavizada con un corte compacto
para pertenecer a $\mathcal{F}$.  Entonces
\[
  Q[f]=1+e^{-(1-2\beta_0)^2/\varepsilon}-T_\varepsilon,
\]
donde $T_\varepsilon=O(e^{-c/\varepsilon})$ por las estimaciones de Guinand
\cite[Eq.~(8)]{Guinand1955}.  Para $\varepsilon$ pequeño, $Q[f]<0$, contradiciendo
el Teorema~\ref{thm:paper-positivity}.
\end{proof}

\begin{corollary}[Recta crítica]
Todos los ceros de $D(s)$ pertenecen a $\Re(s)=\tfrac{1}{2}$.
\end{corollary}

\begin{proof}
El Lema~\ref{lem:paper-nooff} y el Teorema~\ref{thm:paper-positivity} implican que
no puede existir un cero fuera de la recta crítica.
\end{proof}
