\section{Trace-Class Convergence and Canonical Determinants}
\label{sec:appendix-traces}

\subsection*{Set-up and smoothing}

Let $Z$ denote the unperturbed scale generator on a suitable Hilbert space
$\mathcal H$ attached to the S-finite adelic flow (Sections~\ref{sec:axiomas_a_lemas}
and~\ref{sec:rigidez}). For $\delta>0$ fix an even window $w_\delta\in\mathcal S(\R)$
with $\int_\R w_\delta(u)\,du=1$ and define the smoothed resolvent
\[
R_\delta(s;A)\;:=\;\int_\R w_\delta(u)\,(A+u - s)^{-1}\,du,
\qquad s\in\C,
\]
whenever the Bochner integral converges in the operator norm.
Let $A_\delta:=Z+K_\delta$ be the smoothed (bounded) perturbation,
with $K_\delta$ built from the adelic kernel and the double operator integral (DOI)
machinery (cf. Birman–Solomyak). We compare against the reference $A_0:=Z$ and set
\[
B_{S,\delta}(s)\;:=\;R_\delta(s;A_{S,\delta})-R_\delta(s;A_0),
\qquad
B_\delta(s)\;:=\;\lim_{S\uparrow V} B_{S,\delta}(s),
\]
where $S\subset V$ is finite and $V$ runs over all places (S-finiteness).

\begin{lemma}[Boundedness and DOI representation]
\label{lem:bounded-doi}
For every fixed $\delta>0$ and $s$ with $|\Re s-\tfrac12|\ge \varepsilon>0$,
the integral defining $R_\delta(s;A)$ converges in operator norm and yields a bounded
operator on $\mathcal H$. Moreover,
\[
R_\delta(s;A)-R_\delta(s;A_0)
\;=\;\int_{\R} \!\!\int_{\R}
\frac{w_\delta(u)-w_\delta(v)}{(u-s)-(v-s)}\,dE_A(u)\,K_\delta\,dE_{A_0}(v),
\]
where $E_A,E_{A_0}$ are the spectral measures. In particular $B_{S,\delta}(s)$ is a DOI
with kernel in $L^2(\R^2)$.
\end{lemma}

\begin{proof}[Sketch]
The Bochner integral is standard since $w_\delta\in\mathcal S$ and
$\|(A+u-s)^{-1}\|\lesssim \langle u\rangle^{-1}$ for $s$ away from the spectrum of $A$.
The DOI formula follows from Birman–Solomyak's calculus for functions of (possibly
unbounded) self-adjoint operators applied to $f_u(\lambda)=(\lambda+u-s)^{-1}$; see
\cite{birman2003} and \cite[Ch.~9]{simon2005}.
\end{proof}

\subsection*{Trace class and holomorphy}

\begin{proposition}[Schatten bounds and holomorphy]
\label{prop:trace-class}
For every strip $\Omega_\varepsilon=\{s\in\C:\,|\Re s-\tfrac12|\ge \varepsilon\}$ and
fixed $\delta>0$ we have $B_{S,\delta}(s)\in\mathcal S_1$ (\emph{trace class}) uniformly
in $S$, and $s\mapsto B_{S,\delta}(s)$ is holomorphic with values in $\mathcal S_1$.
Consequently $B_\delta(s)\in\mathcal S_1$ and $s\mapsto B_\delta(s)$ is $\mathcal S_1$-holomorphic
on $\Omega_\varepsilon$.
\end{proposition}

\begin{proof}[Sketch]
The DOI kernel belongs to $L^2(\R^2)$ and $K_\delta$ is Hilbert–Schmidt at the local
(adelic) level by S-finiteness; thus $B_{S,\delta}(s)\in \mathcal S_2$.
Refining with Kato–Seiler–Simon type estimates for integral kernels and the smoothing of
$w_\delta$ yields $\mathcal S_1$ bounds, uniformly in $S$. Holomorphy in
$\mathcal S_1$ follows from the analytic vector-valued mapping theorem for trace ideals
(\cite[Thm.~3.7]{simon2005}).
\end{proof}

\begin{corollary}[Lidskii series and normal convergence]
\label{cor:lidskii}
For $s\in\Omega_\varepsilon$ the Lidskii series
\[
\log\det\bigl(I+B_{S,\delta}(s)\bigr)
=\sum_{n\ge 1}\frac{(-1)^{n+1}}{n}\,\mathrm{tr}\bigl(B_{S,\delta}(s)^n\bigr)
\]
converges absolutely and locally uniformly in $s$, uniformly in $S$. The same holds
for $B_\delta(s)$.
\end{corollary}

\begin{proof}
Absolute convergence follows from $\|B_{S,\delta}(s)\|_{\mathcal S_1}$ bounds and
standard estimates for power series in trace ideals; see \cite[Ch.~3,9]{simon2005}.
Uniformity in $S$ is granted by the S-finite truncation and the DOI control.
\end{proof}

\subsection*{Canonical determinant and independence of $\delta,S$}

\begin{theorem}[Canonical determinant and order $\le 1$]
\label{thm:canonical-det}
The function
\[
D_{S,\delta}(s)\;:=\;\det\bigl(I+B_{S,\delta}(s)\bigr),
\qquad
D_\delta(s)\;:=\;\det\bigl(I+B_{\delta}(s)\bigr),
\]
is entire on $\C$, satisfies $D_\delta(1-s)=D_\delta(s)$, and
has order $\le 1$. Moreover $D_\delta$ is independent of $S$ and $\delta$ up to a
nonzero entire factor of \emph{zero exponential type}; after the normalization
$\lim_{\Re s\to +\infty}\log D_\delta(s)=0$ we obtain a canonical $D(s)$.
\end{theorem}

\begin{proof}[Sketch]
Entireness follows from Corollary~\ref{cor:lidskii} and Montel–Vitali arguments for
$\mathcal S_1$-holomorphic families. The functional symmetry comes from the conjugation
$J\varphi(x)=\varphi(-x)$ at the level of the flow, giving $JA_\delta J^{-1}=1-A_\delta$,
hence $B_\delta(1-s)=JB_\delta(s)J^{-1}$ and $\det(I+B_\delta(1-s))=\det(I+B_\delta(s))$.
Order $\le 1$ follows from Hadamard factorization combined with growth bounds of
$\|B_\delta(s)\|_{\mathcal S_1}$ on vertical strips (Paley–Wiener type control,
cf.~\cite{koosis1988,levin1996}). Independence from $S$ and $\delta$ is a standard
Tauberian/approximation argument: variations produce a trace-class coboundary whose
determinant factor has zero type; the normalization fixes this factor to~1.
\end{proof}

\subsection*{Archimedean factor and functional equation}

\begin{proposition}[Archimedean rigidity]
\label{prop:arch-factor}
Under the metaplectic normalization and adelic Poisson summation \cite{Weil1964,tate1967},
the Archimedean contribution equals $\gamma_\infty(s)=\pi^{-s/2}\Gamma(s/2)$ and enforces
$D(1-s)=D(s)$.
\end{proposition}

\begin{proof}[Sketch]
It is the standard computation of the Weil index at the real place with even test
functions; see \cite{Weil1964}, and the usual stationary phase arguments.
\end{proof}

\medskip
\noindent\textbf{References used in this appendix.}
\begin{itemize}
  \item Birman–Solomyak DOI calculus \cite{birman2003};
  \item Simon's trace ideals and analytic families \cite{simon2005};
  \item Paley–Wiener / growth control \cite{koosis1988,levin1996};
  \item Adelic Poisson, Tate's thesis, Weil's index \cite{tate1967,Weil1964}.
\end{itemize}