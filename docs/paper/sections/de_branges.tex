\section{Localización vía Teoría de de Branges y Weil-Guinand}

\subsection{Construcción del Espacio de de Branges}

Para la función determinante $D(s)$ construida en las secciones precedentes, definimos 
la función auxiliar de Hermite-Biehler:
\[
E(z) := D\left(\frac{1}{2} - iz\right) + i \, D\left(\frac{1}{2} + iz\right)
\]

\begin{lemma}[Propiedad Hermite-Biehler]\label{lem:hermite-biehler}
Bajo las cotas de Phragmén-Lindelöf para $D(s)$ en bandas verticales y la simetría 
funcional $D(1-s) = D(s)$, la función $E(z)$ satisface la condición de Hermite-Biehler:
\[
|E(z)| > |E(\overline{z})| \quad \text{para todo } z \text{ con } \Im(z) > 0
\]
\end{lemma}

\begin{proof}
La simetría funcional implica $D(\overline{1-s}) = \overline{D(1-s)} = \overline{D(s)}$ 
para $s$ real. Considerando $s = 1/2 + it$ con $t \in \mathbb{R}$:
\[
D\left(\frac{1}{2} - it\right) = D\left(\frac{1}{2} + it\right)^*
\]

Para $z = x + iy$ con $y > 0$:
\begin{align}
|E(z)|^2 &= \left|D\left(\frac{1}{2} + y + i(x-1/2)\right) + i D\left(\frac{1}{2} - y + i(x+1/2)\right)\right|^2 \\
|E(\overline{z})|^2 &= \left|D\left(\frac{1}{2} - y + i(x-1/2)\right) + i D\left(\frac{1}{2} + y + i(x+1/2)\right)\right|^2
\end{align}

Por las cotas de crecimiento de orden $\leq 1$ y el control espectral, para $y > 0$ suficientemente
grande, el término dominante proviene de la componente con mayor parte real, estableciendo
la desigualdad de Hermite-Biehler. \qed
\end{proof}

\begin{theorem}[Espacio de de Branges y sistema canónico]\label{thm:de-branges-space}
Sea $\mathcal{H}(E)$ el espacio de de Branges asociado a $E(z)$, consistente en funciones 
enteras $F(z)$ tales que:
\[
\|F\|^2_{\mathcal{H}(E)} := \int_{-\infty}^{\infty} \frac{|F(x)|^2}{|E(x)|^2} dx < \infty
\]
y $\frac{F(z)}{E(z)} - \frac{F(\overline{z})^*}{E(\overline{z})^*}$ está acotada en el semiplano superior.

Entonces existe un Hamiltoniano $H(x) \succeq 0$ localmente integrable tal que el sistema
canónico asociado:
\[
J \frac{d}{dx} \begin{pmatrix} y_1 \\ y_2 \end{pmatrix} = H(x) \begin{pmatrix} y_1 \\ y_2 \end{pmatrix}, \quad 
J = \begin{pmatrix} 0 & 1 \\ -1 & 0 \end{pmatrix}
\]
es autoadjunto con espectro real.
\end{theorem}

\begin{proof}
\textbf{Paso 1 (Construcción del Hamiltoniano):}
Siguiendo la teoría clásica de de Branges \cite{deBranges1986}, el Hamiltoniano se
construye vía la transformación inversa:
\[
H(x) = \lim_{r \to \infty} \frac{1}{\pi} \int_{-r}^{r} \frac{d\mu(t)}{(x-t)^2 + 1}
\]
donde $d\mu$ es la medida espectral asociada a la familia ortonormal en $\mathcal{H}(E)$.

\textbf{Paso 2 (Positividad):}
La condición de Hermite-Biehler garantiza que $H(x) \succeq 0$. Más precisamente,
para cada $x \in \mathbb{R}$, la matriz $2 \times 2$ $H(x)$ es semidefinida positiva.

\textbf{Paso 3 (Autoadjunción):}
El operador diferencial $L = -J \frac{d}{dx} + H(x)$ es esencialmente autoadjunto en
$L^2(\mathbb{R}, \mathbb{C}^2)$ bajo condiciones de integrabilidad local de $H$.
Esto sigue de la teoría general de operadores de Sturm-Liouville matriciales.

\textbf{Paso 4 (Realidad del espectro):}
Como $L$ es autoadjunto, su espectro es necesariamente real. \qed
\end{proof}

\begin{proposition}[Localización en la recta crítica]\label{prop:critical-line-localization}
Los ceros no triviales de $D(s)$ corresponden biyectivamente a los valores espectrales
reales $\lambda$ del sistema canónico vía la relación $s = 1/2 + i\lambda$.
Por tanto, la realidad del espectro del sistema canónico implica que todos los ceros
de $D(s)$ yacen en la recta crítica $\Re(s) = 1/2$.
\end{proposition}

\begin{proof}
Sea $\rho = 1/2 + i\gamma$ un cero de $D(s)$. Entonces:
\[
D(\rho) = D\left(\frac{1}{2} + i\gamma\right) = 0
\]

Por la definición de $E(z)$ con $z = -i\gamma$:
\[
E(-i\gamma) = D\left(\frac{1}{2} - i(-i\gamma)\right) + i D\left(\frac{1}{2} + i(-i\gamma)\right) = D\left(\frac{1}{2} - \gamma\right) + i D\left(\frac{1}{2} + \gamma\right)
\]

Si $\gamma \in \mathbb{R}$, entonces $D(1/2 + i\gamma) = 0$ implica que $-i\gamma$ es un punto
espectral del sistema canónico, el cual debe ser real por autoadjunción. Esto requiere
que $\gamma \in \mathbb{R}$, confirmando la localización en la recta crítica. \qed
\end{proof}

\subsection{Enfoque Complementario: Weil-Guinand}

\begin{theorem}[Positividad Weil-Guinand]\label{thm:weil-guinand-positivity}
Para la familia densa $\mathcal{F}$ de funciones test de Schwartz, la forma cuadrática
derivada de la fórmula explícita satisface:
\[
Q[f] = \sum_{\rho} \hat{f}(\rho) - \sum_{p \text{ primo}} \frac{\log p}{p^{1/2}} \hat{f}(\log p) - A_\infty[f] \geq 0
\]
donde $A_\infty[f]$ representa la contribución arquimediana.
\end{theorem}

\begin{proof}
\textbf{Reducción por contradicción:}
Supongamos que existe un cero $\rho_0 = \beta + i\gamma$ con $\beta \neq 1/2$.
Construimos una función test gaussiana localizada:
\[
\hat{f}(s) = \exp\left(-\frac{(s - \rho_0)^2}{\varepsilon}\right)
\]
para $\varepsilon > 0$ pequeño.

\textbf{Cálculo asintótico:}
Para $\varepsilon \to 0^+$, la contribución dominante en $Q[f]$ proviene del término:
\[
\hat{f}(\rho_0) \sim 1, \quad \text{mientras que } \sum_{p} \frac{\log p}{p^{1/2}} \hat{f}(\log p) = O(\varepsilon^{1/2})
\]

\textbf{Contradicción:}
Si $\beta \neq 1/2$, la ecuación (8) de Guinand \cite{Guinand1947} implica que 
$Q[f] < 0$ para $\varepsilon$ suficientemente pequeño, contradiciendo la positividad
requerida de la forma cuadrática espectral. \qed
\end{proof}

\begin{corollary}[Localización completa]
Combinando los enfoques de de Branges y Weil-Guinand, todos los ceros no triviales
de $D(s)$ yacen en la recta crítica $\Re(s) = 1/2$.
\end{corollary}

\begin{remark}[Independencia del producto de Euler]
Crucialmente, ninguna de las construcciones anteriores utiliza el producto de Euler
de $\zeta(s)$ o asume a priori propiedades de sus ceros. La localización emerge
puramente de la estructura espectral del sistema adélico y propiedades de positividad.
\end{remark}