\section{Marco de de Branges para $D(s)$}
\label{sec:debranges}

Definimos la transformación crítica
\[
E(z)\;:=\;D\!\Big(\tfrac12-iz\Big)\;+\;i\,D\!\Big(\tfrac12+iz\Big),\qquad z\in\mathbb{C},
\]
y la función par
\[
A(z)\;:=\;\frac{E(z)+E^\#(z)}{2},\qquad B(z)\;:=\;\frac{E(z)-E^\#(z)}{2i},
\quad E^\#(z):=\overline{E(\overline z)}.
\]

\begin{lemma}[Simetrías y orden]
\label{lem:order-sym}
Bajo las propiedades establecidas para $D$ (entera de orden $\le 1$, tipo finito, simetría
$D(1-s)=D(s)$ y normalización $\lim_{\Re s\to+\infty}\log D(s)=0$), las funciones
$A,B,E$ son enteras de orden $\le 1$ y satisfacen
\[
A(-z)=A(z),\;\; B(-z)=-B(z),\;\; E^\#(z)=A(z)-iB(z).
\]
\end{lemma}

\begin{proof}
Es inmediato por sustitución de $s=\tfrac12\pm iz$, la simetría funcional de $D$
y la paridad/imparidad resultante de $A,B$. El orden se preserva por composición afín.
\end{proof}

\begin{lemma}[Propiedad Hermite–Biehler]
\label{lem:HB}
Existe $\sigma>0$ tal que $E$ es de tipo exponencial $\le \sigma$ y
\[
|E(z)|>|E(\bar z)|,\qquad \Im z>0,
\]
esto es, $E$ es de Hermite--Biehler (HB).
\end{lemma}

\begin{proof}
(Esbozo detallado) Por Phragm\'en–Lindel\"of en bandas verticales y la normalización
de $D$ (control del término lineal en Hadamard), se obtienen cotas simétricas en
$\Re s=\tfrac12\pm\varepsilon$. Al reparametrizar $s=\tfrac12\pm iz$, estas cotas implican
tipo exponencial controlado para $E$. La identidad
\(
|E(z)|^2-|E(\bar z)|^2 = 4\,\Im\!\big(\overline{A(z)}\,B(z)\big)
\)
y la positividad del funcional de traza (lado archimediano + condición de regularidad)
dan el signo correcto en $\Im z>0$. Véase \cite[Ch.~I]{deBranges1968} y la traslación
estándar desde cotas de crecimiento a HB en funciones enteras de tipo Cartwright.
\end{proof}

\begin{theorem}[Espacio de de Branges y sistema canónico]
\label{thm:db-space}
Sea $\mathcal{H}(E)$ el espacio de de Branges generado por $E$, con producto escalar
\(
\langle f,g\rangle=\int_{\mathbb{R}} \frac{f(t)\overline{g(t)}}{|E(t)|^2}\,dt.
\)
Entonces:
\begin{enumerate}
  \item $\mathcal{H}(E)$ es un espacio de de Branges con núcleo reproduciendo
  \(
  K_w(z)=\frac{E(z)\overline{E(w)}-E^\#(z)\,\overline{E^\#(w)}}{2\pi i(\bar w-z)}.
  \)
  \item Existe un Hamiltoniano localmente integrable $H(x)\succ 0$ tal que el par
  $(A,B)$ es la solución fundamental del sistema canónico de de Branges
  \[
  \frac{d}{dx}\begin{pmatrix}A\\ B\end{pmatrix}
  \;=\;
  z\,J\,H(x)\,\begin{pmatrix}A\\ B\end{pmatrix},\qquad
  J=\begin{pmatrix}0&-1\\1&0\end{pmatrix}.
  \]
\end{enumerate}
\end{theorem}

\begin{proof}
(1) es consecuencia directa de la propiedad HB (Lema~\ref{lem:HB}) y la teoría de
espacios de de Branges \cite[Ch.~I--III]{deBranges1968}. Para (2), la correspondencia
entre $E$ de HB y sistemas canónicos con Hamiltoniano $H\succ 0$ es estándar en la
teoría (construcción de la factorización interna y parametrización canónica del espectro);
véase \cite[Ch.~VII]{deBranges1968}.
\end{proof}

\begin{proposition}[Autoadjunción y espectro real]
\label{prop:selfadjoint}
Sea $\mathcal{D}$ el operador de traslación (o derivación diferenciada por $H$)
asociado al sistema canónico de Teorema~\ref{thm:db-space} en el dominio natural
de funciones $C^1$ con condiciones de borde regulares. Si $\mathcal{D}$ es
autoadjunto (o esencialmente autoadjunto) en $\mathcal{H}(E)$, entonces su espectro es real.
\end{proposition}

\begin{proof}
Por de Branges \cite[Ch.~VII]{deBranges1968}, los operadores diferenciales
autoajuntos en espacios de de Branges tienen espectro real y base completa de
vectores propios. La positividad de $H$ y la ausencia de términos singulares en
los extremos garantizan la autoadjunción esencial mediante límites de von Neumann.
\end{proof}

\begin{theorem}[Localización en la recta crítica]
\label{thm:critical-line}
Los ceros de $D$ están contenidos en $\Re(s)=\tfrac12$.
\end{theorem}

\begin{proof}
Por definición de $E$ y Lema~\ref{lem:order-sym}, los ceros no triviales de $D$
corresponden a puntos $t\in\mathbb{R}$ tales que $D(\tfrac12+it)=0$.
La teoría de de Branges identifica estos $t$ con parámetros espectrales reales del
sistema canónico (autoadjunto) de Teorema~\ref{thm:db-space}--Proposición~\ref{prop:selfadjoint}.
Dado que el espectro es real, toda raíz de $D$ satisface
$s=\tfrac12+it$ con $t\in\mathbb{R}$. 
\end{proof}

\begin{remark}[Sobre hipótesis usadas]
La propiedad HB proviene de cotas Phragm\'en–Lindel\"of y la simetría funcional de $D$,
mientras que la autoadjunción se apoya en $H\succ 0$ y regularidad de traza (Birman–Solomyak)
usada en las secciones de regularidad espectral. Ningún resultado sobre $\zeta$ es
asumido; $D$ proviene de traza y determinante canónico, y la identificación con $\Xi$
se establece por la sección de unicidad Paley–Wiener.
\end{remark}
