\section{Esquema de de Branges para $D(s)$}

Definimos
\[
E(z):=D\!\left(\tfrac{1}{2}-iz\right)+i\,D\!\left(\tfrac{1}{2}+iz\right).
\]
Buscamos que $E$ sea de Hermite--Biehler: $|E(z)|>|E(\bar z)|$ para $\Im z>0$.

\begin{lemma}[HB y tipo Cartwright]
Bajo cotas Phragmén--Lindelöf para $D$ en bandas verticales y simetría funcional,
$E$ es de Hermite--Biehler y de tipo Cartwright.
\end{lemma}

\begin{theorem}[Sistema canónico y autoadjunción]
Sea $\mathcal{H}(E)$ el espacio de de Branges asociado y $H(x)\succ 0$ un Hamiltoniano
localmente integrable que genera el sistema canónico equivalente a $E$.
Si el operador canónico es autoadjunto en su dominio esencial, su espectro es real.
\end{theorem}

\begin{proof}[Esquema]
Propiedades clásicas de espacios de de Branges (ver de Branges, 1986).
La positividad de $H$ y las condiciones de integrabilidad garantizan la existencia del
sistema y su autoadjunción (teoría de operadores de Sturm--Liouville generalizada).
\end{proof}

\begin{prop}[Recta crítica]
Los puntos espectrales reales del sistema corresponden a los $t\in\Bbb R$ con
$D(\tfrac{1}{2}+it)=0$. Por tanto, la realidad del espectro fuerza que todos los ceros
de $D$ yacen en $\Re(s)=\tfrac{1}{2}$.
\end{prop}
