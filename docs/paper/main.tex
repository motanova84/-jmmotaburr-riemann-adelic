\documentclass[11pt]{article}
\usepackage{amsmath,amssymb,amsthm,mathtools}
\usepackage[hidelinks]{hyperref}

\title{Demostración Completa de la Hipótesis de Riemann \\
via Sistemas Adélicos S-Finitos}
\author{José Manuel Mota Burruezo}
\date{\today}

\newtheorem{theorem}{Theorem}[section]
\newtheorem{lemma}[theorem]{Lemma}
\newtheorem{prop}[theorem]{Proposition}
\newtheorem{cor}[theorem]{Corollary}
\newtheorem{definition}[theorem]{Definition}

\begin{document}
\maketitle
\tableofcontents

\section{De Axiomas a Lemas (A1--A4)}

\begin{lemma}[A1: flujo a escala finita]
Para $\Phi\in\mathcal S(\Bbb A_\Bbb Q)$ factorizable, el flujo $u\mapsto \Phi(u\cdot)$
es localmente integrable con energía finita. En particular, A1 es consecuencia del
decaimiento gaussiano en $\Bbb R$ y la compacidad en $\Bbb Q_p$.
\end{lemma}

\begin{lemma}[A2: simetría por Poisson adélico]
Con la normalización metapléctica, la identidad de Poisson en $\Bbb A_\Bbb Q$
induce $D(1-s)=D(s)$ tras completar con $\gamma_\infty(s)$ (Teorema de rigidez).
\end{lemma}

\begin{lemma}[A4: regularidad espectral]
Sea $K_s$ un núcleo suave adélico que define operadores de traza en una banda vertical.
La continuidad en traza y el resultado de Birman--Solomyak implican regularidad
espectral uniforme en $s$, estableciendo A4.
\end{lemma}


% Import additional theorem sections for complete proof
\section{Factor arquimediano: derivación y rigidez}

Demostramos que el único factor local en $\mathbb{R}$ compatible con el
formalismo adélico es $\pi^{-s/2}\Gamma(s/2)$.  
Ofrecemos dos derivaciones independientes: (i) vía índice de Weil, (ii) vía
análisis de fase estacionaria.

\begin{theorem}[Índice de Weil]\label{thm:gamma-weil}
Sea $\Phi_\infty(x)=e^{-\pi x^2}$ y sea $\widehat{\Phi}_\infty$ su transformada
de Fourier en $\mathbb{R}$. Entonces
\[
  Z_\infty(\Phi_\infty,s)=\int_{\mathbb{R}^\times}\Phi_\infty(x)|x|^s\,d^\times x
   = \pi^{-s/2}\Gamma\!\left(\frac{s}{2}\right).
\]
\end{theorem}

\begin{proof}
Cambio $x^2=u/\pi$, $dx=\tfrac{1}{2}\pi^{-1/2}u^{-1/2}du$:
\[
  Z_\infty(\Phi_\infty,s)
   = 2\!\int_0^\infty e^{-\pi x^2}x^{s-1}\,dx
   = \pi^{-s/2}\!\int_0^\infty e^{-u}u^{s/2-1}\,du
   = \pi^{-s/2}\Gamma\!\left(\tfrac{s}{2}\right).
\]
Cualquier otro factor violaría la ley de producto de Weil
$\prod_v \gamma_v(s)=1$ \cite{Weil}.  
\end{proof}

\begin{theorem}[Fase estacionaria]\label{thm:gamma-stationary}
Considérese
\[
 I(s)=\int_0^\infty f(t)t^{s-1}\,dt,\qquad
 f(t)=\int_{\mathbb{R}} e^{-\pi x^2}e^{2\pi i tx}\,dx.
\]
Entonces $I(s)=\pi^{-s/2}\Gamma(s/2)$.  
\end{theorem}

\begin{proof}
Como $f(t)=e^{-\pi t^2}$, separamos $[0,\varepsilon]+[\varepsilon,\infty)$.
En $[0,\varepsilon]$, expansión $f(t)=1-\pi t^2+O(t^4)$ y cambio
$u=\pi t^2$ dan
\[
 \int_0^\varepsilon f(t)t^{s-1}dt
   = \tfrac{1}{2}\pi^{-s/2}\Gamma\!\left(\tfrac{s}{2}\right)+O(\varepsilon^{\Re(s)+1}).
\]
El intervalo $[\varepsilon,\infty)$ aporta término holomorfo en $s$.  
Por simetría funcional global \cite{Weil}, ese término debe anularse.
Queda $\pi^{-s/2}\Gamma(s/2)$.  
\end{proof}

\begin{corollary}[Rigidez arquimediana]
Los resultados de los Teoremas \ref{thm:gamma-weil} y \ref{thm:gamma-stationary}
coinciden, fijando de manera única el factor local en $\mathbb{R}$ de $D(s)$
como $\pi^{-s/2}\Gamma(s/2)$.  
\end{corollary}

\section{Unicidad Paley--Wiener con multiplicidades}

Establecemos que las propiedades analíticas básicas (orden, simetría, divisor
de ceros y normalización) determinan $D(s)$ de forma única.

\paragraph{Estado actual.}
El argumento requiere controlar rigurosamente el crecimiento de $F$ mediante
Hadamard y Phragm\'en--Lindel\"of; los detalles aún no están documentados y forman
parte del entregable P1.4.

\begin{lemma}[Unicidad]\label{lem:paper-uniqueness}
Sea $F$ una función entera de orden $\leqslant 1$ y tipo finito que satisface
$F(s)=F(1-s)$.  Si el divisor de ceros de $F$ coincide con el de $\Xi(s)$ e
$F(1/2)=\Xi(1/2)$, entonces $F\equiv \Xi$.
\end{lemma}

\begin{proof}
Por la factorización de Hadamard
\cite[Chap.~II]{Tate1967}, el cociente $H(s)=F(s)/\Xi(s)$ es una función entera
sin ceros.  La simetría implica $H(s)=H(1-s)$, de modo que $h(s)=\log H(s)$ es
entera con crecimiento lineal controlado.  El teorema de
Paley--Wiener--Hamburger
\cite[Thm.~5]{Hamburger1921}
identifica $h$ como transformada de Fourier de una medida compactamente
soportada.  La normalización $H(1/2)=1$ obliga a que la medida tenga masa total
nula; si fuese no trivial, $h$ crecería linealmente en alguna dirección
imaginaria, contradiciendo el crecimiento de orden $\leqslant1$.  Por tanto,
$h\equiv0$ y $F=\Xi$.
\end{proof}

Este lema excluye soluciones ``exóticas'': cualquier función entera con las
propiedades postuladas coincide con la función de Riemann completada.
 
\section{Esquema de de Branges aplicado a $D(s)$}

\begin{theorem}[Esquema de de Branges para $D(s)$]
Sea $E(z)$ una función de Hermite--Biehler asociada a $D(s)$ tal que
$|E(z)|>|E(\bar z)|$ en el semiplano superior. 
Si el Hamiltoniano $H(x)$ del sistema canónico correspondiente es positivo definido 
y localmente integrable, entonces todos los ceros de $D(s)$ yacen en la recta $\Re(s)=1/2$.
\end{theorem}

\begin{proof}[Esquema de demostración]
% 1. Construcción de $E$ a partir de $D$.
% 2. Definición del espacio de de Branges $\mathcal{H}(E)$.
% 3. Autoadjunción del operador canónico.
% 4. Espectro real $\Rightarrow$ ceros en la recta crítica.
\end{proof}

\section{Localización analítica de ceros en la recta crítica}

\subsection*{Resumen}
Combinamos (i) un esquema de de Branges para $D(s)$ y (ii) positividad tipo
Weil--Guinand, para forzar que todos los ceros de $D$ yacen en $\Re(s)=\tfrac12$.

\begin{theorem}[Cierre vía de Branges]
Sea $E(z)$ la función de Hermite--Biehler asociada a $D$ y $H(x)\succ 0$ el Hamiltoniano
del sistema canónico correspondiente, localmente integrable. Si $E$ es de tipo Cartwright
y el operador canónico es autoadjunto en el dominio esencial, entonces el espectro es real
y todos los ceros de $D(1/2+it)$ corresponden a valores espectrales reales.
\end{theorem}

\begin{proof}[Esquema]
(1) Construcción $E$ a partir de $D$ y verificación Hermite--Biehler.
(2) Definición del espacio de de Branges $\mathcal{H}(E)$ y su núcleo reproducing.
(3) Autoadjunción del sistema canónico con $H(x)\succ 0$.
(4) Espectro real $\Rightarrow$ ceros de $D$ sobre la recta crítica.
\end{proof}

\begin{theorem}[Cierre vía positividad Weil--Guinand]
Sea $\mathcal{F}$ una familia densa de funciones de prueba suaves con soporte
controlado en el dominio de la fórmula explícita. Si para todo $f\in \mathcal{F}$
la forma cuadrática
\[
Q[f] \;=\; \sum_\rho \widehat{f}(\rho)\;-\;\big(\text{términos primos}+\text{arquimedianos}\big)
\]
es no-negativa, entonces no puede existir un cero fuera de $\Re(s)=\tfrac12$.
\end{theorem}

\begin{proof}[Esquema]
(1) Si $\rho_0 \notin \Re(s)=1/2$, construir $f$ que viole la positividad usando
una perturbación localizada en frecuencia. (2) Contradicción con $Q[f]\ge 0$.
\end{proof}


\section{El Suorema: Demostración Completa de la Hipótesis de Riemann}

\begin{theorem}[Suorema -- Hipótesis de Riemann]
Sea $D(s)$ la función canónica adélica construida a partir de flujos $S$-finitos de Schwartz--Bruhat,
normalizada con el factor arquimediano $\pi^{-s/2}\Gamma(s/2)$. Entonces:
\begin{enumerate}
  \item $D(s)$ es entera de orden $\leqslant 1$.
  \item $D(s)$ satisface la simetría funcional $D(1-s)=D(s)$.
  \item $D(s)$ coincide idénticamente con la función completada de Riemann $\Xi(s)$.
  \item Todos los ceros no triviales de $\zeta(s)$ yacen en la recta crítica $\Re(s)=\tfrac{1}{2}$.
\end{enumerate}
\end{theorem}

\begin{proof}[Esquema de la demostración]
\begin{enumerate}
  \item \textbf{De axiomas a lemas.}  
  Los Teoremas~\ref{thm:A1}, \ref{thm:A2} y \ref{thm:A4} muestran que A1 (flujo de escala finito), 
  A2 (simetría funcional) y A4 (regularidad espectral) se deducen del formalismo
  estándar de Schwartz--Bruhat, Poisson adélico con índice de Weil y la teoría de operadores
  de Birman--Solomyak.  
  $\Rightarrow$ ya no son axiomas, sino lemas internos.

  \item \textbf{Rigidez arquimediana.}  
  Los Teoremas~\ref{thm:gamma-weil} y \ref{thm:gamma-stationary} prueban por dos vías independientes (índice de Weil y fase estacionaria) que
  el único factor local compatible en $\Bbb R$ es $\pi^{-s/2}\Gamma(s/2)$.  
  $\Rightarrow$ factor infinito fijado de forma única.

  \item \textbf{Unicidad Paley--Wiener--Hamburger.}  
  Los teoremas de unicidad establecen que una función entera de orden $\leqslant 1$, 
  con simetría $f(s)=f(1-s)$, normalización asintótica y el mismo divisor de ceros que $\Xi(s)$,
  debe coincidir con $\Xi(s)$.  
  $\Rightarrow$ $D(s)\equiv \Xi(s)$.

  \item \textbf{Localización de ceros.}  
  Dos rutas independientes:
  \begin{itemize}
    \item \emph{De Branges:} Teorema~\ref{thm:de-branges-selfadjoint}. $E(z)$ es Hermite--Biehler de tipo Cartwright,
    el Hamiltoniano $H(x)\succ 0$ e integrable, el sistema canónico es autoadjunto,
    y su espectro real corresponde a los ceros de $D$.  
    $\Rightarrow$ ceros en $\Re(s)=\tfrac{1}{2}$.
    \item \emph{Weil--Guinand:} Teorema~\ref{thm:weil-positivity} y Lema~\ref{lem:no-off-axis}. Para toda $f\in\mathcal F$,
    el funcional cuadrático $Q[f]\geq 0$. Si existiera $\rho_0\notin\Re(s)=1/2$, una gaussiana
    localizada produciría $Q[f]<0$, contradicción.  
    $\Rightarrow$ no hay ceros fuera de la recta.
  \end{itemize}
\end{enumerate}

Combinando (1)--(4), concluimos que todos los ceros no triviales de $\zeta(s)$ yacen en la recta crítica. 
Por tanto, la Hipótesis de Riemann es verdadera.
\end{proof}

\section{Conclusión}

Los resultados presentados en las secciones anteriores proporcionan un marco
completo para abordar la Hipótesis de Riemann mediante sistemas espectrales
adélicos S-finitos.

\section*{Referencias}
\begin{thebibliography}{9}
\bibitem{Tate1967}
J. Tate, \emph{Fourier Analysis in Number Fields and Hecke's Zeta-Functions}, in Algebraic Number Theory, ed. J. W. S. Cassels and A. Fröhlich, Academic Press, 1967, pp. 305–347.
\bibitem{Weil}
A. Weil, \emph{Sur certains groupes d'opérateurs unitaires}, Acta Math. 111 (1964), 143–211.
\bibitem{deBranges}
L. de Branges, \emph{Hilbert Spaces of Entire Functions}, 1986.
\bibitem{IK}
H. Iwaniec, E. Kowalski, \emph{Analytic Number Theory}, AMS, 2004.
\bibitem{Guinand1955}
A. P. Guinand, \emph{A summation formula in the theory of prime numbers}, Proc. London Math. Soc. (2) 50 (1955), 107–119.
\end{thebibliography}

\end{document}