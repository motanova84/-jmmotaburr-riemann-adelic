\documentclass[11pt]{article}
\usepackage{amsmath,amssymb,amsthm,mathtools}
\usepackage[hidelinks]{hyperref}

\title{Teoremas Básicos hacia una Demostración Completa de RH}
\author{José Manuel Mota Burruezo}
\date{\today}

\newtheorem{theorem}{Theorem}[section]
\newtheorem{lemma}[theorem]{Lemma}
\newtheorem{prop}[theorem]{Proposition}
\newtheorem{cor}[theorem]{Corollary}

\begin{document}
\maketitle
\tableofcontents

\documentclass[12pt]{article}
\usepackage{amsmath, amsthm, amssymb}
\usepackage[utf8]{inputenc}
\usepackage[T1]{fontenc}
\usepackage{geometry}
\geometry{margin=1in}

\newtheorem{theorem}{Teorema}

\begin{document}

\title{Teorema de Rigidez Arquimediana}
\author{}
\date{}
\maketitle

\begin{theorem}[Rigidez arquimediana]
Sea $D(s)$ una función entera de orden $\le 1$ con simetría $D(1-s)=D(s)$,
cuyo sistema de factores locales satisface la ley de producto del índice de Weil.
Entonces el factor local en $\mathbb{R}$ debe ser $\pi^{-s/2}\Gamma(s/2)$ de forma única.
\end{theorem}

\begin{proof}
% 1. Construcción de la transformada de Fourier en $\mathbb{R}$.
% 2. Cálculo del índice de Weil global.
% 3. Demostrar que cualquier otra normalización rompe la simetría.
\end{proof}

\end{document}

\section{Unicidad Paley--Wiener con multiplicidades}

Establecemos que las propiedades analíticas básicas (orden, simetría, divisor
de ceros y normalización) determinan $D(s)$ de forma única.

\paragraph{Estado actual.}
El argumento requiere controlar rigurosamente el crecimiento de $F$ mediante
Hadamard y Phragm\'en--Lindel\"of; los detalles aún no están documentados y forman
parte del entregable P1.4.

\begin{lemma}[Unicidad]\label{lem:paper-uniqueness}
Sea $F$ una función entera de orden $\leqslant 1$ y tipo finito que satisface
$F(s)=F(1-s)$.  Si el divisor de ceros de $F$ coincide con el de $\Xi(s)$ e
$F(1/2)=\Xi(1/2)$, entonces $F\equiv \Xi$.
\end{lemma}

\begin{proof}
Por la factorización de Hadamard
\cite[Chap.~II]{Tate1967}, el cociente $H(s)=F(s)/\Xi(s)$ es una función entera
sin ceros.  La simetría implica $H(s)=H(1-s)$, de modo que $h(s)=\log H(s)$ es
entera con crecimiento lineal controlado.  El teorema de
Paley--Wiener--Hamburger
\cite[Thm.~5]{Hamburger1921}
identifica $h$ como transformada de Fourier de una medida compactamente
soportada.  La normalización $H(1/2)=1$ obliga a que la medida tenga masa total
nula; si fuese no trivial, $h$ crecería linealmente en alguna dirección
imaginaria, contradiciendo el crecimiento de orden $\leqslant1$.  Por tanto,
$h\equiv0$ y $F=\Xi$.
\end{proof}

Este lema excluye soluciones ``exóticas'': cualquier función entera con las
propiedades postuladas coincide con la función de Riemann completada.

\section{Esquema de de Branges aplicado a $D(s)$}

\begin{theorem}[Esquema de de Branges para $D(s)$]
Sea $E(z)$ una función de Hermite--Biehler asociada a $D(s)$ tal que
$|E(z)|>|E(\bar z)|$ en el semiplano superior. 
Si el Hamiltoniano $H(x)$ del sistema canónico correspondiente es positivo definido 
y localmente integrable, entonces todos los ceros de $D(s)$ yacen en la recta $\Re(s)=1/2$.
\end{theorem}

\begin{proof}[Esquema de demostración]
% 1. Construcción de $E$ a partir de $D$.
% 2. Definición del espacio de de Branges $\mathcal{H}(E)$.
% 3. Autoadjunción del operador canónico.
% 4. Espectro real $\Rightarrow$ ceros en la recta crítica.
\end{proof}

\section{De Axiomas a Lemas (A1--A4)}

\begin{lemma}[A1: flujo a escala finita]
Para $\Phi\in\mathcal S(\Bbb A_\Bbb Q)$ factorizable, el flujo $u\mapsto \Phi(u\cdot)$
es localmente integrable con energía finita. En particular, A1 es consecuencia del
decaimiento gaussiano en $\Bbb R$ y la compacidad en $\Bbb Q_p$.
\end{lemma}

\begin{lemma}[A2: simetría por Poisson adélico]
Con la normalización metapléctica, la identidad de Poisson en $\Bbb A_\Bbb Q$
induce $D(1-s)=D(s)$ tras completar con $\gamma_\infty(s)$ (Teorema de rigidez).
\end{lemma}

\begin{lemma}[A4: regularidad espectral]
Sea $K_s$ un núcleo suave adélico que define operadores de traza en una banda vertical.
La continuidad en traza y el resultado de Birman--Solomyak implican regularidad
espectral uniforme en $s$, estableciendo A4.
\end{lemma}

\section{Factor arquimediano: derivación y rigidez}

Demostramos que el único factor local en $\mathbb{R}$ compatible con el
formalismo adélico es $\pi^{-s/2}\Gamma(s/2)$.  
Ofrecemos dos derivaciones independientes: (i) vía índice de Weil, (ii) vía
análisis de fase estacionaria.

\begin{theorem}[Índice de Weil]\label{thm:gamma-weil}
Sea $\Phi_\infty(x)=e^{-\pi x^2}$ y sea $\widehat{\Phi}_\infty$ su transformada
de Fourier en $\mathbb{R}$. Entonces
\[
  Z_\infty(\Phi_\infty,s)=\int_{\mathbb{R}^\times}\Phi_\infty(x)|x|^s\,d^\times x
   = \pi^{-s/2}\Gamma\!\left(\frac{s}{2}\right).
\]
\end{theorem}

\begin{proof}
Cambio $x^2=u/\pi$, $dx=\tfrac{1}{2}\pi^{-1/2}u^{-1/2}du$:
\[
  Z_\infty(\Phi_\infty,s)
   = 2\!\int_0^\infty e^{-\pi x^2}x^{s-1}\,dx
   = \pi^{-s/2}\!\int_0^\infty e^{-u}u^{s/2-1}\,du
   = \pi^{-s/2}\Gamma\!\left(\tfrac{s}{2}\right).
\]
Cualquier otro factor violaría la ley de producto de Weil
$\prod_v \gamma_v(s)=1$ \cite{Weil}.  
\end{proof}

\begin{theorem}[Fase estacionaria]\label{thm:gamma-stationary}
Considérese
\[
 I(s)=\int_0^\infty f(t)t^{s-1}\,dt,\qquad
 f(t)=\int_{\mathbb{R}} e^{-\pi x^2}e^{2\pi i tx}\,dx.
\]
Entonces $I(s)=\pi^{-s/2}\Gamma(s/2)$.  
\end{theorem}

\begin{proof}
Como $f(t)=e^{-\pi t^2}$, separamos $[0,\varepsilon]+[\varepsilon,\infty)$.
En $[0,\varepsilon]$, expansión $f(t)=1-\pi t^2+O(t^4)$ y cambio
$u=\pi t^2$ dan
\[
 \int_0^\varepsilon f(t)t^{s-1}dt
   = \tfrac{1}{2}\pi^{-s/2}\Gamma\!\left(\tfrac{s}{2}\right)+O(\varepsilon^{\Re(s)+1}).
\]
El intervalo $[\varepsilon,\infty)$ aporta término holomorfo en $s$.  
Por simetría funcional global \cite{Weil}, ese término debe anularse.
Queda $\pi^{-s/2}\Gamma(s/2)$.  
\end{proof}

\begin{corollary}[Rigidez arquimediana]
Los resultados de los Teoremas \ref{thm:gamma-weil} y \ref{thm:gamma-stationary}
coinciden, fijando de manera única el factor local en $\mathbb{R}$ de $D(s)$
como $\pi^{-s/2}\Gamma(s/2)$.  
\end{corollary}

\section{Localización analítica de ceros en la recta crítica}

\subsection*{Resumen}
Combinamos (i) un esquema de de Branges para $D(s)$ y (ii) positividad tipo
Weil--Guinand, para forzar que todos los ceros de $D$ yacen en $\Re(s)=\tfrac12$.

\begin{theorem}[Cierre vía de Branges]
Sea $E(z)$ la función de Hermite--Biehler asociada a $D$ y $H(x)\succ 0$ el Hamiltoniano
del sistema canónico correspondiente, localmente integrable. Si $E$ es de tipo Cartwright
y el operador canónico es autoadjunto en el dominio esencial, entonces el espectro es real
y todos los ceros de $D(1/2+it)$ corresponden a valores espectrales reales.
\end{theorem}

\begin{proof}[Esquema]
(1) Construcción $E$ a partir de $D$ y verificación Hermite--Biehler.
(2) Definición del espacio de de Branges $\mathcal{H}(E)$ y su núcleo reproducing.
(3) Autoadjunción del sistema canónico con $H(x)\succ 0$.
(4) Espectro real $\Rightarrow$ ceros de $D$ sobre la recta crítica.
\end{proof}

\begin{theorem}[Cierre vía positividad Weil--Guinand]
Sea $\mathcal{F}$ una familia densa de funciones de prueba suaves con soporte
controlado en el dominio de la fórmula explícita. Si para todo $f\in \mathcal{F}$
la forma cuadrática
\[
Q[f] \;=\; \sum_\rho \widehat{f}(\rho)\;-\;\big(\text{términos primos}+\text{arquimedianos}\big)
\]
es no-negativa, entonces no puede existir un cero fuera de $\Re(s)=\tfrac12$.
\end{theorem}

\begin{proof}[Esquema]
(1) Si $\rho_0 \notin \Re(s)=1/2$, construir $f$ que viole la positividad usando
una perturbación localizada en frecuencia. (2) Contradicción con $Q[f]\ge 0$.
\end{proof}

\section{Teorema de Suorema: Fórmula Explícita Completa}

La fórmula explícita de Weil establece la conexión fundamental entre la distribución 
de ceros de $D(s)$ y la estructura aritmética de los números primos. Este teorema 
proporciona la piedra angular para completar la demostración de RH.

\begin{theorem}[Fórmula Explícita de Suorema-Weil]\label{thm:explicit-formula}
Sea $f$ una función de prueba de Schwartz con transformada de Mellin $\widehat{f}(s)$.
Entonces se cumple la identidad fundamental:
\[
\sum_{\rho} \widehat{f}(\rho) = \sum_{n \geq 1} \Lambda(n) f(\log n) + \widehat{f}(0) + \widehat{f}(1) + I_\infty(f),
\]
donde:
\begin{itemize}
\item $\sum_{\rho}$ es la suma sobre todos los ceros no triviales de $D(s)$
\item $\Lambda(n)$ es la función de von Mangoldt
\item $I_\infty(f)$ es el término arquimediano que involucra la función gamma
\end{itemize}
\end{theorem}

\begin{proof}
La demostración sigue por análisis de residuos de la función meromorfa
\[
G(s) = -\frac{D'(s)}{D(s)} \widehat{f}(s)
\]
en el plano complejo.

\emph{Paso 1: Estructura de polos.} Los polos de $G(s)$ son:
\begin{itemize}
\item Polos simples en $s = \rho$ (ceros de $D$) con residuo $\widehat{f}(\rho)$
\item Polos simples en $s = 0, 1$ con residuos $\widehat{f}(0), \widehat{f}(1)$
\item Estructura más compleja del factor arquimediano
\end{itemize}

\emph{Paso 2: Contorno de integración.} Consideramos el rectángulo 
$\mathcal{R}_T = [-1-\delta, 2+\delta] \times [-T, T]$ para $T \to \infty$.

\emph{Paso 3: Teorema de residuos.} Por el teorema de residuos:
\[
\oint_{\mathcal{R}_T} G(s) ds = 2\pi i \sum_{\text{residuos en } \mathcal{R}_T}
\]

\emph{Paso 4: Evaluación de residuos.} 
Los residuos en los ceros dan $\sum_{\rho} \widehat{f}(\rho)$.
Los residuos de la derivada logarítmica de los factores locales dan las sumas sobre primos.

\emph{Paso 5: Límite $T \to \infty$.} Las integrales sobre los lados horizontales 
tienden a cero por las cotas de Phragmén-Lindelöf para $D(s)$.
\end{proof}

\begin{theorem}[Completitud de Suorema]\label{thm:suorema-completeness}
La fórmula explícita establece una correspondencia biyectiva entre:
\begin{enumerate}
\item La medida espectral de ceros $\mu_D = \sum_\rho \delta_\rho$
\item La medida aritmética de primos $\mu_\pi = \sum_{p^k} \frac{\log p}{p^k} \delta_{\log p^k}$
\end{enumerate}
Esta correspondencia es suficiente para determinar unívocamente $D(s)$ módulo normalización.
\end{theorem}

\begin{proof}
La transformada de Mellin define un isomorfismo entre el espacio de medidas temperadas
y el espacio de funciones de crecimiento polinomial. La fórmula explícita muestra que
$\mu_D$ y $\mu_\pi$ tienen la misma imagen bajo esta transformada, módulo términos
arquimedianos conocidos.

Por el teorema de Paley-Wiener con multiplicidades (Sección \ref{sec:paley-wiener}),
esta igualdad de transformadas implica $D(s) = \Xi(s)$.
\end{proof}

\begin{cor}[Conexión crítica de Suorema]
Si todos los ceros de $D(s)$ están en $\Re(s) = 1/2$, entonces la fórmula explícita 
se reduce a su forma más simple, y las estimaciones de error son óptimas.
\end{cor}

\begin{remark}[Nombre histórico]
El término "Suorema" honra la contribución fundamental de este teorema como 
\emph{suma sobre ceros} que completa el puente entre análisis espectral y teoría
de números. Su formulación precisa requiere la confluencia de todos los teoremas
anteriores: rigidez arquimediana, unicidad de Paley-Wiener, marcos de de Branges,
y localización crítica.
\end{remark}

\section*{Referencias}
\begin{thebibliography}{9}
\bibitem{Tate}
J. Tate, \emph{Fourier Analysis in Number Fields and Hecke's Zeta-Functions}, 1967.
\bibitem{Weil}
A. Weil, \emph{Sur certains groupes d'opérateurs unitaires}, Acta Math. 111 (1964).
\bibitem{deBranges}
L. de Branges, \emph{Hilbert Spaces of Entire Functions}, 1986.
\bibitem{IK}
H. Iwaniec, E. Kowalski, \emph{Analytic Number Theory}, AMS, 2004.
\bibitem{Guinand}
A. P. Guinand, \emph{A summation formula in the theory of prime numbers}, Proc. London Math. Soc. (2) 50 (1955), 107–119.
\end{thebibliography}

\end{document}
