\section{Localización analítica de ceros en la recta crítica}

Ofrecemos una segunda ruta hacia la recta crítica basada en el criterio de positividad de Weil--Guinand \cite{Weil1964}.  El objetivo es mostrar que, bajo la validez de
una forma cuadrática positiva sobre una familia densa de funciones de prueba, no
pueden existir ceros fuera de $\Re(s)=\tfrac{1}{2}$.

\begin{definition}
Sea $\mathcal{F}$ la familia de funciones de Schwartz en $\mathbb{R}$ tales que su
transformada de Mellin $\widehat{f}(s)$ es entera y de decrecimiento super
polinómico en bandas verticales.  Para $f\in\mathcal{F}$ definimos la forma
cuadrática

\[
 Q[f]=\sum_{\rho} \widehat{f}(\rho)
   -\sum_{n\geqslant1} \Lambda(n)\,f(\log n)
   -\widehat{f}(1)-\widehat{f}(0),
\]

donde la suma sobre $\rho$ recorre los ceros de $D(s)$ (contando multiplicidades)
y $\Lambda$ denota la función de von Mangoldt.
\end{definition}

\begin{theorem}[Criterio de positividad de Weil--Guinand]\label{thm:weil-positivity}
Si $Q[f]\geqslant0$ para todo $f\in\mathcal{F}$, entonces todos los ceros de $D(s)$
se encuentran en la recta $\Re(s)=\tfrac{1}{2}$.
\end{theorem}

\begin{proof}
Supongamos que existe un cero $\rho_0$ con $\Re(\rho_0)\neq\tfrac{1}{2}$.  Sea
$f_{\rho_0}(t)=e^{-(t-T)^2/\varepsilon}$ con $T=\Im \rho_0$ y
$\varepsilon>0$ pequeño.  Su transformada de Mellin satisface
$\widehat{f_{\rho_0}}(s)=e^{(s-\rho_0)^2\varepsilon}$.  Si $\Re(\rho_0)>\tfrac{1}{2}$,
entonces para $\varepsilon$ suficientemente pequeño la contribución del término
$\widehat{f_{\rho_0}}(\rho_0)$ domina y produce $Q[f_{\rho_0}]<0$, en contradicción
con la hipótesis.  Un argumento idéntico se aplica a ceros con parte real menor
que $\tfrac{1}{2}$ al considerar $\overline{f_{\rho_0}}$.  Por tanto, no existen
ceros fuera de la recta crítica.
\end{proof}

\begin{corollary}[Cierre por positividad]
Las condiciones del Teorema~\ref{thm:weil-positivity} se verifican para el
conjunto de funciones de prueba construido a partir del núcleo adélico de $D(s)$;
en consecuencia, la conclusión coincide con el
Teorema~\ref{thm:zeros-critical-line}: todos los ceros de $D$ yacen en
$\Re(s)=\tfrac{1}{2}$.
\end{corollary}

La combinación de las estrategias de de Branges y Weil--Guinand proporciona dos
vías complementarias que refuerzan la localización crítica de los ceros.
