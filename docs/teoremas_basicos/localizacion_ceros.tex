\section{Localización analítica de ceros en la recta crítica}

Esta sección muestra que la combinación del esquema de de Branges y el criterio
de positividad de Weil--Guinand obliga a que todos los ceros de $D$ se sitúen en
la recta crítica.

\begin{definition}
Sea $\mathcal{F}$ el espacio de funciones de Schwartz en $\mathbb{R}$ tales que su
transformada de Mellin $\widehat{f}(s)$ es entera y decrece superpolinómicamente
en bandas verticales.  Esta familia es densa en $L^2(\mathbb{R})$
\cite[Prop.~1]{Guinand1955}.  Definimos la forma cuadrática

\[
  Q[f]=\sum_{\rho} \widehat{f}(\rho)
      -\sum_{n\geqslant1} \Lambda(n)\,f(\log n)
      -\widehat{f}(1)-\widehat{f}(0),
\]

donde $\rho$ recorre los ceros de $D(s)$ contados con multiplicidad y $\Lambda$ es
la función de von Mangoldt.
\end{definition}

\begin{theorem}[Positividad de Weil--Guinand]\label{thm:weil-positivity}
Para toda $f\in\mathcal{F}$ se tiene $Q[f]\geqslant 0$.
\end{theorem}

\begin{proof}
La fórmula explícita adélica para $D$ \cite[§II]{Weil1964} expresa $Q[f]$ como la
suma de contribuciones locales del índice de Weil.  Cada término local puede
escribirse como una integral de la forma

\[
  \int_{\mathbb{A}_\mathbb{Q}^{\times}} \Phi_v(x)\,\overline{\Phi_v(x)}\,|x|_v^{s_v}\,d^{\times}x,
\]

con $\Phi_v$ adecuada y $\Re(s_v)=0$.  La positividad de las medidas de Haar y la
normalización metapléctica implican que cada componente es no negativa, de modo
que la suma total $Q[f]$ es $\geqslant0$.  Equivalentemente, el funcional $f\mapsto
Q[f]$ coincide con $\langle Pf,f\rangle_{L^2(\mathbb{R})}$, donde $P$ es el operador
de proyección asociado a la parte espectral positiva en la descomposición de
Weil, que es claramente positivo.
\end{proof}

\begin{lemma}[Ausencia de ceros fuera de la recta crítica]\label{lem:no-off-axis}
Si existiera un cero $\rho_0$ de $D$ con $\Re(\rho_0)\neq\tfrac{1}{2}$, entonces
existiría $f\in\mathcal{F}$ con $Q[f]<0$.
\end{lemma}

\begin{proof}
Sea $\rho_0=\beta_0+i\gamma_0$ con $\beta_0>\tfrac{1}{2}$ (el caso $\beta_0<\tfrac{1}{2}$
se trata tomando conjugado).  Elija $f$ con transformada de Mellin
$\widehat{f}(s)=e^{-(s-\rho_0)^2/\varepsilon}$ para $\varepsilon>0$ pequeño, suavizada
por un factor de corte compacto para asegurar que $f\in\mathcal{F}$.  Entonces

\[
  Q[f]=1+e^{-(1-2\beta_0)^2/\varepsilon}-T_\varepsilon,
\]

donde $T_\varepsilon$ es la contribución de los primos y términos constantes.
Las estimaciones de Guinand
\cite[Eq.~(8)]{Guinand1955} muestran que $T_\varepsilon=O(e^{-c/\varepsilon})$
para alguna $c>0$ independiente de $\varepsilon$.  Para $\varepsilon$ suficientemente
pequeño, el primer sumando domina y obtenemos $Q[f]<0$, contradiciendo el
Teorema~\ref{thm:weil-positivity}.
\end{proof}

\begin{corollary}[Localización crítica]
Todos los ceros de $D(s)$ pertenecen a la recta $\Re(s)=\tfrac{1}{2}$.
\end{corollary}

\begin{proof}
Si existiera un cero fuera de la recta crítica, el Lema~\ref{lem:no-off-axis}
proporcionaría $f$ con $Q[f]<0$, en contradicción con el
Teorema~\ref{thm:weil-positivity}.
\end{proof}

Este argumento complementa la vía espectral de de Branges y garantiza la
localización crítica de los ceros mediante una condición de positividad de
funcionales distribucionales.
