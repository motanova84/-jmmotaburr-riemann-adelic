\section{Localización analítica de ceros en la recta crítica}

\subsection*{Resumen}
Combinamos (i) un esquema de de Branges para $D(s)$ y (ii) positividad tipo
Weil--Guinand, para forzar que todos los ceros de $D$ yacen en $\Re(s)=\tfrac12$.

\begin{theorem}[Cierre vía de Branges]
Sea $E(z)$ la función de Hermite--Biehler asociada a $D$ y $H(x)\succ 0$ el Hamiltoniano
del sistema canónico correspondiente, localmente integrable. Si $E$ es de tipo Cartwright
y el operador canónico es autoadjunto en el dominio esencial, entonces el espectro es real
y todos los ceros de $D(1/2+it)$ corresponden a valores espectrales reales.
\end{theorem}

\begin{proof}[Esquema]
(1) Construcción $E$ a partir de $D$ y verificación Hermite--Biehler.
(2) Definición del espacio de de Branges $\mathcal{H}(E)$ y su núcleo reproducing.
(3) Autoadjunción del sistema canónico con $H(x)\succ 0$.
(4) Espectro real $\Rightarrow$ ceros de $D$ sobre la recta crítica.
\end{proof}

\begin{theorem}[Cierre vía positividad Weil--Guinand]
Sea $\mathcal{F}$ una familia densa de funciones de prueba suaves con soporte
controlado en el dominio de la fórmula explícita. Si para todo $f\in \mathcal{F}$
la forma cuadrática
\[
Q[f] \;=\; \sum_\rho \widehat{f}(\rho)\;-\;\big(\text{términos primos}+\text{arquimedianos}\big)
\]
es no-negativa, entonces no puede existir un cero fuera de $\Re(s)=\tfrac12$.
\end{theorem}

\begin{proof}[Esquema]
(1) Si $\rho_0 \notin \Re(s)=1/2$, construir $f$ que viole la positividad usando
una perturbación localizada en frecuencia. (2) Contradicción con $Q[f]\ge 0$.
\end{proof}
