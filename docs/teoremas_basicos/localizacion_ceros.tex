\section{Localización analítica de ceros}

\newtheorem{theoremD}{Theorem}[section]
\newtheorem{lemmaD}[theoremD]{Lemma}

\subsection*{Ruta A: de Branges}
Combina los resultados de la sección anterior: si $E$ es HB, $H\succ 0$ e integrable,
y el operador es autoadjunto, el espectro es real $\Rightarrow$ ceros en la recta crítica.

\subsection*{Ruta B: Positividad tipo Weil--Guinand}

\begin{theoremD}[Criterio de positividad]
Sea $\mathcal F$ una familia densa de funciones de prueba suaves
cuyas transformadas de Mellin/Paley--Wiener cumplen las hipótesis estándar.
Si para todo $f\in\mathcal F$ la forma
\[
Q[f]=\sum_{\rho}\widehat f(\rho)-\Big(\sum_{n\ge1}\Lambda(n)f(\log n)+\text{término arquimediano}\Big)\ge 0,
\]
entonces no existen ceros de $D$ fuera de $\Re(s)=\tfrac{1}{2}$.
\end{theoremD}

\begin{proof}[Esquema]
Si existiera $\rho_0$ con $\Re(\rho_0)\neq \tfrac{1}{2}$, se construye $f$
concentrada en frecuencia cerca de $\rho_0$ (ajustada por simetrías) que hace
$Q[f]<0$, contradiciendo la positividad (véase el criterio de Weil).
\end{proof}
