\section{Factor arquimediano: derivación y rigidez}

Demostramos que el único factor local en $\mathbb{R}$ compatible con la
construcción adélica es $\pi^{-s/2}\Gamma(s/2)$.  Ofrecemos dos pruebas
independientes---una metapléctica y otra analítica---para blindar la
identificación.

\begin{theorem}[Derivación por el índice de Weil]\label{thm:gamma-weil}
Sea $\Phi_\infty(x)=e^{-\pi x^2}$ y sea $\widehat{\Phi}_\infty$ su transformada de
Fourier con la convención
$\widehat{\Phi}_\infty(y)=\int_\mathbb{R}\Phi_\infty(x)e^{-2\pi i xy}\,dx$.  Entonces

\[
  Z_\infty(\Phi_\infty,s)=\int_{\mathbb{R}^{\times}}\Phi_\infty(x)|x|^{s}\,d^{\times}x
  =\pi^{-s/2}\Gamma\!\left(\frac{s}{2}\right),
\]

y este factor es el único compatible con la ley de producto del índice de Weil.
\end{theorem}

\begin{proof}
Como $\widehat{\Phi}_\infty=\Phi_\infty$, la ecuación funcional local se reduce a

\[
  \gamma_\infty(s)Z_\infty(\Phi_\infty,s)=\gamma_\infty(1-s)Z_\infty(\Phi_\infty,1-s).
\]

Calculamos el integral aplicando el cambio $x=e^t$ y usando $d^{\times}x=dt$:

\[
  Z_\infty(\Phi_\infty,s)
   = 2\int_0^{\infty} e^{-\pi x^2}x^{s-1}\,dx
   = 2\cdot\frac{1}{2}\pi^{-s/2}\Gamma\!\left(\frac{s}{2}\right),
\]

por la definición clásica de la función gamma.  Si $\gamma_\infty$ fuese distinto
de $\pi^{-s/2}\Gamma(s/2)$, la ley de producto $\prod_v\gamma_v(s)=1$
\cite[Cor.~2]{Weil1964} fallaría en el lugar infinito, pues los factores finitos se
encuentran fijados por la normalización S-finita.  De aquí la unicidad.
\end{proof}

\begin{theorem}[Derivación por fase estacionaria]\label{thm:gamma-stationary}
Considere la contribución arquimediana del lado geométrico de la fórmula
explícita, expresada como

\[
  I(s)=\int_0^{\infty} f(t)t^{s-1}\,dt,
  \qquad f(t)=\int_{\mathbb{R}} e^{-\pi x^2}e^{2\pi i tx}\,dx.
\]

Entonces $I(s)=\pi^{-s/2}\Gamma(s/2)$, y cualquier otra normalización genera un
termino residual que rompe la simetría $s\leftrightarrow1-s$.
\end{theorem}

\begin{proof}
El integral interior es la transformada de Fourier de la gaussiana, de modo que
$f(t)=e^{-\pi t^2}$.  Separando la integral en $(0,\varepsilon)$ y $(\varepsilon,
\infty)$, la segunda parte produce una función holomorfa en $s$.  En la vecindad
de $0$ usamos $f(t)=1-\pi t^2+O(t^4)$, obteniendo

\[
  \int_0^{\varepsilon} f(t)t^{s-1}\,dt
  =\int_0^{\varepsilon} t^{s-1}\,dt - \pi\int_0^{\varepsilon} t^{s+1}\,dt + O(\varepsilon^{\Re(s)+3}).
\]

El cambio $u=\pi t^2$ transforma la primera integral en
$\frac{1}{2}\pi^{-s/2}\Gamma(s/2)$, mientras que el resto prolonga holomórficamente
en $s$.  Extender $\varepsilon$ a infinito añade únicamente una función entera.  Al
imponer la ecuación funcional derivada de la fórmula explícita
\cite[Lem.~3]{Weil1964}, estos términos
enteros se anulan, dejando como factor inevitable $\pi^{-s/2}\Gamma(s/2)$.
\end{proof}

\begin{corollary}[Rigidez arquimediana]
Los Teoremas \ref{thm:gamma-weil} y \ref{thm:gamma-stationary} coinciden y
proporcionan el mismo factor local.  Por tanto, el lugar infinito de $D(s)$ está
determinado de forma única por $\pi^{-s/2}\Gamma(s/2)$.
\end{corollary}

Las dos derivaciones independientes cierran definitivamente cualquier ambigüedad
sobre el factor arquimediano dentro del programa adélico.
