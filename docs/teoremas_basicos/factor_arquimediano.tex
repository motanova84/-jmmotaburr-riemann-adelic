\section{Factor arquimediano: derivación y rigidez}

\subsection*{Resumen}
Damos dos derivaciones independientes que fijan de manera única el factor infinito:
$\pi^{-s/2}\Gamma(s/2)$.

\begin{theorem}[Derivación por índice de Weil]
Con la normalización estándar de la transformada de Fourier en $\mathbb{R}$ y el índice
de Weil, el único factor arquimediano compatible con la simetría $s\mapsto 1-s$ y la
ley de producto global es $\pi^{-s/2}\Gamma(s/2)$.
\end{theorem}

\begin{proof}[Borrador]
(1) Usar la gaussiana $e^{-\pi x^2}$ y su invariancia. (2) Cálculo del índice local.
(3) La reciprocidad global impone $\prod_v \gamma_v(s)=1$. (4) Los factores finitos
quedan fijados; el infinito se determina como $\pi^{-s/2}\Gamma(s/2)$.
\end{proof}

\begin{theorem}[Derivación por fase estacionaria]
El análisis de fase estacionaria del integral arquimediano que aparece en la fórmula
explícita conduce al mismo factor $\pi^{-s/2}\Gamma(s/2)$, de modo que cualquier
otra normalización contradice el balance local-global.
\end{theorem}

\begin{proof}[Borrador]
(1) Reescribir el término infinito como integral oscilatoria. (2) Evaluar la fase crítica.
(3) Extraer la constante global y comparar con la versión de índice de Weil.
\end{proof}
