\section{Factor arquimediano: derivación y rigidez}

En esta sección justificamos que el único factor local infinito compatible con la
construcción adélica y la simetría funcional es $\pi^{-s/2}\Gamma(s/2)$.  Proveemos
dos argumentos independientes: uno basado en el índice de Weil real y otro en el
análisis de fase estacionaria aplicado al integral arquimediano de la fórmula
explícita.

\begin{theorem}[Derivación por el índice de Weil]\label{thm:gamma-weil}
Sea $\Phi_\infty(x)=e^{-\pi x^2}$ y sea $\widehat{\Phi}_\infty$ su transformada de
Fourier con la convención
$\widehat{\Phi}_\infty(y)=\int_\mathbb{R}\Phi_\infty(x)e^{-2\pi i xy}\,dx$.
Entonces la contribución local del lugar infinito en el zeta-integral de Tate es
exactamente $\pi^{-s/2}\Gamma(s/2)$ y constituye la única opción compatible con la
ley de producto del índice de Weil.
\end{theorem}

\begin{proof}
El cálculo explícito de la transformada muestra que
$\widehat{\Phi}_\infty=\Phi_\infty$, de modo que la igualdad de Poisson local se
lee como $\Phi_\infty(0)=\widehat{\Phi}_\infty(0)$.  Para el zeta-integral obtenemos

\[
 Z_\infty(\Phi_\infty,s)
 =\int_{\mathbb{R}^\times} \Phi_\infty(x) |x|^{s}\,d^\times x
 =\pi^{-s/2}\Gamma\!\left(\frac{s}{2}\right),
\]

donde usamos el cambio de variable $x=e^t$ y la definición de la función gamma
clásica.  Para un factor local arbitrario $\gamma_\infty(s)$, la ecuación
funcional $Z(\widehat{\Phi},1-s)=Z(\Phi,s)$ exige que

\[
 \gamma_\infty(s)\,Z_\infty(\Phi_\infty,s)
 =\gamma_\infty(1-s)\,Z_\infty(\widehat{\Phi}_\infty,1-s).
\]

Debido a que $Z_\infty(\Phi_\infty,s)=Z_\infty(\widehat{\Phi}_\infty,s)$ y a la ley
de producto $\prod_v \gamma_v(s)=1$ \cite{Weil1964}, la única solución meromorfa
del sistema es $\gamma_\infty(s)=\pi^{-s/2}\Gamma(s/2)$.  Cualquier otra elección
introduciría una discrepancia en el signo global, contradiciendo la reciprocidad.
\end{proof}

\begin{theorem}[Derivación por fase estacionaria]\label{thm:gamma-stationary}
Considere la contribución arquimediana del lado geométrico en la fórmula
explícita, dada (tras reducción adélica) por

\[
 I(s)=\int_0^\infty f(t)\,t^{s-1}\,dt,
\]

donde $f(t)=2\cos(2\pi t)$ es la transformada de Mellin de la gaussiana.
Entonces, aplicando el método de fase estacionaria al punto crítico $t=0$, se
recupera el mismo factor $\pi^{-s/2}\Gamma(s/2)$.
\end{theorem}

\begin{proof}
Descomponemos el integral en una vecindad $t\in(0,\varepsilon)$ y en su complemento.
En la parte compacta $[\varepsilon,\infty)$ el integrando es analítico y no
contribuye a la singularidad.  Cerca de $t=0$ escribimos
$f(t)=2\exp(-2\pi^2 t^2)+O(t^4)$, de modo que

\[
 I(s)=\int_0^\varepsilon 2\exp(-2\pi^2 t^2)t^{s-1}\,dt + H(s),
\]

donde $H(s)$ es holomorfa.  El cambio $u=2\pi t$ produce

\[
 I(s)=\pi^{-s}\int_0^{2\pi \varepsilon} e^{-u^2/2} u^{s-1}\,du + H(s)
 =\pi^{-s/2}\Gamma\!\left(\frac{s}{2}\right)+H(s),
\]

tras extender el integral hasta el infinito utilizando que la cola
$\int_{2\pi \varepsilon}^\infty e^{-u^2/2}u^{s-1}\,du$ es una función entera de
$s$.  De nuevo, imponer la simetría $s\mapsto1-s$ obliga a que la contribución
holomorfa $H(s)$ sea invariante y, por tanto, absorbible en el lado algebraico de
la fórmula explícita.  Como consecuencia, el factor singular inevitable es
$\pi^{-s/2}\Gamma(s/2)$.
\end{proof}

\begin{corollary}[Rigidez arquimediana]
Las ecuaciones funcionales obtenidas en los teoremas
\ref{thm:gamma-weil} y \ref{thm:gamma-stationary} coinciden; por tanto, el único
factor infinito compatible con la construcción adélica de $D(s)$ y con la fórmula
de producto global es $\pi^{-s/2}\Gamma(s/2)$.
\end{corollary}

Los dos enfoques---metapléctico y analítico---producen la misma constante,
blindando la identificación del factor arquimediano y eliminando posibles ambigüedades.
