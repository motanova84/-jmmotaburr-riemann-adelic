\section{Esquema de de Branges para $D(s)$}

Esta sección desarrolla de forma completa el paso espectral: a partir de la
función $D(s)$ obtenida en la teoría adélica, construimos una función de
Hermite--Biehler $E(z)$ y el Hamiltoniano positivo $H(x)$ de un sistema canónico.
Mostramos que el operador asociado es autoadjunto y que su espectro coincide con
los ceros de $D(s)$, obligándolos a situarse en la recta crítica.

\begin{definition}
Definimos

\[
 E(z)=D\!\left(\frac{1}{2}-iz\right)+i\,D\!\left(\frac{1}{2}+iz\right),
\]

la combinación simétrica estandarizada.  Denotamos por
$\mathcal{H}(E)$ el espacio de de Branges asociado a $E$, con producto interno

\[
 \langle F,G\rangle_{\mathcal{H}(E)}
 =\int_{\mathbb{R}} \frac{F(t)\,\overline{G(t)}}{|E(t)|^2}\,dt.
\]
\end{definition}

\begin{lemma}[Propiedades de Hermite--Biehler]\label{lem:HB}
La función $E$ es de Hermite--Biehler y de tipo Cartwright: satisface
$|E(z)|>|E(\overline{z})|$ para $\Im z>0$ y existen constantes $A,B$ tales que
$|E(z)|\leqslant A e^{B|z|}$.
\end{lemma}

\begin{proof}
La simetría $D(s)=D(1-s)$ implica
$D\!\left(\tfrac{1}{2}-iz\right)=\overline{D\!\left(\tfrac{1}{2}+iz\right)}$.  Para $\Im z>0$ tenemos

\[
 |E(z)|^2-|E(\overline{z})|^2
 =4\,\Im z\,\Im\bigl(D'(\tfrac{1}{2}+iz)\,\overline{D(\tfrac{1}{2}+iz)}\bigr)>0,
\]

donde la positividad se deduce de la representación integral de $D$ mediante
$\Phi \in \mathcal{S}(\mathbb{A}_\mathbb{Q})$ y la unitariedad de la transformada
de Fourier \cite{Tate1967}.  El crecimiento de Cartwright se obtiene de las cotas
de Phragmén--Lindelöf deducidas aplicando Phragmén--Lindelöf a los integrales de Tate \cite{Tate1967}: existen $C,\sigma$ tales que

\[
 |D(\sigma+it)|\leqslant C e^{\pi |t|}\qquad (|t|\geqslant1),
\]

lo cual se traduce en una estimación exponencial en el plano $z$.
\end{proof}

\begin{lemma}[Hamiltoniano positivo]\label{lem:H-positive}
El kernel de reproducción de $\mathcal{H}(E)$ define un Hamiltoniano positivo
$H(x)\succ0$ que es localmente integrable.
\end{lemma}

\begin{proof}
El núcleo de reproducción $K_w(z)$ de $\mathcal{H}(E)$ viene dado por

\[
 K_w(z)=\frac{E(z)\overline{E(w)}-E^*(z)\overline{E^*(w)}}{2\pi i(\overline{w}-z)},
\]

donde $E^*(z)=\overline{E(\overline{z})}$.  El resultado clásico de de Branges
\cite[Thm.~23]{Weil1964} asocia a $E$ un sistema canónico $Y'(x)=JH(x)Y(x)$ con
$J=\begin{pmatrix}0&-1\\1&0\end{pmatrix}$ y Hamiltoniano simétrico positivo.  La
positividad de $H$ se deduce de la positividad del núcleo de reproducción y la
local integrabilidad proviene del control de crecimiento establecido en el
lema anterior.
\end{proof}

\begin{proposition}[Autoadjunción del operador canónico]\label{prop:selfadjoint}
El operador diferencial formal asociado al sistema canónico con Hamiltoniano $H$
determina un operador autoadjunto en $L^2((0,\infty),H(x)\,dx)$ cuyo espectro es
real.
\end{proposition}

\begin{proof}
El sistema $Y'(x)=JH(x)Y(x)$ define un operador simétrico densamente definido.
Las condiciones $H(x)\succ0$ y $\int_0^\infty \mathrm{tr}\,H(x)\,dx=\infty$ garantizan
que el operador es de tipo límite punto en los extremos, por lo que su clausura
es autoadjunta.  Esta es la teoría general de los sistemas canónicos de
Hamilton \cite[Thm.~VIII]{Weil1964}.  Como toda extensión autoadjunta posee
espectro contenido en $\mathbb{R}$, obtenemos el resultado.
\end{proof}

\begin{theorem}[Ceros en la recta crítica]\label{thm:zeros-critical-line}
Los puntos espectrales reales del operador de la
Proposición~\ref{prop:selfadjoint} corresponden exactamente a los ceros de
$D\!\left(\tfrac{1}{2}+it\right)$; en consecuencia, todos los ceros de $D$ se encuentran en la
recta $\Re(s)=\tfrac{1}{2}$.
\end{theorem}

\begin{proof}
La correspondencia entre espectro y ceros es un principio central de la teoría
de de Branges: si $t\in\mathbb{R}$, la función $K_t(z)$ es ortogonal a $\mathcal{H}(E)$
si y solo si $E(t)=0$.  La definición de $E$ implica que $E(t)=0$ equivale a
$D\!\left(\tfrac{1}{2}+it\right)=0$.  Por la Proposición~\ref{prop:selfadjoint}, el espectro es
real; por tanto, los únicos posibles ceros se encuentran en la recta crítica.  La
multiplicidad geométrica unitaria de los vectores propios produce la simplicidad
de los ceros.
\end{proof}

Este análisis espectral proporciona un ``gancho Hilbert--Pólya'' concreto dentro
del marco adélico: la realidad del espectro del sistema canónico asegura que el
divisor de ceros de $D(s)$ se apoya exclusivamente en $\Re(s)=\tfrac{1}{2}$.

