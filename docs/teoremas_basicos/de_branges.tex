\section{Esquema de de Branges aplicado a $D(s)$}

\begin{theorem}[Esquema de de Branges para $D(s)$]
Sea $E(z)$ una función de Hermite--Biehler asociada a $D(s)$ tal que
$|E(z)|>|E(\bar z)|$ en el semiplano superior. 
Si el Hamiltoniano $H(x)$ del sistema canónico correspondiente es positivo definido 
y localmente integrable, entonces todos los ceros de $D(s)$ yacen en la recta $\Re(s)=1/2$.
\end{theorem}

\begin{proof}[Esquema]
1. Construir $E$ a partir de $D$: $E(z)=D(1/2-iz)+iD(1/2+iz)$.
2. Definir el espacio de de Branges $\mathcal{H}(E)$ con la norma inducida.
3. Mostrar que el operador canónico asociado a $H(x)$ es autoadjunto.
4. El espectro real del operador corresponde a los ceros de $D$,
   que deben situarse en $\Re(s)=1/2$.
\end{proof}
