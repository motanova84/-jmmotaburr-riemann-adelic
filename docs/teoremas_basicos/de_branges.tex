\section{Esquema de de Branges aplicado a $D(s)$}

\begin{theorem}[Esquema de de Branges para $D(s)$]
Sea $E(z)$ una función de Hermite--Biehler asociada a $D(s)$ tal que
$|E(z)|>|E(\bar z)|$ en el semiplano superior. 
Si el Hamiltoniano $H(x)$ del sistema canónico correspondiente es positivo definido 
y localmente integrable, entonces todos los ceros de $D(s)$ yacen en la recta $\Re(s)=1/2$.
\end{theorem}

\begin{proof}[Esquema de demostración]
% 1. Construcción de $E$ a partir de $D$.
% 2. Definición del espacio de de Branges $\mathcal{H}(E)$.
% 3. Autoadjunción del operador canónico.
% 4. Espectro real $\Rightarrow$ ceros en la recta crítica.
\end{proof}
