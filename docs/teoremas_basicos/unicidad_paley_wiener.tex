\section{Unicidad Paley--Wiener con multiplicidades}\label{sec:unicidad}

El teorema de unicidad es crucial para establecer que $D(s) \equiv \Xi(s)$, eliminando cualquier ambigüedad en la identificación y completando el paso más delicado hacia la Coronación V5.

\begin{theorem}[Unicidad Paley--Wiener--Hamburger]\label{thm:unicidad-principal}
Sea $F(s)$ una función entera de orden $\leq 1$ y tipo finito, con simetría $F(1-s)=F(s)$.
Suponga que $F$ y $\Xi(s)$ (la función completada de Riemann) tienen la misma medida
espectral de ceros incluyendo multiplicidades y que $F(1/2)=\Xi(1/2)\neq 0$.
Entonces $F\equiv \Xi$.
\end{theorem}

\begin{proof}
\textbf{Paso 1: Representación de Hadamard.}
Por teoría de Hadamard para funciones enteras de orden $\leq 1$, tanto $F$ como $\Xi$
admiten productos canónicos:
\[
F(s)=e^{a+bs}\prod_\rho E_1\!\left(\frac{s}{\rho}\right),\qquad
\Xi(s)=e^{a'+b's}\prod_\rho E_1\!\left(\frac{s}{\rho}\right),
\]
donde el producto es sobre los mismos ceros (con multiplicidad) por hipótesis,
y $E_1(z)=(1-z)e^{z}$ es el factor primario de Weierstrass.

\textbf{Paso 2: Análisis de la razón.}
La razón $H(s):=\frac{F(s)}{\Xi(s)}$ es entera sin ceros ni polos, luego por el teorema de Liouville generalizado:
\[
H(s)=e^{c+ds}
\]
para constantes $c, d \in \mathbb{C}$.

\textbf{Paso 3: Imposición de simetría.}
Las simetrías $F(1-s)=F(s)$ y $\Xi(1-s)=\Xi(s)$ implican:
\[
H(1-s)=\frac{F(1-s)}{\Xi(1-s)}=\frac{F(s)}{\Xi(s)}=H(s)
\]
Es decir, $e^{c+d(1-s)}=e^{c+ds}$ para todo $s$, lo que fuerza $d(1-2s)=0$ para todo $s$.
Por tanto, $d=0$.

\textbf{Paso 4: Normalización.}
Con $d=0$, tenemos $H(s)\equiv e^c$. La condición de normalización $F(1/2)=\Xi(1/2)$ implica:
\[
e^c = \frac{F(1/2)}{\Xi(1/2)} = 1
\]
Por tanto, $c=0$ y $H(s)\equiv 1$.

\textbf{Conclusión:} $F(s) = \Xi(s)$ idénticamente.
\end{proof}

\begin{lemma}[Control de crecimiento]\label{lem:control-crecimiento}
Si $F$ y $\Xi$ son de orden $\leq 1$, la razón $H$ tiene crecimiento subexponencial en bandas verticales; combinado con la simetría, esto implica $d=0$ incluso sin evaluar en $s=1/2$, siempre que se fije una normalización alternativa (p.ej. el coeficiente principal).
\end{lemma}

\begin{proof}
En una banda vertical $|\Re(s) - \sigma_0| \leq \delta$, las cotas de Phragmén--Lindelöf dan:
\[
|F(s)|, |\Xi(s)| \leq C e^{\epsilon |s|}
\]
para cualquier $\epsilon > 0$. Por tanto:
\[
|H(s)| = \left|\frac{F(s)}{\Xi(s)}\right| \leq C' e^{\epsilon |s|}
\]
Como $H(s) = e^{c+ds}$, esto implica $|d| \leq \epsilon$ para cualquier $\epsilon > 0$, forzando $d = 0$.
\end{proof}

\begin{cor}[Aplicación a $D(s)$]\label{cor:D-equiv-Xi}
La función $D(s)$ construida via el formalismo adélico S-finito satisface todas las hipótesis del Teorema \ref{thm:unicidad-principal}. Por tanto:
\[
D(s) \equiv \Xi(s)
\]
\end{cor}

\begin{proof}
\textbf{Verificación de hipótesis:}
\begin{itemize}
\item \emph{Orden $\leq 1$:} Consecuencia de la construcción adélica y cotas de crecimiento.
\item \emph{Simetría:} Establecida por el Lema A2 (ahora derivado) via Poisson adélico.
\item \emph{Mismos ceros:} Por construcción, $D(s)$ y $\Xi(s)$ tienen idénticos ceros no triviales.
\item \emph{Normalización:} Fijada por la constante adélica global.
\end{itemize}
Aplicación directa del Teorema \ref{thm:unicidad-principal}.
\end{proof}

\subsection*{Implicaciones para la Coronación V5}

El Corolario \ref{cor:D-equiv-Xi} completa el segundo paso crucial de la Coronación V5: habiendo convertido A1-A4 en lemas derivados, ahora establecemos que la función emergente $D(s)$ es \emph{idéntica} a la función xi de Riemann $\Xi(s)$. 

Esta identificación no es circular: $D(s)$ se construye independientemente via el formalismo adélico, y luego se demuestra que debe coincidir con $\Xi(s)$ por razones de unicidad analítica.

El siguiente paso será demostrar que todos los ceros de $D(s)$ (y por tanto de $\Xi(s)$) yacen en la recta crítica $\Re(s) = 1/2$.
