\section{Lema de Unicidad Paley--Wiener con Multiplicidades}

\begin{lemma}[Unicidad con multiplicidades]
Sea $F(s)$ entera de orden $\leq 1$ y tipo finito, con simetría $F(s)=F(1-s)$.
Si $F$ y $\Xi(s)$ tienen la misma medida espectral de ceros (incluyendo multiplicidades),
y $F(1/2)=\Xi(1/2)$, entonces $F \equiv \Xi$.
\end{lemma}

\begin{proof}
1. Por Paley--Wiener, $F$ y $\Xi$ corresponden a transformadas de funciones de soporte compacto.
2. La igualdad de medidas espectrales implica que las transformadas son idénticas en la recta crítica.
3. La normalización $F(1/2)=\Xi(1/2)$ elimina la constante multiplicativa.
4. Concluimos que $F(s)=\Xi(s)$ para todo $s$.
\end{proof}
