\documentclass[12pt]{article}
\usepackage{amsmath, amsthm, amssymb}
\usepackage[utf8]{inputenc}
\usepackage[T1]{fontenc}
\usepackage{geometry}
\geometry{margin=1in}

\newtheorem{lemma}{Lema}

\begin{document}

\title{Lema de Unicidad Paley--Wiener con Multiplicidades}
\author{}
\date{}
\maketitle

\begin{lemma}[Unicidad de Paley--Wiener con multiplicidades]
Sea $F(s)$ entera de orden $\le 1$ y tipo finito, con simetría $F(s)=F(1-s)$.
Si $F$ y $\Xi(s)$ tienen la misma medida espectral de ceros (incluyendo multiplicidades),
y $F(1/2)=\Xi(1/2)$, entonces $F \equiv \Xi$.
\end{lemma}

\begin{proof}
% 1. Aplicar Paley--Wiener: transformadas con soporte compacto.
% 2. Igualdad de medidas espectrales $\Rightarrow$ igualdad de transformadas.
% 3. Normalización en $s=1/2$ fija la constante.
\end{proof}

\end{document}
