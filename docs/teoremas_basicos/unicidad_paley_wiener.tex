\section{Lema de unicidad Paley--Wiener con multiplicidades}

Mostramos que las propiedades analíticas básicas de $D(s)$---orden, simetría y
localización de ceros---determinan la función por completo.  La prueba combina la
factorización de Hadamard con el principio de unicidad de Paley--Wiener--Hamburger.

\begin{lemma}[Unicidad con multiplicidades]\label{lem:unicidad-paley-wiener}
Sea $F$ una función entera de orden a lo sumo $1$ y tipo finito que satisface la
ecuación funcional $F(s)=F(1-s)$.  Supongamos que el divisor de ceros de $F$
coincide con el de la función completada $\Xi(s)$ (incluyendo multiplicidades) y
que $F(1/2)=\Xi(1/2)$.  Entonces $F\equiv \Xi$.
\end{lemma}

\begin{proof}
Por la factorización de Hadamard \cite[Chap.~II]{Tate1967}, el cociente

\[
 H(s)=\frac{F(s)}{\Xi(s)}
\]

es una función entera sin ceros, de orden $0$.  La simetría $F(s)=F(1-s)$ y
$\Xi(s)=\Xi(1-s)$ implica que $H(s)=H(1-s)$, por lo que la función $h(s)=\log H(s)$
es entera de crecimiento a lo sumo lineal en bandas verticales.  El teorema de
Paley--Wiener reforzado de Hamburger
\cite[Thm.~5]{Hamburger1921}
establece que $h$ es la transformada de Fourier de una medida compactamente
soportada.

Por otro lado, la condición $H(1/2)=1$ (derivada de la normalización en
$s=1/2$) obliga a que la medida tenga masa total cero.  Si esa medida no fuera
nula, $h$ crecería linealmente en alguna dirección imaginaria, contradictorio con
la acotación proporcionada por la teoría de crecimiento de orden $\leqslant1$.
En consecuencia, $h$ debe ser constante y, por la normalización, $h\equiv0$.
Esto muestra que $H(s)\equiv1$ y, por tanto, $F(s)=\Xi(s)$ para todo $s$.
\end{proof}

Este resultado cierra la posibilidad de soluciones ``exóticas'' que compartan los
datos espectrales con $\Xi$: cualquier función entera con las propiedades
postuladas es necesariamente igual a la función de Riemann completada.
