\section{Unicidad Paley--Wiener con multiplicidades}

\newtheorem{theoremB}{Theorem}[section]
\newtheorem{lemmaB}[theoremB]{Lemma}

\begin{theoremB}[Unicidad con multiplicidades]
Sea $F(s)$ una función entera de orden $\le 1$ y tipo finito, con simetría $F(1-s)=F(s)$.
Suponga que $F$ y $\Xi(s)$ (la función completada de Riemann) tienen la misma medida
espectral de ceros incluyendo multiplicidades y que $F(1/2)=\Xi(1/2)\neq 0$.
Entonces $F\equiv \Xi$.
\end{theoremB}

\begin{proof}
Por teoría de Hadamard para funciones enteras de orden $\le 1$, $F$ y $\Xi$
admiten productos canónicos
\[
F(s)=e^{a+bs}\prod_\rho E_1\!\left(\frac{s}{\rho}\right),\qquad
\Xi(s)=e^{a'+b's}\prod_\rho E_1\!\left(\frac{s}{\rho}\right),
\]
donde el producto es sobre los mismos ceros (con multiplicidad) por hipótesis,
y $E_1(z)=(1-z)e^{z}$.
Por tanto, la razón $H(s):=\frac{F(s)}{\Xi(s)}$ es entera sin ceros (y sin polos), luego $H(s)=e^{c+ds}$.

La simetría $F(1-s)=F(s)$ y $\Xi(1-s)=\Xi(s)$ implican
$H(1-s)=H(s)$, es decir $e^{c+d(1-s)}=e^{c+ds}$ para todo $s$, lo que fuerza $d=0$.
Así $H$ es constante. La normalización $F(1/2)=\Xi(1/2)$ fija $H\equiv 1$.
\end{proof}

\begin{lemmaB}[Control de crecimiento]
Si $F$ y $\Xi$ son de orden $\le 1$, la razón $H$ tiene crecimiento subexponencial en bandas verticales; combinado con la simetría implica $d=0$ incluso sin evaluar en $s=1/2$, siempre que se fije una normalización alternativa (p.ej. el coeficiente principal).
\end{lemmaB}
