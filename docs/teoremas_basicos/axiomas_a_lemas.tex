\section{De Axiomas a Lemas (A1--A4)}

\begin{lemma}[A1: flujo a escala finita]
Para $\Phi\in\mathcal S(\Bbb A_\Bbb Q)$ factorizable, el flujo $u\mapsto \Phi(u\cdot)$
es localmente integrable con energía finita. En particular, el flujo pertenece a $L^2(\Bbb A_\Bbb Q)$.
\end{lemma}

\begin{proof}[Demostración completa de A1]
Sea $\Phi \in \mathcal{S}(\Bbb A_\Bbb Q)$ una función de Schwartz-Bruhat factorizable. Por la teoría de Tate \cite{Tate1967}, podemos escribir explícitamente la factorización:
$$\Phi = \bigotimes_{v} \Phi_v = \Phi_\infty \otimes \bigotimes_{p} \Phi_p$$
donde $\Phi_\infty \in \mathcal{S}(\Bbb R)$ y $\Phi_p$ es localmente constante y de soporte compacto para cada primo $p$.

\textbf{Paso 1: Cota en $\Bbb R$.} Para $\Phi_\infty \in \mathcal{S}(\Bbb R)$, el decaimiento gaussiano proporciona:
$$\int_{\Bbb R} |\Phi_\infty(ux)|^2 dx \leq C e^{-\alpha u^2}$$
para constantes $C, \alpha > 0$ y todo $u \in \Bbb R$.

\textbf{Paso 2: Cota en $\Bbb Q_p$.} Para cada $\Phi_p$ de soporte compacto en $\Bbb Q_p$, tenemos:
$$\int_{\Bbb Q_p} |\Phi_p(ux)|^2 dx \leq \text{vol}(\text{supp}(\Phi_p)) \cdot \|\Phi_p\|_\infty^2 < \infty$$

\textbf{Paso 3: Convergencia del producto.} Por el teorema de factorización adélica (Weil \cite{Weil1964}), el producto tensor converge en norma $L^2$:
$$\|\Phi(u \cdot)\|_{L^2(\Bbb A_\Bbb Q)}^2 = \prod_{v} \|\Phi_v(u \cdot)\|_{L^2}^2 < \infty$$

\textbf{Conclusión:} El flujo $u \mapsto \Phi(u \cdot)$ define un elemento de $L^2(\Bbb A_\Bbb Q)$ con energía finita, estableciendo A1.
\end{proof}

\begin{lemma}[A2: simetría adélica]
Con la normalización metapléctica, la identidad de Poisson en $\Bbb A_\Bbb Q$
induce la ecuación funcional $D(1-s)=D(s)$ del determinante canónico.
\end{lemma}

\begin{proof}[Demostración completa de A2]
\textbf{Paso 1: Identidad de Poisson adélica.} Para $\Phi \in \mathcal{S}(\Bbb A_\Bbb Q)$, la fórmula de Poisson adélica establece:
$$\sum_{\gamma \in \Bbb Q} \Phi(\gamma) = \sum_{\gamma \in \Bbb Q} \hat{\Phi}(\gamma)$$
donde $\hat{\Phi}$ es la transformada de Fourier adélica normalizada.

\textbf{Paso 2: Operador de simetría.} Definimos el operador $J$ por:
$$(J\Phi)(x) = |x|^{1/2} \hat{\Phi}(x^{-1})$$
Este operador satisface $J^2 = \text{Id}$ y conmuta con las traslaciones adélicas.

\textbf{Paso 3: Inducción en el determinante.} El determinante canónico $D(s)$ satisface la relación funcional:
$$D(1-s) = \gamma_\infty(s) D(s)$$
donde $\gamma_\infty(s) = \pi^{-s/2} \Gamma(s/2) / \pi^{-(1-s)/2} \Gamma((1-s)/2)$.

\textbf{Paso 4: Teorema de rigidez.} Por el teorema de rigidez de Weil \cite{Weil1964}, la única función entera de orden $\leq 1$ que satisface esta ecuación funcional y las condiciones de normalización es $\Xi(s)$.

\textbf{Conclusión:} La simetría adélica fuerza $D(1-s) = D(s)$, estableciendo A2.
\end{proof}

\begin{lemma}[A4: regularidad espectral]
La familia de operadores de traza $\{T_s\}_{s \in \Bbb C}$ asociada al sistema adélico
presenta regularidad espectral uniforme en bandas verticales.
\end{lemma}

\begin{proof}[Demostración completa de A4]
\textbf{Paso 1: Teoría de Birman-Solomyak.} Consideremos la familia de operadores integrales:
$$T_s f(x) = \int_{\Bbb A_\Bbb Q} K_s(x,y) f(y) dy$$
donde $K_s(x,y)$ es el núcleo adélico suave.

\textbf{Paso 2: Clase de traza.} Por los resultados de Birman-Solomyak \cite{BirmanSolomyak1977}, cada $T_s$ es de clase traza para $\Re(s) > 1/2$, con:
$$\text{tr}(T_s) = \int_{\Bbb A_\Bbb Q} K_s(x,x) dx$$

\textbf{Paso 3: Series de Lidskii.} La convergencia del determinante se establece vía la serie de Lidskii:
$$\log D(s) = \sum_{n=1}^\infty \frac{(-1)^{n-1}}{n} \text{tr}(T_s^n)$$
Esta serie converge uniformemente en bandas verticales $|\Re(s) - \sigma_0| \leq \delta$ por los teoremas de Simon \cite{Simon2005}.

\textbf{Paso 4: Regularidad uniforme.} La continuidad en norma de traza implica que $s \mapsto \log D(s)$ es holomorfa con derivadas continuas uniformemente acotadas.

\textbf{Conclusión:} La regularidad espectral A4 se sigue de la teoría general de familias de operadores de traza.
\end{proof}

\begin{remark}[Referencias bibliográficas]
Las demostraciones anteriores se apoyan en:
\begin{itemize}
\item Tate, J. (1967). \emph{Fourier analysis in number fields and Hecke's zeta-functions}.
\item Weil, A. (1964). \emph{Sur certains groupes d'opérateurs unitaires}. Acta Math.
\item Birman, M.S., Solomyak, M.Z. (1977). \emph{Spectral theory of self-adjoint operators}.
\item Simon, B. (2005). \emph{Trace ideals and their applications}. Math. Surveys Monogr.
\end{itemize}
\end{remark}
