\section{De Axiomas a Lemas (A1--A4)}

\begin{lemma}[A1: flujo a escala finita]
Para $\Phi\in\mathcal S(\Bbb A_\Bbb Q)$ factorizable, el flujo $u\mapsto \Phi(u\cdot)$
es localmente integrable con energía finita. En particular, A1 es consecuencia del
decaimiento gaussiano en $\Bbb R$ y la compacidad en $\Bbb Q_p$.
\end{lemma}

\begin{lemma}[A2: simetría por Poisson adélico]
Con la normalización metapléctica, la identidad de Poisson en $\Bbb A_\Bbb Q$
induce $D(1-s)=D(s)$ tras completar con $\gamma_\infty(s)$ (Teorema de rigidez).
\end{lemma}

\begin{lemma}[A4: regularidad espectral]
Sea $K_s$ un núcleo suave adélico que define operadores de traza en una banda vertical.
Por los resultados combinados de Tate (conmutatividad Haar), Weil (identificación de órbitas), 
y Birman--Solomyak (ligaduras para trazas), se tiene que:
\begin{enumerate}
  \item La longitud de órbita $\ell_v = \log q_v$ deriva geométricamente, sin input de $\zeta(s)$.
  \item El operador $K_s$ es Hilbert--Schmidt para $\Re(s) = 1/2$.
  \item La dependencia holomorfa en $s$ garantiza continuidad espectral.
  \item La suma $\sum |\lambda_i| < \infty$ converge, estableciendo regularidad uniforme.
\end{enumerate}
Así, A4 es consecuencia de estos resultados establecidos, haciendo la propuesta incondicional.
\end{lemma}

\begin{proof}
\textbf{Lemma 1 (Tate):} La medida de Haar adélica factoriza como producto de medidas locales.
Para $\Phi \in \mathcal{S}(\mathbb{A}_\mathbb{Q})$ factorizable, $\Phi = \prod_v \Phi_v$, 
la transformada de Fourier conmuta: $\mathcal{F}(\Phi) = \prod_v \mathcal{F}(\Phi_v)$.

\textbf{Lemma 2 (Weil):} Las órbitas cerradas del flujo geodésico corresponden biyectivamente
a clases de conjugación. Sus longitudes son exactamente $\ell_v = \log q_v$.

\textbf{Lemma 3 (Birman--Solomyak):} Los operadores de clase traza con dependencia holomorfa
tienen espectro que varía continuamente. La convergencia $\sum |\lambda_i| < \infty$ garantiza
regularidad espectral uniforme.

Por lo tanto, combinando estos tres lemas, la regularidad espectral A4 está demostrada.
\end{proof}
