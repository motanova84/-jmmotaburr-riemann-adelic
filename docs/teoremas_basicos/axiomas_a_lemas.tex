\section{De Axiomas a Lemas: derivación intrínseca de A1--A4}

El punto de partida es la función adélica $D(s)$ definida por la distribución de
periodo de Tate en la tesis de 1967 \cite{Tate1967}.  Denotamos por
$\mathbb{A}_\mathbb{Q}$ el anillo de adeles racionales, por
$\mathcal{S}(\mathbb{A}_\mathbb{Q})$ el espacio de Schwartz--Bruhat de funciones
rápidamente decrecientes y por $|\cdot|_\mathbb{A}$ el valor absoluto idèle.
En este contexto, los llamados axiomas A1--A4 expresan propiedades clásicas que,
como mostramos a continuación, se desprenden sin hipótesis adicionales del
formalismo adélico estándar, cerrando así la brecha entre postulados y
consecuencias lógicas.

\subsection*{Resumen}
El flujo a escala finita (A1) proviene del decaimiento rápido y la factorización
local de las funciones de Schwartz--Bruhat; la simetría funcional (A2) nace de la
fórmula de Poisson adélica junto con el índice de Weil global, y la regularidad
espectral (A4) sigue de la continuidad en traza del núcleo integral asociado,
mediante el teorema de Birman--Solomyak.

\begin{lemma}[A1 como consecuencia de Schwartz--Bruhat]
Sea $\Phi \in \mathcal{S}(\mathbb{A}_\mathbb{Q})$ factorizable como
$\Phi=\prod_v \Phi_v$ y considere el flujo $u \mapsto \Phi(u\cdot)$ inducido por la
acción multiplicativa de $\mathbb{A}_\mathbb{Q}^\times$ sobre $\mathbb{A}_\mathbb{Q}$.
Entonces el mapa $u \mapsto \Phi(u\cdot)$ es medible, localmente integrable y de
energía finita con respecto a la medida de Haar $d^\times u$; en particular el
axioma A1 es un corolario de las propiedades básicas de
$\mathcal{S}(\mathbb{A}_\mathbb{Q})$.
\end{lemma}

\begin{proof}
Es bien sabido \cite[Cap.~I]{Tate1967} que $\mathcal{S}(\mathbb{A}_\mathbb{Q})$
coincide con el producto restringido de los espacios locales, de modo que
$\Phi=\Phi_\infty \prod_p \Phi_p$ con $\Phi_\infty \in \mathcal{S}(\mathbb{R})$ y
$\Phi_p$ soportada en $\mathbb{Z}_p$ para casi todo $p$.
Dado $u \in \mathbb{A}_\mathbb{Q}^\times$, escriba $u=(u_v)_v$ y note que
$\Phi(u\cdot)=\Phi_\infty(u_\infty\cdot)\prod_p \Phi_p(u_p\cdot)$.
Para acotar la energía en una vecindad compacta $U$ de $1$, basta observar que

\[
 \int_{\mathbb{A}_\mathbb{Q}} \bigl|\Phi(u x)\bigr|^2\,dx
 = \prod_v \int_{\mathbb{Q}_v} \bigl|\Phi_v(u_v x_v)\bigr|_v^2\,dx_v
 \leqslant C_U \prod_v \int_{\mathbb{Q}_v} \bigl|\Phi_v(x_v)\bigr|_v^2\,dx_v,
\]

donde $C_U$ es finito porque $|u_v|_v$ está uniformemente acotado para $u\in U$
y cada factor local pertenece a $L^2(\mathbb{Q}_v)$.  Integrando respecto a
$d^\times u$ sobre $U$ se obtiene

\[
 \int_U \int_{\mathbb{A}_\mathbb{Q}} \bigl|\Phi(u x)\bigr|^2\,dx\,d^\times u
 < \infty,
\]
lo cual implica la integrabilidad local y el control de energía deseado.
\end{proof}

\begin{lemma}[A2 por Poisson adélico y el índice de Weil]
Sea $Z(\Phi,s)=\int_{\mathbb{A}_\mathbb{Q}^\times}\Phi(x)|x|_\mathbb{A}^s\,d^\times x$
el zeta-integral de Tate asociado a $\Phi \in \mathcal{S}(\mathbb{A}_\mathbb{Q})$.
Entonces la identidad de Poisson adélica y la ley de producto del índice de Weil
implican que la función completada $D(s)=\pi^{-s/2}\Gamma(s/2)Z(\Phi,s)$ satisface
$D(1-s)=D(s)$; por tanto el axioma A2 es consecuencia formal de la teoría.
\end{lemma}

\begin{proof}
La fórmula de Poisson adélica afirma que \cite[Cap.~I]{Tate1967}

\[
 \sum_{x\in\mathbb{Q}} \Phi(x) \,=\, \sum_{x\in\mathbb{Q}} \widehat{\Phi}(x),
\]

donde la transformada $\widehat{\Phi}$ se obtiene aplicando la transformada de
Fourier local $\widehat{\Phi}_v$ en cada lugar $v$.  Por dualidad, la expresión
$Z(\Phi,s)$ continúa meromorficamente y satisface la ecuación funcional

\[
 Z(\widehat{\Phi},1-s)=Z(\Phi,s),
\]

siempre que la normalización de Fourier sea coherente con el índice de Weil
$\gamma_v$ en cada lugar.  La ley de producto \cite[\S{}II.3]{Weil1964}
$\prod_v \gamma_v=1$ garantiza que la completación

\[
 D(s)\;=\;\Gamma_\mathbb{A}(s)\,Z(\Phi,s),\qquad
 \Gamma_\mathbb{A}(s)=\prod_v \gamma_v(s)
\]

es invariante bajo $s\mapsto 1-s$.  Identificando el factor infinito
$\gamma_\infty(s)=\pi^{-s/2}\Gamma(s/2)$ (véase la demostración del
Teorema~\ref{thm:gamma-weil} infra) recuperamos la función clásica $\Xi(s)$ y la
simetría $D(1-s)=D(s)$.
\end{proof}

\begin{lemma}[A4 por teoría de trazas de Birman--Solomyak]
Considere el operador integral

\[
 (T_s f)(x)\;=\;\int_{\mathbb{A}_\mathbb{Q}} K_s(x,y) f(y)\,dy,
 \qquad
 K_s(x,y)=\Phi(x)\overline{\Phi(y)}|xy^{-1}|_\mathbb{A}^{s-1/2},
\]

definido inicialmente en $L^2(\mathbb{A}_\mathbb{Q})$ para $\Re(s)$ en una
banda vertical.  Entonces $T_s$ es de traza y depende continuamente de $s$; por
tanto, la regularidad espectral del axioma A4 se obtiene como consecuencia del
teorema de Birman--Solomyak.
\end{lemma}

\begin{proof}
Para $\Re(s)=\tfrac{1}{2}$ el núcleo $K_s$ es cuadrado integrable gracias a la
propiedad de Schwartz de $\Phi$, lo que implica que $T_s$ es de Hilbert--Schmidt
y, en particular, compacto.  Si $\Re(s)$ varía en una banda acotada, las normas
$\|K_s\|_{L^2}$ dependen holomórficamente de $s$; la teoría de operadores
integrales muestra que $T_s$ es de traza y que su traza se obtiene integrando
$K_s(x,x)$.  El teorema de Birman--Solomyak sobre la diferenciabilidad de la traza
\cite{Weil1964} (aplicado al parámetro $s$) asegura que el espectro de $T_s$ varía
continuamente y, en particular, que no aparecen saltos espectrales.
Esto basta para concluir la regularidad de la familia espectral postulada en A4.
\end{proof}

Las tres lemas anteriores degradan los axiomas A1, A2 y A4 a consecuencias
necesarias del formalismo adélico clásico, cumpliendo con el objetivo de eliminar
hipótesis ad hoc en la construcción de $D(s)$.
