\section{De Axiomas a Lemas (A1--A4)}

\begin{lemma}[A1: flujo a escala finita]
Para $\Phi\in\mathcal S(\Bbb A_\Bbb Q)$ factorizable, el flujo $u\mapsto \Phi(u\cdot)$
es localmente integrable con energía finita. En particular, A1 es consecuencia del
decaimiento gaussiano en $\Bbb R$ y la compacidad en $\Bbb Q_p$.
\end{lemma}

\begin{lemma}[A2: simetría por Poisson adélico]
Con la normalización metapléctica, la identidad de Poisson en $\Bbb A_\Bbb Q$
induce $D(1-s)=D(s)$ tras completar con $\gamma_\infty(s)$ (Teorema de rigidez).
\end{lemma}

\begin{lemma}[A4: regularidad espectral y longitudes de órbitas]
Sea $K_s$ un núcleo suave adélico que define operadores de traza en una banda vertical.
La continuidad en traza y el resultado de Birman--Solomyak implican regularidad
espectral uniforme en $s$, estableciendo A4.

Además, las longitudes de órbitas primitivas $\ell_v$ están determinadas únicamente por
la estructura adélica local: $\ell_v = \log q_v$ para todo lugar finito $v$.

Esta identidad se deriva de tres resultados fundamentales:
\begin{enumerate}
\item \textbf{Conmutatividad e invarianza Haar (Tate 1967):} El flujo de escala $S_u$ y los
operadores locales $U_v$ conmutan por invarianza de la medida de Haar adélica.
En coordenadas logarítmicas, $U_v$ actúa como traslación discreta $\tau \mapsto \tau + \log q_v$.

\item \textbf{Identificación de órbitas cerradas (Weil 1964):} El operador $U_v$ genera un
subgrupo discreto de traslaciones con longitud primitiva determinada por el normador local:
$\ell_v = \log q_v$ donde $|\pi_v|_v = q_v^{-1}$.

\item \textbf{Estabilidad de traza (Birman--Solomyak 1977):} Los operadores suavizados
$f(X) K_\delta f(X)$ son de clase traza $\mathcal{S}_1$. La fórmula de traza preserva
la estructura discreta de órbitas, y la identidad $\ell_v = \log q_v$ es estable en
el límite $\delta \to 0^+$.
\end{enumerate}

Por tanto, A4 no es un axioma sino un lema probado dentro del formalismo adélico estándar.
\end{lemma}
