\section{De Axiomas a Lemas (A1--A4)}\label{sec:axiomas-lemas}

\subsection*{Transformación conceptual: de supuestos a consecuencias}

En la versión original del marco teórico, los axiomas A1, A2 y A4 eran postulados fundamentales del sistema S-finito. La Coronación V5 representa el paso crucial donde estos axiomas se convierten en \emph{lemas derivados} del formalismo adélico, eliminando así cualquier circularidad en la construcción.

\begin{lemma}[A1: flujo de escala finita - \emph{derivado}]\label{lem:A1-derivado}
Para $\Phi\in\mathcal{S}(\mathbb{A}_\mathbb{Q})$ factorizable, el flujo $u\mapsto \Phi(u\cdot)$
es localmente integrable con energía finita.
\end{lemma}

\begin{proof}
\textbf{Fundamento:} Schwartz--Bruhat en $\mathbb{A}_\mathbb{Q}$.

\textbf{Componente arquimediana:} Para $\Phi_\infty \in \mathcal{S}(\mathbb{R})$, el decaimiento gaussiano 
\[
|\Phi_\infty(x)| \leq C e^{-\alpha x^2}
\]
implica que $\int_\mathbb{R} |u \Phi_\infty(ux)|^2 |u|^{s-1} d^\times u < \infty$ para $\Re(s) > 0$.

\textbf{Componentes finitas:} Para cada $p$, $\Phi_p$ tiene soporte compacto en $\mathbb{Q}_p$ (típicamente $\mathbf{1}_{\mathbb{Z}_p}$), por lo que la integral local es finita automáticamente.

\textbf{Producto adélico:} La factorización $\Phi = \prod_v \Phi_v$ con casi todos los factores triviales garantiza convergencia del producto euleriano, estableciendo que el flujo global tiene energía finita.

$\therefore$ A1 es \emph{consecuencia} del marco Schwartz--Bruhat, no un axioma independiente.
\end{proof}

\begin{lemma}[A2: simetría funcional - \emph{derivada}]\label{lem:A2-derivado}
La simetría $D(1-s) = D(s)$ se deriva de la identidad de Poisson adélica combinada con la normalización metapléctica del índice de Weil.
\end{lemma}

\begin{proof}
\textbf{Identidad de Poisson global:} Para $\Phi \in \mathcal{S}(\mathbb{A}_\mathbb{Q})$,
\[
\sum_{x \in \mathbb{Q}} \Phi(x) = \sum_{x \in \mathbb{Q}} \widehat{\Phi}(x),
\]
donde $\widehat{\Phi}$ es la transformada de Fourier adélica.

\textbf{Índice de Weil:} La normalización metapléctica fija de manera única las constantes locales $\gamma_v(s)$ para que
\[
\prod_v \gamma_v(s) = 1.
\]

\textbf{Simetría derivada:} Al aplicar la transformada de Fourier adélica a la función zeta parcial construida via $\Phi$, la identidad de Poisson induce automáticamente la relación
\[
D(s) = \gamma(s) D(1-s),
\]
donde $\gamma(s) = \prod_v \gamma_v(s) = 1$, estableciendo $D(1-s) = D(s)$.

$\therefore$ A2 es \emph{consecuencia} de la teoría adélica de Poisson-Weil, no un postulado.
\end{proof}

\begin{lemma}[A4: regularidad espectral - \emph{derivada}]\label{lem:A4-derivado}
La regularidad espectral en $s$ se deriva de la teoría de operadores de traza de Birman--Solomyak aplicada al núcleo integral adélico.
\end{lemma}

\begin{proof}
\textbf{Núcleo adélico:} Sea $K_s(x,y)$ el núcleo integral que define el operador 
\[
(T_s f)(x) = \int_{\mathbb{A}_\mathbb{Q}} K_s(x,y) f(y) d\mu(y).
\]

\textbf{Clase de traza:} Por construcción adélica, $K_s$ satisface
\[
\int_{\mathbb{A}_\mathbb{Q}} |K_s(x,x)| d\mu(x) < \infty
\]
uniformemente en bandas verticales $|\Re(s) - \sigma_0| \leq \delta$.

\textbf{Birman--Solomyak:} La continuidad de la traza como función de parámetros complejos implica que el espectro de $T_s$ varía continuamente con $s$, estableciendo regularidad espectral.

\textbf{Uniformidad:} La factorización adélica y las cotas locales uniformes garantizan que la regularidad se mantiene globalmente.

$\therefore$ A4 es \emph{consecuencia} de la teoría general de operadores de traza, no una condición ad hoc.
\end{proof}

\begin{theorem}[Conversión completa de axiomas]\label{thm:axiomas-a-lemas}
Los axiomas fundamentales A1, A2, A4 del sistema S-finito adélico son todos derivables como lemas dentro del marco teórico, eliminando cualquier dependencia de postulados externos.
\end{theorem}

\begin{proof}
Combinación directa de los Lemas \ref{lem:A1-derivado}, \ref{lem:A2-derivado} y \ref{lem:A4-derivado}.
\end{proof}

\subsection*{Impacto en la construcción de $D(s)$}

Con A1-A4 establecidos como lemas derivados, la función $D(s)$ emerge naturalmente como una función entera de orden $\leq 1$ con simetría funcional, preparando el terreno para la identificación única con $\Xi(s)$ en la siguiente sección.
