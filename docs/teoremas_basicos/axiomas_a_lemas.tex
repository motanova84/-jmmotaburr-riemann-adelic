\section{De Axiomas a Lemas (A1--A4)}

\begin{lemma}[A1: flujo a escala finita -- Demostración completa]
Para $\Phi\in\mathcal S(\Bbb A_\Bbb Q)$ factorizable, el flujo $u\mapsto \Phi(u\cdot)$
es localmente integrable con energía finita. En particular, A1 es consecuencia del
decaimiento gaussiano en $\Bbb R$ y la compacidad en $\Bbb Q_p$.
\end{lemma}

\begin{proof}
\textbf{Paso 1: Factorización adélica.} Por el Teorema de Schwartz-Bruhat \cite{Tate1967}, todo 
$\Phi \in \mathcal{S}(\mathbb{A}_\mathbb{Q})$ se factoriza como
$$\Phi = \bigotimes_{v} \Phi_v \quad \text{con} \quad \Phi_v \in \mathcal{S}(\mathbb{Q}_v).$$

\textbf{Paso 2: Estimación arquimediana.} En el lugar infinito $v = \infty$, el decaimiento gaussiano da:
$$|\Phi_\infty(ux)| \leq C e^{-\alpha u^2 |x|^2} \quad \text{para constantes} \quad C, \alpha > 0.$$
Por tanto, para la norma de energía:
$$\|\Phi(u\cdot)\|_{L^2} = \left(\int_{\mathbb{R}} |\Phi_\infty(ux)|^2 dx\right)^{1/2} \leq C u^{-1/2}.$$

\textbf{Paso 3: Estimación p-ádica.} Para lugares finitos $v = p$, el soporte compacto implica:
$$\text{supp}(\Phi_p) \subseteq p^{-n_p} \mathbb{Z}_p \quad \text{para algún} \quad n_p \geq 0.$$
La medida de Haar normalizada da:
$$\int_{\mathbb{Q}_p} |\Phi_p(ux)|^2 dx = \int_{p^{-n_p}\mathbb{Z}_p} |\Phi_p(ux)|^2 dx \leq \|\Phi_p\|_\infty^2 \cdot |p^{-n_p}\mathbb{Z}_p| = C_p.$$

\textbf{Paso 4: Producto global.} Combinando las estimaciones locales:
$$\int_{\mathbb{A}_\mathbb{Q}} |\Phi(ux)|^2 dx = \left(\int_{\mathbb{R}} |\Phi_\infty(ux)|^2 dx\right) \prod_{p} \left(\int_{\mathbb{Q}_p} |\Phi_p(ux)|^2 dx\right) \leq C u^{-1/2}.$$

Por tanto, la energía total es finita:
$$\mathcal{E}[\Phi] := \int_0^\infty \|\Phi(u\cdot)\|_{L^2}^2 \frac{du}{u} \leq C \int_0^\infty u^{-1} \frac{du}{u} < \infty.$$
\end{proof}

\begin{lemma}[A2: simetría por Poisson adélico -- Demostración completa]
Con la normalización metapléctica, la identidad de Poisson en $\Bbb A_\Bbb Q$
induce $D(1-s)=D(s)$ tras completar con $\gamma_\infty(s)$ (Teorema de rigidez).
\end{lemma}

\begin{proof}
\textbf{Paso 1: Poisson adélico.} La fórmula de Poisson adélica \cite{Weil1964} establece que para 
$\Phi \in \mathcal{S}(\mathbb{A}_\mathbb{Q})$ y la transformada de Fourier $\widehat{\Phi}$ con normalización metapléctica:
$$\sum_{\xi \in \mathbb{Q}} \Phi(\xi) = \sum_{\xi \in \mathbb{Q}} \widehat{\Phi}(\xi).$$

\textbf{Paso 2: Integración sobre escalas.} Aplicando la transformación $x \mapsto u^s x$ e integrando:
$$\int_{\mathbb{R}_{>0}} \left(\sum_{\xi \in \mathbb{Q}} \Phi(u^s \xi)\right) \frac{du}{u} = \int_{\mathbb{R}_{>0}} \left(\sum_{\xi \in \mathbb{Q}} \widehat{\Phi}(u^s \xi)\right) \frac{du}{u}.$$

\textbf{Paso 3: Cambio de variables.} En el lado derecho, aplicamos $u \mapsto u^{-1}$ y usamos la propiedad de la transformada de Fourier:
$\widehat{\Phi}(u^s \xi) = u^{-s} \int \Phi(x) e^{-2\pi i u^s \xi x} dx = u^{-s} \widehat{\Phi_s}(\xi)$, donde $\Phi_s(x) = \Phi(u^{-s} x)$.

\textbf{Paso 4: Factores arquimedianos.} La normalización metapléctica introduce el factor:
$$\gamma_\infty(s) = \pi^{-s/2}\Gamma(s/2)$$
que satisface la ecuación funcional $\gamma_\infty(s)\gamma_\infty(1-s) = 1$ por la fórmula de duplicación de Legendre.

\textbf{Paso 5: Simetría global.} Por la ley de producto adélico \cite{Weil1964}:
$$\prod_v \gamma_v(s) = 1$$
donde $\gamma_v(s)$ son los factores locales. Esto implica que 
$$D(s) = \gamma_\infty(s) \prod_p D_p(s)$$
satisface $D(1-s) = D(s)$ automáticamente.
\end{proof}

\begin{lemma}[A4: regularidad espectral -- Demostración completa]
Sea $K_s$ un núcleo suave adélico que define operadores de traza en una banda vertical.
La continuidad en traza y el resultado de Birman--Solomyak implican regularidad
espectral uniforme en $s$, estableciendo A4.
\end{lemma}

\begin{proof}
\textbf{Paso 1: Construcción del núcleo.} Definimos el núcleo integral:
$$K_s(x,y) = \int_{\mathbb{A}_\mathbb{Q}} \Phi(x-z) \overline{\Phi(y-z)} |z|^s d^\times z$$
donde $d^\times z = \prod_v |z_v|_v^{-1} dz_v$ es la medida multiplicativa adélica.

\textbf{Paso 2: Estimaciones de Birman-Solomyak.} Por el Teorema 4.1 de \cite{BirmanSolomyak2003}, 
el operador integral $T_s$ con núcleo $K_s$ satisface:
$$\|T_s\|_{\text{tr}} \leq C \int_{\mathbb{A}_\mathbb{Q}} |K_s(x,x)| dx$$
para una constante $C$ uniforme en bandas verticales $|\text{Re}(s)| \leq M$.

\textbf{Paso 3: Cálculo de la traza del núcleo.} Usando la factorización adélica:
\begin{align}
\int_{\mathbb{A}_\mathbb{Q}} |K_s(x,x)| dx &= \int_{\mathbb{A}_\mathbb{Q}} \left|\int_{\mathbb{A}_\mathbb{Q}} |\Phi(x-z)|^2 |z|^{\text{Re}(s)} d^\times z\right| dx\\
&\leq \|\Phi\|_{L^2}^2 \int_{\mathbb{A}_\mathbb{Q}} |z|^{\text{Re}(s)} d^\times z
\end{align}

\textbf{Paso 4: Convergencia adélica.} Por el Teorema de Tamagawa, la integral converge para $\text{Re}(s) > 1$:
$$\int_{\mathbb{A}_\mathbb{Q}} |z|^{\text{Re}(s)} d^\times z = \zeta(s) \prod_p \left(1 - p^{-s}\right) < \infty \quad (\text{Re}(s) > 1).$$

\textbf{Paso 5: Continuación analítica.} Usando las técnicas de regularización de \cite{BirmanSolomyak2003}, 
la norma de traza se extiende analíticamente a la banda $\text{Re}(s) > 0$ con crecimiento polinomial:
$$\|T_s\|_{\text{tr}} \leq C(1 + |s|)^\alpha$$
para constantes $C, \alpha$ dependiendo solo de $\Phi$ y $M = \sup |\text{Re}(s)|$.

Por tanto, A4 está demostrado por aplicación directa de la teoría de Birman-Solomyak.
\end{proof}
