\section{De Axiomas a Lemas (A1--A4)}

\begin{lemma}[A1: flujo a escala finita -- Tate-Weil Theory]\label{lem:A1}
Para $\Phi \in \mathcal{S}(\mathbb{A}_\mathbb{Q})$ factorizable, el flujo $u \mapsto \Phi(u\cdot)$
es localmente integrable con energía finita. En particular, A1 sigue del decaimiento gaussiano en 
$\mathbb{R}$ y la compacidad en $\mathbb{Q}_p$.
\end{lemma}

\begin{proof}
Escribimos $\Phi = \otimes_v \Phi_v$ donde el producto corre sobre todos los lugares $v$ de $\mathbb{Q}$. 

\textbf{Caso arquimediano ($v=\infty$):} Para el lugar infinito, si $\Phi_\infty(x) = e^{-\pi x^2}$ (gaussiana estándar), entonces
\[
\int_{\mathbb{R}} |\Phi_\infty(ux)|^2 |x|^s \, dx < \infty \quad \text{para } \Re(s) > -1/2.
\]
Más generalmente, para $\Phi_\infty \in \mathcal{S}(\mathbb{R})$ con decaimiento gaussiano, tenemos la estimación
\[
|\Phi_\infty(x)| \leq C e^{-\alpha x^2} \quad \text{para algunas constantes } C, \alpha > 0.
\]
Esto implica que la integral $\int_{\mathbb{R}} |\Phi_\infty(ux)|^{1+\epsilon} |x|^s \, dx$ converge para $\Re(s) > -1/2 - \epsilon$.

\textbf{Caso no-arquimediano ($v$ finito):} Para cada primo $p$, $\Phi_p$ tiene soporte compacto en $\mathbb{Z}_p$, de modo que
\[
\int_{\mathbb{Q}_p} |\Phi_p(ux)| \, d^\ast x < \infty \quad \forall u \in \mathbb{Q}_p^\ast,
\]
donde $d^\ast x$ es la medida de Haar normalizada con $\text{vol}(\mathbb{Z}_p) = 1$.

\textbf{Convergencia del producto adélico:} Por el teorema de convergencia de Tate \cite{Tate1967}, el producto
\[
\zeta(s, \Phi) = \prod_v \zeta_v(s, \Phi_v) = \zeta_\infty(s, \Phi_\infty) \prod_{p} \zeta_p(s, \Phi_p)
\]
converge absolutamente en $\Re(s) > 1/2$, donde cada factor local está definido por
\[
\zeta_v(s, \Phi_v) = \int_{\mathbb{Q}_v^\ast} \Phi_v(x) |x|_v^s \, d^\ast x.
\]

\textbf{Integrabilidad del flujo:} La integrabilidad local del flujo $u \mapsto \Phi(u\cdot)$ se sigue de la factorización
\[
\int_{\mathbb{A}_\mathbb{Q}^\ast} |\Phi(ux)| |x|^s \, d^\ast x = \prod_v \int_{\mathbb{Q}_v^\ast} |\Phi_v(ux_v)| |x_v|_v^s \, d^\ast x_v,
\]
y cada factor converge por los argumentos anteriores.
\end{proof}

\begin{lemma}[A2: simetría Poisson adélica -- Weil Theory]\label{lem:A2}
Con la normalización metapléctica, la identidad de Poisson en $\mathbb{A}_\mathbb{Q}$
induce $D(1-s) = D(s)$ tras completar con $\gamma_\infty(s)$ (Teorema de rigidez).
\end{lemma}

\begin{proof}
Usamos la fórmula de Poisson de Weil \cite{Weil1964} en el contexto adélico. Sea $\Phi \in \mathcal{S}(\mathbb{A}_\mathbb{Q})$ factorizable.

\textbf{Transformada de Fourier adélica:} La transformada de Fourier adélica $\hat{\Phi}$ se define como
\[
\hat{\Phi}(y) = \int_{\mathbb{A}_\mathbb{Q}} \Phi(x) \psi(xy) \, dx
\]
donde $\psi$ es el carácter aditivo estándar de $\mathbb{A}_\mathbb{Q}/\mathbb{Q}$.

\textbf{Fórmula de Poisson adélica:} Para cualquier $t \in \mathbb{A}_\mathbb{Q}^\ast$, tenemos
\[
\sum_{\gamma \in \mathbb{Q}} \Phi(t\gamma) = |t|_{\mathbb{A}}^{-1} \sum_{\gamma \in \mathbb{Q}} \hat{\Phi}\left(\frac{\gamma}{t}\right)
\]

\textbf{Factor arquimediano:} El factor arquimediano está dado por
\[
\gamma_\infty(s) = \pi^{-s/2} \Gamma\left(\frac{s}{2}\right)
\]
que satisface la ecuación funcional $\gamma_\infty(1-s) = \gamma_\infty(s)$.

\textbf{Operador de inversión $J$:} Definimos el operador $J : \varphi(x) \mapsto \varphi(-x)$. Este operador conmuta con la transformada de Fourier en el sentido que
\[
\widehat{J\varphi}(y) = J\hat{\varphi}(y)
\]
para funciones adecuadas $\varphi$.

\textbf{Simetría espectral:} La combinación de la simetría de Fourier adélico con el factor arquimediano produce
\[
D(1-s) = \gamma_\infty(1-s) \prod_{p} L_p(1-s, \Phi_p) = \gamma_\infty(s) \prod_{p} L_p(s, \hat{\Phi}_p) = D(s)
\]
donde usamos que $\gamma_\infty(1-s) = \gamma_\infty(s)$ y la dualidad de Fourier local $L_p(1-s, \Phi_p) = L_p(s, \hat{\Phi}_p)$.

La acción del operador $J$ preserva la estructura espectral y conlleva la simetría funcional deseada.
\end{proof}

\begin{lemma}[A4: regularidad espectral -- Birman-Solomyak-Simon Theory]\label{lem:A4}
Sea $K_s$ un núcleo suave adélico que define operadores de traza en una banda vertical.
La continuidad en traza y el resultado de Birman--Solomyak implican regularidad
espectral uniforme en $s$, estableciendo A4.
\end{lemma}

\begin{proof}
Aplicamos la teoría de operadores de traza de Birman–Solomyak \cite{birman2003} y Simon \cite{simon2005}.

\textbf{Operadores de traza en bandas verticales:} Sea $B_\delta(s)$ el operador integral definido por el núcleo $K_s(x,y)$ en la banda vertical $|\Re(s) - s_0| < \delta$. Por hipótesis, $K_s(x,y)$ es suave y satisface estimaciones uniformes.

\textbf{Continuidad en norma de traza:} Para $s_1, s_2$ en la banda vertical, tenemos
\[
\|B_\delta(s_1) - B_\delta(s_2)\|_{\mathcal{I}_1} \leq C|s_1 - s_2|^\alpha
\]
para algunas constantes $C > 0$ y $\alpha > 0$, donde $\mathcal{I}_1$ denota los operadores de clase traza.

\textbf{Holomorfía en norma de traza:} Por el teorema de Birman–Solomyak, la función $s \mapsto B_\delta(s)$ es holomorfa como función valuada en $\mathcal{I}_1$. Esto significa que
\[
\frac{d}{ds} B_\delta(s) = \lim_{h \to 0} \frac{B_\delta(s+h) - B_\delta(s)}{h}
\]
existe en norma $\|\cdot\|_{\mathcal{I}_1}$.

\textbf{Fórmula de Lidskii:} Para el determinante regularizado, usamos la fórmula de Lidskii:
\[
\log D(s) = \text{Tr}(\log(I + B_\delta(s))) = \sum_{n=1}^\infty \lambda_n(s)
\]
donde $\{\lambda_n(s)\}$ son los valores propios de $B_\delta(s)$, ordenados por magnitud decreciente.

\textbf{Convergencia uniforme:} La convergencia uniforme de la serie $\sum_{n=1}^\infty \lambda_n(s)$ en bandas verticales compactas se sigue de:
\begin{enumerate}
\item La continuidad de $s \mapsto B_\delta(s)$ en norma de traza,
\item La estimación $|\lambda_n(s)| \leq C n^{-\beta}$ para algún $\beta > 1$,
\item El principio de máximo para funciones holomorfas.
\end{enumerate}

\textbf{Regularidad espectral uniforme:} Combinando los argumentos anteriores, concluimos que $D(s)$ es holomorfo con crecimiento controlado en bandas verticales, lo que establece la regularidad espectral requerida en A4.

La uniformidad en $s$ proviene de la uniformidad de las constantes en las estimaciones de traza sobre conjuntos compactos en el plano complejo.
\end{proof}
