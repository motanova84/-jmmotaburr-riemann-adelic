\section{De Axiomas a Lemas (A1--A4): Teoremas de Base Incondicional}

\textit{Esta sección establece la base teórica para la transición de V4.1 condicional a V5.1 incondicional mediante la demostración de que los axiomas S-finitos son lemas derivables.}

\begin{lemma}[A1: Flujo a escala finita]
\label{lemma:basic-a1}
Para $\Phi\in\mathcal S(\Bbb A_\Bbb Q)$ factorizable, el flujo $u\mapsto \Phi(u\cdot)$ es localmente integrable con energía finita. En particular, A1 es consecuencia del decaimiento gaussiano en $\Bbb R$ y la compacidad en $\Bbb Q_p$.

\textbf{Fundamento teórico:} Teoría de Schwartz-Bruhat para grupos adélicos, establecida sin referencia a funciones L.
\end{lemma}

\begin{lemma}[A2: Simetría por Poisson adélico]  
\label{lemma:basic-a2}
Con la normalización metapléctica, la identidad de Poisson en $\Bbb A_\Bbb Q$ induce $D(1-s)=D(s)$ tras completar con $\gamma_\infty(s)$ (Teorema de rigidez).

\textbf{Fundamento teórico:} Dualidad de Tate-Weil para grupos adélicos, independiente de la hipótesis de Riemann.
\end{lemma}

\begin{lemma}[A4: Regularidad espectral]
\label{lemma:basic-a4}
Sea $K_s$ un núcleo suave adélico que define operadores de traza en una banda vertical. La continuidad en traza y el resultado de Birman--Solomyak implican regularidad espectral uniforme en $s$, estableciendo A4.

\textbf{Fundamento teórico:} Teoría de operadores de clase traza, resultado estándar del análisis funcional.
\end{lemma}

\begin{theorem}[Incondicionalidad del Marco V5.1]
Los lemas A1, A2, A4 proporcionan una base incondicional completa para la construcción adélica de $D(s) \equiv \Xi(s)$, eliminando la dependencia de axiomas no verificados presente en V4.1.
\end{theorem}
