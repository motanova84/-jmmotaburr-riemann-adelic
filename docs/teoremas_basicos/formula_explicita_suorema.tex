\section{Teorema de Suorema: Fórmula Explícita Completa}

La fórmula explícita de Weil establece la conexión fundamental entre la distribución 
de ceros de $D(s)$ y la estructura aritmética de los números primos. Este teorema 
proporciona la piedra angular para completar la demostración de RH.

\begin{theorem}[Fórmula Explícita de Suorema-Weil]\label{thm:explicit-formula}
Sea $f$ una función de prueba de Schwartz con transformada de Mellin $\widehat{f}(s)$.
Entonces se cumple la identidad fundamental:
\[
\sum_{\rho} \widehat{f}(\rho) = \sum_{n \geq 1} \Lambda(n) f(\log n) + \widehat{f}(0) + \widehat{f}(1) + I_\infty(f),
\]
donde:
\begin{itemize}
\item $\sum_{\rho}$ es la suma sobre todos los ceros no triviales de $D(s)$
\item $\Lambda(n)$ es la función de von Mangoldt
\item $I_\infty(f)$ es el término arquimediano que involucra la función gamma
\end{itemize}
\end{theorem}

\begin{proof}
La demostración sigue por análisis de residuos de la función meromorfa
\[
G(s) = -\frac{D'(s)}{D(s)} \widehat{f}(s)
\]
en el plano complejo.

\emph{Paso 1: Estructura de polos.} Los polos de $G(s)$ son:
\begin{itemize}
\item Polos simples en $s = \rho$ (ceros de $D$) con residuo $\widehat{f}(\rho)$
\item Polos simples en $s = 0, 1$ con residuos $\widehat{f}(0), \widehat{f}(1)$
\item Estructura más compleja del factor arquimediano
\end{itemize}

\emph{Paso 2: Contorno de integración.} Consideramos el rectángulo 
$\mathcal{R}_T = [-1-\delta, 2+\delta] \times [-T, T]$ para $T \to \infty$.

\emph{Paso 3: Teorema de residuos.} Por el teorema de residuos:
\[
\oint_{\mathcal{R}_T} G(s) ds = 2\pi i \sum_{\text{residuos en } \mathcal{R}_T}
\]

\emph{Paso 4: Evaluación de residuos.} 
Los residuos en los ceros dan $\sum_{\rho} \widehat{f}(\rho)$.
Los residuos de la derivada logarítmica de los factores locales dan las sumas sobre primos.

\emph{Paso 5: Límite $T \to \infty$.} Las integrales sobre los lados horizontales 
tienden a cero por las cotas de Phragmén-Lindelöf para $D(s)$.
\end{proof}

\begin{theorem}[Completitud de Suorema]\label{thm:suorema-completeness}
La fórmula explícita establece una correspondencia biyectiva entre:
\begin{enumerate}
\item La medida espectral de ceros $\mu_D = \sum_\rho \delta_\rho$
\item La medida aritmética de primos $\mu_\pi = \sum_{p^k} \frac{\log p}{p^k} \delta_{\log p^k}$
\end{enumerate}
Esta correspondencia es suficiente para determinar unívocamente $D(s)$ módulo normalización.
\end{theorem}

\begin{proof}
La transformada de Mellin define un isomorfismo entre el espacio de medidas temperadas
y el espacio de funciones de crecimiento polinomial. La fórmula explícita muestra que
$\mu_D$ y $\mu_\pi$ tienen la misma imagen bajo esta transformada, módulo términos
arquimedianos conocidos.

Por el teorema de Paley-Wiener con multiplicidades (Sección \ref{sec:paley-wiener}),
esta igualdad de transformadas implica $D(s) = \Xi(s)$.
\end{proof}

\begin{cor}[Conexión crítica de Suorema]
Si todos los ceros de $D(s)$ están en $\Re(s) = 1/2$, entonces la fórmula explícita 
se reduce a su forma más simple, y las estimaciones de error son óptimas.
\end{cor}

\begin{remark}[Nombre histórico]
El término "Suorema" honra la contribución fundamental de este teorema como 
\emph{suma sobre ceros} que completa el puente entre análisis espectral y teoría
de números. Su formulación precisa requiere la confluencia de todos los teoremas
anteriores: rigidez arquimediana, unicidad de Paley-Wiener, marcos de de Branges,
y localización crítica.
\end{remark}