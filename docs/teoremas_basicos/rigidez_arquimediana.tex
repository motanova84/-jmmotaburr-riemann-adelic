\section{Teorema de Rigidez Arquimediana}

\newtheorem{theoremA}{Theorem}[section]
\newtheorem{lemmaA}[theoremA]{Lemma}
\newtheorem{propA}[theoremA]{Proposition}

\begin{theoremA}[Rigidez arquimediana]
Sea $D(s)$ una función entera de orden $\le 1$, con simetría $D(1-s)=D(s)$, tal que
sus factores locales satisfacen la ley de producto global del índice de Weil. Entonces
el factor local en $v=\infty$ (lugar arquimediano) está fijado de forma única como
\[
\gamma_\infty(s)=\pi^{-s/2}\Gamma\!\left(\frac{s}{2}\right).
\]
\end{theoremA}

\begin{proof}
Trabajamos en el marco de Schwartz--Bruhat $\mathcal{S}(\Bbb A_\Bbb Q)$ y su transformada de Fourier adélica $\widehat{\Phi}$ con normalización metapléctica (índice de Weil).
Sea $\Phi=\prod_v \Phi_v$ factorizable localmente con $\Phi_\infty(x)=e^{-\pi x^2}$ y $\Phi_p=\mathbf 1_{\Bbb Z_p}$.

\emph{(1) Identidad de Poisson global.} La fórmula de Poisson adélica (Weil) da
\[
\sum_{x\in\Bbb Q}\Phi(x)=\sum_{x\in\Bbb Q}\widehat\Phi(x).
\]
Al descomponer localmente, aparecen factores $\gamma_v(s)$ en la ecuación funcional local de las integrales de Tate. Para cada lugar $v$,
\[
Z_v(\Phi_v,s)=\gamma_v(s)\,Z_v(\widehat{\Phi_v},1-s),
\]
y el \emph{producto global} de índices satisface $\prod_v \gamma_v(s)=1$ (reciprocidad).

\emph{(2) Fijación en los lugares finitos.} Con la elección estándar $\Phi_p=\mathbf 1_{\Bbb Z_p}$ se obtiene la normalización canónica en $p$ (véase Tate). Así, $\gamma_p(s)$ queda determinado y coincide con el factor local usual de Riemann.

\emph{(3) Caso arquimediano y simetría.} La condición global $\prod_v \gamma_v(s)=1$ y la simetría $D(1-s)=D(s)$ fuerzan que el único candidato para compensar los factores finitos sea el factor
\[
\gamma_\infty(s)=\pi^{-s/2}\Gamma\!\left(\frac{s}{2}\right),
\]
que es precisamente el que se obtiene con la gaussiana $e^{-\pi x^2}$ y la normalización metapléctica estándar (Weil). Cualquier otra normalización en $v=\infty$ rompería bien la ley de producto (no se obtiene $1$) o la simetría $s\mapsto 1-s$.
\end{proof}

\begin{propA}[Rigidez por método de fase estacionaria]
Sea $I_\infty(s)$ el término arquimediano en la fórmula explícita asociado a un kernel gaussiano.
La evaluación por fase estacionaria del integral oscilatorio en el lugar real produce
el mismo factor $\pi^{-s/2}\Gamma(s/2)$. Por tanto, la constante global se fija de modo independiente de la ruta metapléctica.
\end{propA}

\begin{proof}[Bosquejo]
Se reescribe $I_\infty(s)$ como integral de Mellin--Fourier con fase cuadrática; al
aplicar estacionaria y cambio de variables estándar (Gaussian integral), el término principal y la constante coinciden con la de la transformada de Fourier normalizada de la gaussiana, lo cual reproduce $\pi^{-s/2}\Gamma(s/2)$.
\end{proof}
