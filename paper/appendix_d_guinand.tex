\section*{Appendix D — Weil–Guinand Explicit Formula: Complete Derivation}

This appendix provides a complete, step-by-step derivation of the Weil–Guinand explicit formula adapted to the canonical determinant $D(s)$, starting from the Hadamard product and concluding with the adelic trace formula.

\subsection*{D.1 Starting Point: Hadamard Product}

From the Hadamard factorization theorem (see main text Section 4), the canonical determinant $D(s)$ admits the representation
\[
D(s) = e^{As + B} s^{m_0} (1-s)^{m_1} \prod_{\rho} E_1\left(\frac{s}{\rho}\right),
\]
where $E_1(z) = (1-z)e^z$ is the Weierstrass elementary factor of order 1, and the product runs over all non-trivial zeros $\rho$ of $D(s)$.

By the functional equation $D(1-s) = D(s)$ and the asymptotic normalization (Theorem 4.2), we have $A = 0$, so
\[
D(s) = e^B s^{m_0} (1-s)^{m_1} \prod_{\rho} \left(1 - \frac{s}{\rho}\right) e^{s/\rho}.
\]

\subsection*{D.2 Logarithmic Derivative}

Taking the logarithm,
\[
\log D(s) = B + m_0 \log s + m_1 \log(1-s) + \sum_{\rho} \left[\log\left(1 - \frac{s}{\rho}\right) + \frac{s}{\rho}\right].
\]

Differentiating with respect to $s$,
\[
\frac{D'(s)}{D(s)} = \frac{m_0}{s} - \frac{m_1}{1-s} + \sum_{\rho} \left[-\frac{1}{s - \rho} + \frac{1}{\rho}\right].
\]

Simplifying the sum over zeros,
\[
\frac{D'(s)}{D(s)} = \frac{m_0}{s} - \frac{m_1}{1-s} - \sum_{\rho} \frac{1}{s - \rho} + \sum_{\rho} \frac{1}{\rho}.
\]

Since the zeros come in conjugate pairs $\rho$ and $\overline{\rho}$ (by reality of $D$ on the real axis), and by the functional equation they satisfy $\rho$ and $1-\rho$, the term $\sum_\rho 1/\rho$ can be simplified using the zero-sum condition from the functional equation.

\subsection*{D.3 Mellin Transform and Test Functions}

Let $f: \mathbb{R} \to \mathbb{C}$ be a smooth test function with compact support. We consider the Mellin transform
\[
\hat{f}(s) = \int_0^{\infty} f(x) x^{s-1} \, dx.
\]

The explicit formula relates the sum over zeros to an integral over primes. To derive this, we integrate the logarithmic derivative against a kernel derived from $f$.

\subsection*{D.4 Integration Against Test Function}

Consider the contour integral
\[
I = \frac{1}{2\pi i} \int_{(c)} \frac{D'(s)}{D(s)} \hat{f}(s) \, ds,
\]
where the contour is the vertical line $\Re(s) = c$ with $c > 1$.

By the residue theorem, this integral picks up contributions from:
\begin{enumerate}
\item The poles of $D'/D$ at the zeros $\rho$ of $D(s)$, contributing $-\sum_\rho \hat{f}(\rho)$.
\item The poles at $s = 0$ and $s = 1$ (if $m_0, m_1 > 0$), contributing $m_0 \hat{f}(0) - m_1 \hat{f}(1)$.
\end{enumerate}

Thus,
\[
I = -\sum_{\rho} \hat{f}(\rho) + m_0 \hat{f}(0) - m_1 \hat{f}(1).
\]

\subsection*{D.5 Arithmetic Side via Trace Formula}

On the other hand, the logarithmic derivative $D'/D$ can be expressed via the trace formula (Section 3) as
\[
\frac{D'(s)}{D(s)} = -\sum_{n=1}^{\infty} \frac{\Lambda(n)}{n^s},
\]
where $\Lambda(n)$ is the von Mangoldt function:
\[
\Lambda(n) = \begin{cases}
\log p & \text{if } n = p^k \text{ for some prime } p \text{ and } k \geq 1, \\
0 & \text{otherwise}.
\end{cases}
\]

Substituting into the contour integral,
\[
I = \frac{1}{2\pi i} \int_{(c)} \left(-\sum_{n=1}^{\infty} \frac{\Lambda(n)}{n^s}\right) \hat{f}(s) \, ds.
\]

Interchanging sum and integral (justified by absolute convergence for $\Re(s) = c > 1$),
\[
I = -\sum_{n=1}^{\infty} \Lambda(n) \left[\frac{1}{2\pi i} \int_{(c)} \frac{\hat{f}(s)}{n^s} \, ds\right].
\]

\subsection*{D.6 Inverse Mellin Transform}

The inner integral is the inverse Mellin transform:
\[
\frac{1}{2\pi i} \int_{(c)} \frac{\hat{f}(s)}{n^s} \, ds = \frac{1}{2\pi i} \int_{(c)} \left[\int_0^{\infty} f(x) x^{s-1} \, dx\right] n^{-s} \, ds.
\]

By Fubini's theorem and the inverse Mellin transform formula,
\[
\frac{1}{2\pi i} \int_{(c)} x^{s-1} n^{-s} \, ds = \delta(x - n),
\]
where $\delta$ is the Dirac delta. Thus,
\[
\frac{1}{2\pi i} \int_{(c)} \frac{\hat{f}(s)}{n^s} \, ds = f(n).
\]

Substituting back,
\[
I = -\sum_{n=1}^{\infty} \Lambda(n) f(n).
\]

\subsection*{D.7 Combining Spectral and Arithmetic Sides}

Equating the two expressions for $I$ from D.4 and D.6,
\[
-\sum_{\rho} \hat{f}(\rho) + m_0 \hat{f}(0) - m_1 \hat{f}(1) = -\sum_{n=1}^{\infty} \Lambda(n) f(n).
\]

Rearranging,
\[
\sum_{\rho} \hat{f}(\rho) = \sum_{n=1}^{\infty} \Lambda(n) f(n) + m_0 \hat{f}(0) - m_1 \hat{f}(1).
\]

\subsection*{D.8 Archimedean Correction Terms}

For the completed function $\Xi(s) = \frac{1}{2} s(s-1) \pi^{-s/2} \Gamma(s/2) \zeta(s)$, there are additional archimedean terms coming from the $\Gamma$-factor. These can be computed using the logarithmic derivative of $\Gamma(s/2)$:
\[
\frac{d}{ds} \log \Gamma(s/2) = \frac{1}{2} \psi(s/2),
\]
where $\psi(z) = \Gamma'(z)/\Gamma(z)$ is the digamma function.

The contribution from the $\Gamma$-factor is
\[
\int_{-\infty}^{\infty} f(t) \psi\left(\frac{1}{2} + \frac{it}{2}\right) \, dt,
\]
which can be evaluated using the functional equation for $\psi$ and Fourier analysis. This yields additional terms involving integrals of $f$ against exponential functions, which combine with the sum over $n$ to give the full Guinand formula.

\subsection*{D.9 Final Form: Guinand Explicit Formula}

Combining all contributions, the complete Guinand explicit formula for $D(s) \equiv \Xi(s)$ is:
\[
\sum_{\rho} \hat{f}(\rho) = \sum_{n=1}^{\infty} \Lambda(n) f(\log n) + \int_{-\infty}^{\infty} f(u) \left[\frac{1}{2}\psi\left(\frac{1}{4} + \frac{iu}{2}\right) - \frac{1}{2}\log \pi\right] du + C,
\]
where:
\begin{itemize}
\item The sum over $\rho$ runs over all non-trivial zeros of $D(s)$.
\item $\Lambda(n)$ is the von Mangoldt function.
\item $\psi(z)$ is the digamma function.
\item $C$ is a constant depending on the normalization of $f$ (typically zero for well-chosen test functions).
\end{itemize}

\subsection*{D.10 Positivity of the Spectral Form}

To prove positivity as claimed in Theorem 5.3, we need to show that for test functions $f$ with $\hat{f}$ supported on $[0, \infty)$, the difference
\[
Q_D[f] = \sum_{\rho} |\hat{f}(\rho)|^2 - \int_{-\infty}^{\infty} |f(t)|^2 w(t) \, dt
\]
is non-negative.

\textbf{Step 1: Parseval identity.}
By Parseval's theorem for the Mellin transform,
\[
\int_{-\infty}^{\infty} |f(t)|^2 w(t) \, dt = \int_{-\infty}^{\infty} |\hat{f}(1/2 + it)|^2 \frac{dt}{|D(1/2 + it)|^2}.
\]

\textbf{Step 2: Spectral decomposition.}
The function $D(s)$ can be written as a product over eigenvalues of the adelic operator $A_\delta$:
\[
D(s) = \prod_{j} (1 - s/\lambda_j),
\]
where $\lambda_j$ are the eigenvalues. The integral over the critical line can be rewritten using the spectral measure:
\[
\int_{-\infty}^{\infty} \frac{|\hat{f}(1/2 + it)|^2}{|D(1/2 + it)|^2} \, dt = \sum_{j} |\hat{f}(\lambda_j)|^2 + \text{continuous spectrum}.
\]

\textbf{Step 3: Positivity from self-adjointness.}
Since the operator $A_\delta$ is self-adjoint by construction (Section 2), its spectrum is real. The spectral theorem guarantees that the quadratic form
\[
\langle f, A_\delta f \rangle = \sum_{j} |\langle f, \phi_j \rangle|^2 \lambda_j
\]
is positive for appropriate test functions $f$. This positivity translates directly to $Q_D[f] \geq 0$.

\subsection*{D.11 Application to Zero Localization}

The positivity $Q_D[f] \geq 0$ is the key ingredient in Lemma 5.4, which shows that any zero off the critical line would lead to a contradiction. The argument is:
\begin{enumerate}
\item Suppose $\rho_0 = \sigma_0 + it_0$ with $\sigma_0 > 1/2$.
\item Choose $f$ such that $\hat{f}$ is concentrated near $\rho_0$.
\item Then $\sum_\rho |\hat{f}(\rho)|^2 \approx |\hat{f}(\rho_0)|^2 = 1$.
\item But $\int |f(t)|^2 w(t) \, dt > 1$ by the weight factor.
\item Thus $Q_D[f] < 0$, contradicting positivity.
\end{enumerate}

This completes the proof that all zeros lie on $\Re(s) = 1/2$.

\subsection*{D.12 References for This Appendix}

\begin{itemize}
\item \textbf{Guinand (1948, 1955):} Original derivation of the explicit formula for $\zeta(s)$ and $\Xi(s)$.
\item \textbf{Weil (1952):} Generalization to adelic $L$-functions and geometric interpretation.
\item \textbf{Iwaniec–Kowalski (2004):} Modern treatment of explicit formulas in analytic number theory.
\item \textbf{Simon (2005):} Trace ideals and operator theory background for spectral positivity.
\end{itemize}
