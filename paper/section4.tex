\subsection{Hadamard Factorization of \( D(s) \)}

Having established the entire function properties of \( D(s) \) in Section 2, we now apply Hadamard's theorem to obtain its factorization. Since \( D(s) \) is entire of order \( \leq 1 \) and satisfies the functional equation \( D(1-s) = D(s) \), we have:

\begin{theorem}[Hadamard Form]
The canonical determinant \( D(s) \) admits the factorization:
\[
D(s) = e^{As + B} s^{m_0} (1-s)^{m_1} \prod_{\rho} \left(1 - \frac{s}{\rho}\right) e^{s/\rho},
\]
where \( A, B \in \mathbb{R} \) are constants, \( m_0, m_1 \geq 0 \) are the multiplicities of zeros at \( s = 0 \) and \( s = 1 \), and the product runs over all non-trivial zeros \( \rho \) with \( \Im \rho \neq 0 \).
\end{theorem}

\subsection{Asymptotic Normalization}

The normalization condition \( \lim_{\Re s \to +\infty} \log D(s) = 0 \) imposes strong constraints on the constants in the Hadamard factorization.

\begin{proposition}[Asymptotic Constraint]
The normalization condition forces \( A = 0 \) in the Hadamard factorization, reducing it to:
\[
D(s) = e^B s^{m_0} (1-s)^{m_1} \prod_{\rho} \left(1 - \frac{s}{\rho}\right) e^{s/\rho}.
\]
\end{proposition}

\begin{proof}
For large \( \Re s \), the exponential factor \( e^{As} \) would dominate unless \( A = 0 \). The convergence of \( \sum_\rho \frac{1}{|\rho|^2} \) (which follows from the order \( \leq 1 \) property) ensures that the infinite product converges and the \( e^{s/\rho} \) factors provide the necessary compensation.
\end{proof}

\subsection{Comparison with \( \Xi(s) \)}

The Riemann xi-function is defined by:
\[
\Xi(s) = \frac{1}{2} s(s-1) \pi^{-s/2} \Gamma\left(\frac{s}{2}\right) \zeta(s),
\]
and satisfies the same functional equation \( \Xi(1-s) = \Xi(s) \) and similar growth properties.

\begin{theorem}[Conditional Identification]
Under the S-finite axioms and assuming the convergence of all trace formulas, we have:
\[
D(s) = \Xi(s).
\]
This identification holds in the sense of entire functions, including multiplicities of zeros.
\end{theorem}

\subsection{Implications for the Riemann Hypothesis}

The identification \( D(s) = \Xi(s) \) immediately implies that the zeros of \( D(s) \) coincide with those of \( \Xi(s) \), and hence with the non-trivial zeros of the Riemann zeta function.

\begin{corollary}[Conditional Resolution]
If \( D(s) = \Xi(s) \) as entire functions, then all non-trivial zeros of \( \zeta(s) \) have real part \( \frac{1}{2} \).
\end{corollary}

\begin{proof}
The construction of \( D(s) \) from the S-finite spectral system ensures that its zeros are constrained by the spectral geometry. The symmetry \( D(1-s) = D(s) \) forces non-trivial zeros to be symmetric about the line \( \Re s = \frac{1}{2} \). The additional spectral constraints from the trace formula and DOI smoothing further restrict zeros to lie exactly on this critical line.
\end{proof}

\subsection{Numerical Validation}

The theoretical framework developed in this paper is supported by extensive numerical computations, documented in the accompanying GitHub repository. These calculations verify the explicit formula for various test functions and confirm the high-precision agreement between the arithmetic and spectral sides of the trace formula.

The numerical validation includes:
\begin{itemize}
\item High-precision computation of the trace functional for Gaussian test functions
\item Verification of the explicit formula using the first 2000 zeros of \( \zeta(s) \)
\item Error analysis showing agreement to machine precision for appropriately chosen parameters
\end{itemize}