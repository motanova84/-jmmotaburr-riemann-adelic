\section{Adelic Preliminaries}
\label{sec:preliminaries}

This section establishes the foundational adelic framework underlying our spectral construction. We introduce the key concepts without assuming prior knowledge of the Riemann zeta function or its properties.

\subsection{Places and Local Fields}

Let $\mathbb{Q}$ denote the field of rational numbers. A \textbf{place} of $\mathbb{Q}$ is an equivalence class of absolute values on $\mathbb{Q}$.

\begin{definition}[Places of $\mathbb{Q}$]
The places of $\mathbb{Q}$ consist of:
\begin{enumerate}
\item \textbf{Archimedean place} $v = \infty$: The standard absolute value $|x|_\infty = |x|$ for $x \in \mathbb{Q}$.
\item \textbf{Non-archimedean places} $v = p$: For each prime $p$, the $p$-adic absolute value
\[
|x|_p = p^{-v_p(x)},
\]
where $v_p(x)$ is the $p$-adic valuation (the exponent of $p$ in the prime factorization of $x$).
\end{enumerate}
\end{definition}

For each place $v$, the \textbf{completion} of $\mathbb{Q}$ with respect to $|\cdot|_v$ gives a local field:
\begin{itemize}
\item $\mathbb{Q}_\infty = \mathbb{R}$ (the real numbers)
\item $\mathbb{Q}_p$ (the $p$-adic numbers) for finite places $p$
\end{itemize}

\subsection{The Adele Ring}

\begin{definition}[Adele Ring]
The \textbf{adele ring} $\mathbb{A}_\mathbb{Q}$ is the restricted product:
\[
\mathbb{A}_\mathbb{Q} = \mathbb{R} \times \prod_{p}' \mathbb{Q}_p,
\]
where the prime $'$ indicates that for all but finitely many primes $p$, the component lies in $\mathbb{Z}_p$ (the $p$-adic integers).
\end{definition}

An element $a \in \mathbb{A}_\mathbb{Q}$ is a sequence $(a_\infty, a_2, a_3, a_5, \ldots)$ where $a_v \in \mathbb{Q}_v$ for each place $v$, and $a_p \in \mathbb{Z}_p$ for all but finitely many primes $p$.

The rational numbers $\mathbb{Q}$ embed diagonally into $\mathbb{A}_\mathbb{Q}$ via:
\[
x \mapsto (x, x, x, \ldots).
\]

\subsection{S-Finite Systems}

For practical and theoretical purposes, we work with \textbf{S-finite} restrictions.

\begin{definition}[S-Finite Adelic System]
\label{def:s_finite}
Let $S$ be a finite set of places including the archimedean place $\infty$. The \textbf{S-finite adelic ring} is:
\[
\mathbb{A}_S = \prod_{v \in S} \mathbb{Q}_v.
\]
\end{definition}

For example, if $S = \{\infty, 2, 3, 5\}$, then:
\[
\mathbb{A}_S = \mathbb{R} \times \mathbb{Q}_2 \times \mathbb{Q}_3 \times \mathbb{Q}_5.
\]

\textbf{Key observation}: The S-finite framework allows:
\begin{enumerate}
\item Concrete matrix constructions (finite-dimensional approximations)
\item Numerical validation with explicit computations
\item Control over trace-class convergence
\item Eventual passage to the full adelic limit
\end{enumerate}

\subsection{Local Measures and Haar Measure}

Each local field $\mathbb{Q}_v$ carries a canonical \textbf{Haar measure} $d\mu_v$, which is translation-invariant.

\begin{definition}[Local Haar Measures]
\begin{enumerate}
\item At $v = \infty$: $d\mu_\infty(x) = dx$ (Lebesgue measure on $\mathbb{R}$)
\item At $v = p$: Normalized so that $\mu_p(\mathbb{Z}_p) = 1$
\end{enumerate}
\end{definition}

The product measure on $\mathbb{A}_S$ is:
\[
d\mu_S = \prod_{v \in S} d\mu_v.
\]

\subsection{Local Norm and Canonical Scaling}

\begin{definition}[Local Norm]
For a finite place $v = p$ corresponding to a prime $p$, the \textbf{local norm} is:
\[
q_v = p^{f_v},
\]
where $f_v$ is the residue field degree. For $\mathbb{Q}_p$, we have $f_p = 1$, so $q_p = p$.
\end{definition}

\textbf{Critical quantity}: The local length scale is defined as:
\[
\ell_v = \log q_v.
\]

For $v = p$ a prime, this gives:
\[
\ell_p = \log p.
\]

\textbf{Important}: At this stage, $\ell_v$ is merely a \textbf{definition}. In Section~\ref{sec:local_length}, we prove that this quantity emerges \emph{geometrically} from closed orbit lengths, without circular reasoning involving $\zeta(s)$.

\subsection{Local Fourier Analysis (Tate's Theorem)}

The foundation of our construction relies on Tate's groundbreaking work on local Fourier analysis.

\begin{theorem}[Tate, 1950]
\label{thm:tate}
Let $v$ be a place of $\mathbb{Q}$, and let $\mathbb{Q}_v$ be the corresponding local field. There exists a unique (up to normalization) additive character $\psi_v: \mathbb{Q}_v \to \mathbb{C}^\times$ and a Haar measure $d\mu_v$ on $\mathbb{Q}_v$ such that the Fourier transform:
\[
\hat{f}(y) = \int_{\mathbb{Q}_v} f(x) \psi_v(xy) \, d\mu_v(x)
\]
satisfies the inversion formula:
\[
f(x) = \int_{\mathbb{Q}_v} \hat{f}(y) \psi_v(-xy) \, d\mu_v(y).
\]
Moreover, the local zeta integral:
\[
Z_v(f, s) = \int_{\mathbb{Q}_v^\times} f(x) |x|_v^s \, d^\times \mu_v(x)
\]
satisfies a functional equation relating $s$ to $1-s$.
\end{theorem}

\textbf{Significance}: Tate's theorem provides the local-to-global machinery without assuming global properties of $\zeta(s)$. Each place contributes independently via its local zeta integral.

\subsection{Global-Local Principle}

The philosophy of adelic methods is:
\begin{center}
\emph{Global properties arise from combining local contributions.}
\end{center}

In our framework:
\begin{itemize}
\item \textbf{Local}: Each place $v$ contributes a local operator $T_v$ acting on $L^2(\mathbb{Q}_v)$
\item \textbf{Global}: The global spectral determinant is constructed from:
\[
D(s) = \prod_{v \in S} D_v(s),
\]
where $D_v(s) = \det(I - s T_v)$ is the local spectral determinant.
\end{itemize}

\subsection{Why S-Finite Suffices}

A natural question: Why not work with the full adele ring $\mathbb{A}_\mathbb{Q}$?

\textbf{Answer}: 
\begin{enumerate}
\item \textbf{Trace-class control}: For $S$ finite, we can explicitly verify trace-class properties (see Appendix~\ref{app:trace_doi}).
\item \textbf{Numerical validation}: Concrete computations are feasible (Section~\ref{sec:numerics}).
\item \textbf{Approximation}: The S-finite case approximates the full adelic result as $|S| \to \infty$.
\item \textbf{Theoretical clarity}: All key phenomena (functional equation, zero localization) already manifest in the S-finite case.
\end{enumerate}

\subsection{Connection to Classical Number Theory}

While we do not assume $\zeta(s)$ or its Euler product, we note (for context) that classical theory gives:
\[
\zeta(s) = \prod_{p \text{ prime}} \frac{1}{1 - p^{-s}} \quad (\Re s > 1).
\]

Our construction will ultimately \emph{recover} this structure, but from first principles via spectral methods, not by assumption.

\subsection{Summary}

This section established:
\begin{itemize}
\item Places, local fields, and the adele ring
\item S-finite restrictions for computational tractability
\item Local Haar measures and canonical length scales $\ell_v = \log q_v$
\item Tate's theorem as the foundation for local Fourier analysis
\item The global-local philosophy underlying our construction
\end{itemize}

In the next section, we prove that $\ell_v = \log q_v$ emerges \emph{geometrically} from closed orbits, making the framework autonomous and non-circular.
