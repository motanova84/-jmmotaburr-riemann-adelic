\section{Adelic Preliminaries}

\subsection{The Ring of Adèles}

Let $K$ be a number field. For each place $v$ of $K$, let $K_v$ denote the completion of $K$ at $v$. The \textbf{ring of adèles} is defined as the restricted product:
\[
\mathbb{A}_K = \prod'_v K_v = \left\{ (x_v)_{v} \in \prod_v K_v : x_v \in \mathcal{O}_v \text{ for all but finitely many } v \right\},
\]
where $\mathcal{O}_v$ denotes the ring of integers in $K_v$ for non-archimedean places.

For the rational number field $K = \mathbb{Q}$, the places are:
\begin{itemize}
  \item The \textbf{archimedean place} $v = \infty$ with $\mathbb{Q}_\infty = \mathbb{R}$.
  \item The \textbf{non-archimedean places} $v = p$ (for each prime $p$) with $\mathbb{Q}_p$ the field of $p$-adic numbers.
\end{itemize}

\subsection{Local Absolute Values and Haar Measures}

At each place $v$, there is a normalized absolute value $|\cdot|_v$ satisfying the product formula:
\[
\prod_{v} |x|_v = 1 \quad \text{for all } x \in K^\times.
\]

For non-archimedean places $v = p$, we normalize:
\[
|p|_p = p^{-1}.
\]

Each local field $K_v$ carries a \textbf{Haar measure} $\mu_v$, unique up to scaling. For $\mathbb{Q}_p$, we normalize so that $\mu_p(\mathbb{Z}_p) = 1$. For $\mathbb{R}$, we take the Lebesgue measure.

By \textbf{Tate's theorem} \cite{tate1967}, the adelic Haar measure factorizes:
\[
d\mu_{\mathbb{A}} = \prod_v d\mu_v,
\]
and the adelic Fourier transform satisfies:
\[
\hat{\Phi}(y) = \int_{\mathbb{A}_K} \Phi(x) \psi(x \cdot y) \, d\mu(x),
\]
where $\psi$ is a non-trivial additive character on $\mathbb{A}_K/K$.

\subsection{Schwartz-Bruhat Functions}

The \textbf{Schwartz-Bruhat space} $\mathcal{S}(\mathbb{A}_K)$ consists of functions $\Phi: \mathbb{A}_K \to \mathbb{C}$ that:
\begin{itemize}
  \item Are smooth at archimedean places (rapidly decreasing in all derivatives).
  \item Are locally constant with compact support at non-archimedean places.
  \item Factorize as $\Phi = \prod_v \Phi_v$ with $\Phi_v \in \mathcal{S}(K_v)$ and $\Phi_v$ is the characteristic function of $\mathcal{O}_v$ for all but finitely many $v$.
\end{itemize}

The Fourier transform preserves $\mathcal{S}(\mathbb{A}_K)$ and satisfies the \textbf{adelic Poisson summation formula}:
\[
\sum_{\xi \in K} \Phi(\xi) = \sum_{\eta \in K} \hat{\Phi}(\eta).
\]

\subsection{S-Finite Restriction}

For a finite set $S$ of places containing all archimedean places and possibly finitely many non-archimedean places, we consider the \textbf{S-finite adelic ring}:
\[
\mathbb{A}_K^S = \prod_{v \in S} K_v \times \prod_{v \notin S} \mathcal{O}_v.
\]

This restriction serves two purposes:
\begin{enumerate}
  \item \textbf{Convergence}: Infinite products over all primes are replaced by finite products, ensuring trace-class properties.
  \item \textbf{Approximation}: As $S$ grows, $\mathbb{A}_K^S$ approximates $\mathbb{A}_K$, and spectral quantities converge via Kato-Seiler-Simon estimates.
\end{enumerate}

\subsection{Local Fields and Uniformizers}

For a non-archimedean place $v$ corresponding to a prime $p$ with residue field degree $f$ (i.e., $K_v$ is an extension of $\mathbb{Q}_p$ of degree $f$), the \textbf{local field} has residue field $\mathbb{F}_q$ where $q = p^f$. 

A \textbf{uniformizer} $\pi_v$ is a generator of the maximal ideal of $\mathcal{O}_v$, characterized by:
\[
|\pi_v|_v = q^{-1}.
\]

The \textbf{local factor} at $v$ in the Euler product (when it emerges from the spectral construction) is:
\[
\left(1 - q_v^{-s}\right)^{-1},
\]
where $q_v = q = p^f$.

\subsection{Tate's Thesis and Commutativity}

A fundamental result of \textbf{Tate's thesis} \cite{tate1967} is that local zeta integrals satisfy:
\[
\int_{K_v^\times} f_v(x) |x|_v^s \, d^\times x
\]
and these integrals commute with multiplication:
\[
U_v S_u = S_u U_v,
\]
where $U_v$ and $S_u$ are the local and scaling operators, respectively. This commutativity is essential for defining the global spectral determinant.
