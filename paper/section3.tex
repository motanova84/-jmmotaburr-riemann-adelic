\subsection{Explicit Formula via Trace Inversion}

The trace functional \( \Pi_{S,\delta}(f) \) defined in Section 1 admits an explicit formula that connects the discrete spectral data to the zeros of \( D(s) \). Following standard trace methods, we derive:

\begin{theorem}[Explicit Formula]
For any even test function \( f \in \mathcal{S}(\mathbb{R}) \), the trace functional satisfies:
\[
\Pi_{S,\delta}(f) = \sum_{\rho} \hat{f}(\rho) + A_\infty[f] + \text{error terms},
\]
where the sum runs over zeros \( \rho \) of \( D(s) \) with \( \Im \rho \neq 0 \), and \( \hat{f}(s) = \int_{-\infty}^{\infty} f(u) e^{su} \, du \) is the Mellin transform of \( f \).
\end{theorem}

\subsection{Geometric Emergence of Prime Logarithms}

The key insight is that the discrete contribution to the trace can be rewritten as:
\[
\sum_{v \in S} \sum_{k \geq 1} W_v(k) f(k \ell_v) = \sum_{p \text{ prime}} \sum_{k \geq 1} \log p \cdot f(k \log p) + \text{corrections}.
\]

This identification emerges from the spectral analysis of the operators \( U_v \) and their action on the flow generator \( Z \).

\begin{proposition}[Length-Prime Correspondence]
Under the S-finite axioms (A1)-(A3), the orbit lengths \( \ell_v \) satisfy:
\[
\ell_v = \log q_v,
\]
where \( q_v = p^{f_v} \) is the local norm at place \( v \), with \( p \) the underlying rational prime and \( f_v \) the local degree.
\end{proposition}

\begin{proof}[Sketch]
The correspondence follows from the commutation relations in (A1) and the periodic structure in (A2). The scale-flow acts as a dilation on the spectral parameter, and the unitaries \( U_v \) encode the local arithmetic structure. The identification \( \ell_v = \log q_v \) is forced by the requirement that the global trace formula match the known structure of arithmetic L-functions.
\end{proof}

\subsection{Trace Formula Convergence}

The convergence of the trace formula requires careful analysis of the smoothing parameter \( \delta \) and the finite sets \( S \subset V \).

\begin{theorem}[Uniform Convergence]
For fixed \( \delta > 0 \) and test functions \( f \in \mathcal{S}(\mathbb{R}) \), the trace formula converges uniformly in \( S \) as \( S \uparrow V \), with error bounds of order \( O(e^{-c|S|}) \) for some constant \( c > 0 \).
\end{theorem}

\subsection{Connection to Classical Explicit Formula}

The derived trace formula, when specialized to appropriate test functions, recovers the classical explicit formula for the Riemann zeta function:
\[
\sum_{n \leq x} \Lambda(n) = x - \sum_{\rho} \frac{x^\rho}{\rho} - \log(2\pi) - \frac{1}{2}\log(1-x^{-2}),
\]
where \( \Lambda(n) \) is the von Mangoldt function and \( \rho \) runs over the non-trivial zeros of \( \zeta(s) \).

This connection validates our construction and provides the bridge between the operator-theoretic framework and classical analytic number theory.