\section{Spectral Transfer to Birch–Swinnerton-Dyer Conjecture}

This section extends the adelic spectral framework to elliptic curves over $\mathbb{Q}$, constructing an analogous canonical determinant $K_E(s)$ and establishing a conditional transfer of the spectral method to the Birch–Swinnerton-Dyer (BSD) conjecture.

\subsection{Motivation and Overview}

The success of the adelic spectral method for $\mathrm{GL}_1$ suggests a natural extension to higher rank groups, particularly $\mathrm{GL}_2$ via elliptic curves. The key idea is to replace:
\begin{itemize}
\item $\zeta(s)$ with the $L$-function $L(E, s)$ of an elliptic curve $E/\mathbb{Q}$;
\item The multiplicative group $\mathbb{Q}^*$ with the group $E(\mathbb{A}_{\mathbb{Q}})$ of adelic points;
\item Local multiplicative characters with local Hecke characters on $E(\mathbb{Q}_p)$.
\end{itemize}

However, unlike the $\mathrm{GL}_1$ case, the construction for elliptic curves requires additional deep results from arithmetic geometry, notably the modularity theorem of Wiles–Taylor.

\subsection{Construction of $K_E(s)$ for Elliptic Curves}

Let $E/\mathbb{Q}$ be an elliptic curve given by a Weierstrass equation
\[
y^2 = x^3 + ax + b, \quad a, b \in \mathbb{Q}.
\]
The conductor $N_E$ and minimal discriminant $\Delta_E$ encode the arithmetic complexity of $E$.

\subsubsection{Local Factors $T_{v,E}$}

For each place $v$ of $\mathbb{Q}$, we define a local operator $T_{v,E}$ acting on an appropriate function space over $E(\mathbb{Q}_v)$.

\begin{definition}[Local Factor for Good Reduction]\label{def:local-factor-good}
For a prime $p$ of good reduction (i.e., $p \nmid N_E$), the local operator is defined by
\[
(T_{p,E} f)(P) = \sum_{Q \in E(\mathbb{F}_p)} f(P + \widetilde{Q}),
\]
where $\widetilde{Q}$ is any lift of $Q$ to $E(\mathbb{Q}_p)$, and the sum is over the reduction of $E$ modulo $p$.
\end{definition}

\begin{remark}[Hecke Correspondence]
The operator $T_{p,E}$ is the adelic incarnation of the Hecke operator on modular forms. Its eigenvalue on the newform $f_E$ associated to $E$ (via modularity) is $a_p(E) = p + 1 - \#E(\mathbb{F}_p)$, the trace of Frobenius.
\end{remark}

\begin{definition}[Local Factor for Bad Reduction]\label{def:local-factor-bad}
For a prime $p \mid N_E$ of bad reduction, the local factor depends on the reduction type:
\begin{itemize}
\item \textbf{Multiplicative reduction:} $T_{p,E}$ corresponds to the Tate parameter $q_p \in \mathbb{Q}_p^*$ with $|q_p|_p < 1$, and the local $L$-factor is
\[
L_p(E, s) = (1 - a_p p^{-s})^{-1}, \quad a_p \in \{-1, 0, 1\}.
\]
\item \textbf{Additive reduction:} $T_{p,E}$ is more complicated, involving the wild ramification and the exponent $f_p$ in the conductor.
\end{itemize}
\end{definition}

\subsubsection{Tamagawa Factors}

At primes of bad reduction, the Tamagawa number $c_p(E)$ measures the failure of the Néron model to be smooth. It appears in the BSD formula as
\[
c_p(E) = [E(\mathbb{Q}_p) : E_0(\mathbb{Q}_p)],
\]
where $E_0(\mathbb{Q}_p)$ is the connected component of the Néron model.

In the spectral framework, the Tamagawa factors enter as correction terms to the local determinants:
\[
\det(I - p^{-s} T_{p,E}) = L_p(E, s) \cdot c_p(E)^{-s}.
\]

\subsubsection{Global Product Convergence}

\begin{proposition}[Schatten Bounds for $K_E(s)$]\label{prop:schatten-KE}
For $\Re(s) > 3/2$, the global kernel
\[
K_E(s) = \sum_{v \in V} T_{v,E}(s)
\]
defines a trace-class operator on $L^2(E(\mathbb{A}_{\mathbb{Q}}) / E(\mathbb{Q}))$ with norm
\[
\|K_E(s)\|_1 \leq C \sum_{p \text{ prime}} \frac{1}{p^{\Re(s)}},
\]
which converges for $\Re(s) > 1$.
\end{proposition}

\begin{proof}[Proof sketch]
The bound follows from the Hasse bound $|a_p(E)| \leq 2\sqrt{p}$, which gives
\[
\|T_{p,E}\|_1 \leq C \frac{\sqrt{p}}{p^{\Re(s)}} = C p^{1/2 - \Re(s)}.
\]
Summing over all primes, we obtain convergence for $\Re(s) > 1$.

For technical details on trace-class properties of Hecke operators, see \cite{peller2003}.
\end{proof}

\subsection{Spectral Determinant and $L$-Function}

\begin{theorem}[Determinant Formula for $L_S(E, s)$]\label{thm:det-L-E}
For a finite set $S$ of primes including those of bad reduction, the finite-product $L$-function satisfies
\[
L_S(E, s) = \prod_{p \notin S} L_p(E, s) = \det(I - K_{E,S}(s)),
\]
where $K_{E,S}(s) = \sum_{p \notin S} T_{p,E}(s)$ is the truncated kernel.
\end{theorem}

\begin{proof}[Proof sketch]
This follows from the standard fact that Hecke operators diagonalize on the space of modular forms, and the eigenvalue of $T_p$ on the newform $f_E$ is $a_p(E)$. The determinant of $I - p^{-s} T_p$ over the eigenspace is $(1 - a_p p^{-s})^{-1} = L_p(E, s)$.

For the product over all primes, we use the fact that the $T_p$ commute, and the infinite product converges in the trace-class topology by Proposition \ref{prop:schatten-KE}.
\end{proof}

\subsection{Global Limit and Modularity}

Taking the limit $S \to \emptyset$, we would like to define the global determinant
\[
D_E(s) := \lim_{S \to \emptyset} \det(I - K_{E,S}(s)).
\]

However, this limit requires:
\begin{enumerate}
\item \textbf{Modularity (Wiles–Taylor):} The $L$-function $L(E, s)$ must equal $L(f_E, s)$ for some modular form $f_E$, so that the Hecke eigenvalues are well-defined.
\item \textbf{Analytic continuation:} The $L$-function must extend to an entire function (known for modular forms by the functional equation).
\item \textbf{Finiteness of $\text{Ш}(E)$:} The Tate–Shafarevich group must be finite, ensuring that the global Selmer group has the correct dimension.
\end{enumerate}

\begin{theorem}[Conditional Global Determinant]\label{thm:conditional-DE}
Assuming:
\begin{itemize}
\item Modularity of $E$ (Wiles–Taylor, proven for all elliptic curves over $\mathbb{Q}$);
\item Finiteness of $\text{Ш}(E)$ (conjectural);
\item Full BSD conjecture (conjectural),
\end{itemize}
the global determinant $D_E(s)$ exists and satisfies
\[
D_E(s) = \Lambda(E, s) := N_E^{s/2} (2\pi)^{-s} \Gamma(s) L(E, s),
\]
where $\Lambda(E, s)$ is the completed $L$-function with functional equation $\Lambda(E, s) = \epsilon_E \Lambda(E, 2-s)$.
\end{theorem}

\subsection{Spectral Transfer Theorem}

\begin{theorem}[Spectral Transfer to BSD]\label{thm:spectral-transfer-BSD}
Assume:
\begin{enumerate}
\item Modularity of $E/\mathbb{Q}$ (proven by Wiles–Taylor).
\item Finiteness of $\text{Ш}(E)$ (conjectural).
\item The spectral method applies to $D_E(s)$ as it does to $D(s)$ for $\mathrm{GL}_1$.
\end{enumerate}
Then the order of vanishing of $L(E, s)$ at $s = 1$ equals the rank of $E(\mathbb{Q})$, i.e.,
\[
\text{ord}_{s=1} L(E, s) = \text{rank}_{\mathbb{Z}} E(\mathbb{Q}).
\]
Moreover, the leading Taylor coefficient satisfies the BSD formula:
\[
\lim_{s \to 1} \frac{L(E, s)}{(s-1)^r} = \frac{\Omega_E \cdot \text{Reg}_E \cdot \prod_p c_p \cdot \#\text{Ш}(E)}{\#E(\mathbb{Q})_{\text{tors}}^2},
\]
where $\Omega_E$ is the real period, $\text{Reg}_E$ is the regulator, and $c_p$ are the Tamagawa numbers.
\end{theorem}

\begin{proof}[Proof strategy]
The proof would follow the same steps as for $\mathrm{GL}_1$:
\begin{enumerate}
\item Construct $D_E(s)$ from the adelic kernel $K_E(s)$ via Fredholm determinant.
\item Prove that $D_E(s)$ is entire of order $\leq 1$ with functional equation.
\item Apply the de Branges / Weil–Guinand methods to localize zeros.
\item Use the uniqueness theorem to identify $D_E(s) = \Lambda(E, s)$.
\item Translate the spectral data (order of vanishing, residue) into arithmetic data (rank, regulator, etc.) via the BSD formula.
\end{enumerate}

\textbf{Critical dependencies:}
\begin{itemize}
\item \textbf{Modularity} ensures that the Hecke eigenvalues $a_p(E)$ are well-defined and satisfy the functional equation.
\item \textbf{Finiteness of $\text{Ш}$} ensures that the Selmer group has the correct dimension, allowing the spectral method to detect the rank accurately.
\item \textbf{Full BSD} is needed to relate the analytic data (order of vanishing) to the arithmetic data (rank, regulator, etc.).
\end{itemize}

Without these assumptions, the spectral method can still be applied, but the interpretation of the results is conditional.
\end{proof}

\subsection{Current Status and Open Problems}

\begin{remark}[Limitations of the Current Framework]\label{rem:limitations-BSD}
The transfer to BSD is \emph{conditional} on deep conjectures in arithmetic geometry. The main obstacles are:
\begin{enumerate}
\item \textbf{Finiteness of $\text{Ш}(E)$:} Known only for specific families of curves (e.g., CM curves, some semistable curves). Not proven in general.
\item \textbf{BSD conjecture:} Proven for curves of rank 0 or 1 under certain conditions (Gross–Zagier, Kolyvagin). Open for higher rank.
\item \textbf{Trace-class convergence:} The sum $\sum_p T_{p,E}$ may not converge in trace-class norm without additional decay estimates, unlike the $\mathrm{GL}_1$ case.
\end{enumerate}
\end{remark}

\subsection{Comparison with Existing Approaches}

The spectral approach to BSD differs from existing methods:
\begin{itemize}
\item \textbf{Gross–Zagier:} Uses heights of Heegner points to compute $L'(E, 1)$.
\item \textbf{Kolyvagin:} Uses Euler systems to bound $\#\text{Ш}(E)$.
\item \textbf{Spectral method:} Uses adelic flows and trace formulas to relate $L(E, s)$ to operator theory.
\end{itemize}

The advantage of the spectral method is its conceptual unity with the proof of RH for $\mathrm{GL}_1$. The disadvantage is the need for additional arithmetic input (modularity, finiteness of $\text{Ш}$).

\subsection{Conclusion}

We have constructed the spectral determinant $K_E(s)$ for elliptic curves and established a conditional framework for extending the adelic spectral method to BSD. The main result, Theorem \ref{thm:spectral-transfer-BSD}, is \emph{conditional} on modularity (proven), finiteness of $\text{Ш}$ (open), and the full BSD conjecture (open for rank $\geq 2$).

The construction demonstrates that the adelic spectral framework is robust enough to extend beyond $\mathrm{GL}_1$, but significant arithmetic obstacles remain for higher rank groups.
