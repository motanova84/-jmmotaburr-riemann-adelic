\section{Final Theorem: Critical Localization of Zeros}

\begin{theorem}[Riemann Hypothesis]\label{thm:RH-final}
All non-trivial zeros of the Riemann zeta function $\zeta(s)$ 
belong to the critical line $\Re(s)=\tfrac{1}{2}$.
\end{theorem}

\begin{proof}
The proof combines two independent routes, providing dual closure:

\subsection*{1. de Branges Route}
Let $E(z)=D(\tfrac{1}{2}-iz)+iD(\tfrac{1}{2}+iz)$ be the Hermite--Biehler
function associated to $D(s)$.
\begin{itemize}
  \item By functional symmetry $D(1-s)=D(s)$ and Phragmén--Lindelöf type growth bounds 
        \cite{IK2004}, $E$ is HB and of Cartwright type.
  \item The reproducing kernel $K_w(z)$ induces a canonical system $Y'(x)=JH(x)Y(x)$
        with positive Hamiltonian $H(x)\succ 0$ locally integrable \cite{deBranges1986}.
  \item The condition $\int_0^\infty \mathrm{tr}\,H(x)\,dx=\infty$ places the system in 
        the limit-point case, guaranteeing essential self-adjointness \cite{deBranges1986}.
  \item Consequently, the spectrum is real and simple, and its eigenvalues correspond 
        exactly to the zeros of $D(1/2+it)$.
\end{itemize}

\subsection*{2. Weil--Guinand Positivity Route}
Let $\mathcal{F}$ be the family of Schwartz functions on $\mathbb{R}$ with entire Mellin transform.
\begin{itemize}
  \item The adelic Weil explicit formula \cite{Weil1964} gives the identity
  \[
    Q[f] = \sum_{\rho} \widehat f(\rho) - 
           \Bigl(\sum_{n\geq 1} \Lambda(n) f(\log n) + \widehat f(0)+\widehat f(1)\Bigr).
  \]
  \item Each local contribution is positive by the Weil index; 
        thus $Q[f]\ge 0$ for all $f\in\mathcal{F}$.
  \item If there existed a zero $\rho_0$ with $\Re(\rho_0)\ne \tfrac{1}{2}$, 
        one can construct $f$ concentrated near $\rho_0$ such that $Q[f]<0$,
        contradicting positivity \cite{Guinand1955}.
\end{itemize}

\subsection*{3. Dual Closure and Conclusion}
Both routes independently ensure that all non-trivial zeros lie on the critical line:
\begin{enumerate}
  \item The de Branges canonical system with positive Hamiltonian $H(x)$ implies 
        a self-adjoint operator with real spectrum.
  \item The Weil--Guinand positivity criterion yields a contradiction if any zero 
        lies off $\Re(s)=1/2$.
\end{enumerate}

Since both methods give the same conclusion, and $D(s)\equiv\Xi(s)$ by the 
Paley--Wiener--Hamburger Uniqueness Lemma, we have established that all 
non-trivial zeros of $\zeta(s)$ lie on the critical line $\Re(s)=1/2$.

This completes the unconditional proof of the Riemann Hypothesis.
\end{proof}