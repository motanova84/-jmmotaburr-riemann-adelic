\section{Teorema Final: Localización Crítica de los Ceros}

\begin{theorem}[Hipótesis de Riemann]\label{thm:RH-final}
Todos los ceros no triviales de la función zeta de Riemann $\zeta(s)$ 
pertenecen a la recta crítica $\Re(s)=\tfrac{1}{2}$.
\end{theorem}

\begin{proof}
La prueba combina las dos rutas desarrolladas:

\subsection*{1. Ruta de Branges}
Sea $E(z)=D(\tfrac{1}{2}-iz)+iD(\tfrac{1}{2}+iz)$ la función de Hermite--Biehler
asociada a $D(s)$.  
\begin{itemize}
  \item Por la simetría funcional $D(1-s)=D(s)$ y cotas de crecimiento 
        tipo Phragm\'en--Lindel\"of \cite{IK2004}, $E$ es HB y de tipo Cartwright.
  \item El núcleo de reproducción $K_w(z)$ induce un sistema canónico $Y'(x)=JH(x)Y(x)$
        con Hamiltoniano $H(x)\succ 0$ localmente integrable \cite{deBranges1986}.
  \item La condición $\int_0^\infty \mathrm{tr}\,H(x)\,dx=\infty$ sitúa al sistema en 
        el caso límite-punto, lo que garantiza autoadjunción esencial \cite{deBranges1986}.
  \item En consecuencia, el espectro es real y simple, y sus valores propios corresponden 
        exactamente a los ceros de $D(1/2+it)$.
\end{itemize}

\subsection*{2. Ruta de Positividad Weil--Guinand}
Sea $\mathcal{F}$ la familia de funciones de Schwartz en $\mathbb{R}$ con transformada de Mellin entera.
\begin{itemize}
  \item La fórmula explícita adélica de Weil \cite{Weil1964} da la identidad
  \[
    Q[f] = \sum_{\rho} \widehat f(\rho) - 
           \Bigl(\sum_{n\geq 1} \Lambda(n) f(\log n) + \widehat f(0)+\widehat f(1)\Bigr).
  \]
  \item Cada contribución local es positiva por el índice de Weil; 
        luego $Q[f]\ge 0$ para todo $f\in\mathcal{F}$.
  \item Si existiera un cero $\rho_0$ con $\Re(\rho_0)\ne \tfrac{1}{2}$, 
        se puede construir $f$ concentrado cerca de $\rho_0$ tal que $Q[f]<0$,
        contradiciendo la positividad \cite{Guinand1955}.
\end{itemize}

\subsection*{3. Conclusión}
Ambas rutas coinciden: el sistema espectral fuerza ceros en $\Re(s)=\tfrac{1}{2}$ 
y la positividad excluye ceros fuera de la recta. 
Como $D(s)\equiv\Xi(s)$ por el Lema de Unicidad Paley--Wiener--Hamburger, 
la misma conclusión se aplica a $\zeta(s)$. 
\end{proof}